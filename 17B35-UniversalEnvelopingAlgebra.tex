\documentclass[12pt]{article}
\usepackage{pmmeta}
\pmcanonicalname{UniversalEnvelopingAlgebra}
\pmcreated{2013-03-22 13:03:35}
\pmmodified{2013-03-22 13:03:35}
\pmowner{draisma}{715}
\pmmodifier{draisma}{715}
\pmtitle{universal enveloping algebra}
\pmrecord{7}{33466}
\pmprivacy{1}
\pmauthor{draisma}{715}
\pmtype{Definition}
\pmcomment{trigger rebuild}
\pmclassification{msc}{17B35}
\pmclassification{msc}{16S30}
\pmrelated{LieAlgebra}
\pmrelated{PoincareBirkhoffWittTheorem}
\pmrelated{WeylAlgebra}
\pmrelated{FreeLieAlgebra}

\endmetadata

\usepackage{amssymb}
\usepackage{amsmath}
\usepackage{amsfonts}
\usepackage[all]{xypic}
\begin{document}
A {\em universal enveloping algebra} of a Lie algebra $\mathfrak{g}$ over
a field $k$ is an associative \PMlinkid{algebra}{Algebra} $U$ (with unity) over $k$, together
with a Lie algebra homomorphism $\iota:\mathfrak{g} \rightarrow U$ (where
the Lie algebra structure on $U$ is given by the commutator), such that
if $A$ is a another associative algebra over $k$ and $\phi:\mathfrak{g}
\rightarrow A$ is another Lie algebra homomorphism, then there exists a
unique homomorphism $\psi:U \rightarrow A$ of associative algebras such
that the diagram
\[\xymatrix{
  \mathfrak{g} \ar[dr]_\phi \ar[r]^\iota & U \ar[d]^\psi\\
                                         & A}
\]
commutes. Any $\mathfrak{g}$ has a universal enveloping algebra: let
$T$ be the associative tensor algebra generated by the vector space
$\mathfrak{g}$, and let $I$ be the two-sided ideal of $T$ generated by
elements of the form
\[ xy-yx-[x,y] \text{ for } x,y \in \mathfrak{g}; \]
then $U=T/I$ is a universal enveloping algebra of $\mathfrak{g}$.
Moreover, the universal property above ensures that all universal
enveloping algebras of $\mathfrak{g}$ are canonically isomorphic; this
justifies the standard notation $U(\mathfrak{g})$.

Some remarks:
\begin{enumerate}
\item By the Poincar\'e-Birkhoff-Witt theorem, the map $\iota$ is
injective; usually $\mathfrak{g}$ is identified with
$\iota(\mathfrak{g})$. From the construction above it is clear that
this space generates $U(\mathfrak{g})$ as an associative algebra with
unity.
\item By definition, the (left) representation theory of $U(\mathfrak{g})$
is identical to that of $\mathfrak{g}$. In particular, any irreducible
$\mathfrak{g}$-module corresponds to a maximal left ideal of
$U(\mathfrak{g})$.
\end{enumerate}
Example: let $\mathfrak{g}$ be the Lie algebra generated by the
elements $p,q,$ and $e$ with Lie bracket determined by $[p,q]=e$ and
$[p,e]=[q,e]=0$. Then $U(g)/(e-1)$ (where $(e-1)$ denotes the two-sided
ideal generated by $e-1$) is isomorphic to the skew polynomial algebra
$k[x,\frac{\partial}{\partial x}]$, the isomorphism being determined by
\begin{align*}
        p + (e-1) &\mapsto \frac{\partial}{\partial x} \text{ and}\\
        q + (e-1) &\mapsto x.
\end{align*}
%%%%%
%%%%%
\end{document}
