\documentclass[12pt]{article}
\usepackage{pmmeta}
\pmcanonicalname{Octonion}
\pmcreated{2013-03-22 15:21:42}
\pmmodified{2013-03-22 15:21:42}
\pmowner{CWoo}{3771}
\pmmodifier{CWoo}{3771}
\pmtitle{octonion}
\pmrecord{15}{37185}
\pmprivacy{1}
\pmauthor{CWoo}{3771}
\pmtype{Definition}
\pmcomment{trigger rebuild}
\pmclassification{msc}{17A75}
\pmclassification{msc}{17D05}
\pmsynonym{Cayley algebra}{Octonion}
\pmrelated{TheoremsOnSumsOfSquares}
\pmrelated{DivisionAlgebra}
\pmdefines{octonion algebra}

\endmetadata

\usepackage{amssymb,amscd}
\usepackage{amsmath}
\usepackage{amsfonts}
\usepackage{tabls}
% used for TeXing text within eps files
%\usepackage{psfrag}
% need this for including graphics (\includegraphics)
%\usepackage{graphicx}
% for neatly defining theorems and propositions
%\usepackage{amsthm}
% making logically defined graphics
%%%\usepackage{xypic}

% define commands here

\newcommand{\R}{\mathbb{R}}
\newcommand{\C}{\mathbb{C}}
\newcommand{\Q}{\mathbb{H}}
\newcommand{\Oc}{\mathbb{O}}
\newcommand{\conj}[1]{\overline{#1}}
\newcommand{\norm}[1]{\lVert #1 \rVert}
\newcommand{\abs}[1]{\lvert #1 \rvert}
\begin{document}
Let $\Q$ be the quaternions over the reals $\R$.  Apply the
Cayley-Dickson construction to $\Q$ once, and we obtain an algebra,
variously called \emph{Cayley algebra}, \emph{the octonion algebra}, or simply \emph{the octonions}, over $\R$.  Specifically the construction is carried out as follows:
\begin{enumerate}
\item Form the vector space $\Oc=\Q\oplus\Q\mathbf{k}$; any element of
$\Oc$ can be written as $a+b\mathbf{k}$, where $a,b\in\Q$;
\item Define a binary operation on $\Oc$ called \emph{the multiplication in
$\Oc$} by
$$(a+b\mathbf{k})(c+d\mathbf{k}):=(ac-\conj{d}b)+(da+b\conj{c})\mathbf{k},$$
where $a,b,c,d\in\Q$, and $\conj{c}$ is the quaternionic conjugation
of $c\in\Q$. When $b=d=0$, the multiplication is reduced the
multiplication in $\Q$. In addition, the multiplication rule above
imply the following:
\begin{eqnarray}
a(d\mathbf{k})=(da)\mathbf{k} \\
(b\mathbf{k})c=(b\conj{c})\mathbf{k} \\
(b\mathbf{k})(d\mathbf{k})=-\conj{d}b.
\end{eqnarray}
In particular, in the last equation, if $b=d=1$, $\mathbf{k}^2=-1$.
\item Define a unary operation on $\Oc$ called \emph{the octonionic
conjugation in $\Oc$} by
$$\overline{a+b\mathbf{k}}:=\conj{a}-b\mathbf{k},$$  where $a,b\in\Q$.
Clearly, the octonionic conjugation is an \PMlinkname{involution}{Involution2}
($\overline{\conj{x}}=x$).
\item Finally, define a unary operation $N$ on $\Oc$ called \emph{the norm in
$\Oc$} by $N(x):=x\conj{x}$, where $x\in\Oc$. Write
$x=a+b\mathbf{k}$, then $$N(x)=(a+b\mathbf{k})(\conj{a}-b\mathbf{k})
=(a\conj{a}+\conj{b}b)+(-ba+b\conj{\conj{a}})\mathbf{k}=
a\conj{a}+b\conj{b}\ge0.$$
It is not hard to see that $N(x)=0$ iff $x=0$.
\end{enumerate}
The above four (actually, only the first two suffice) steps makes
$\Oc$ into an $8$-dimensional algebra over $\R$ such that $\Q$ is
embedded as a subalgebra.
\\\\
With the last two steps, one can define the inverse of a non-zero
element $x\in\Oc$ by $$x^{-1}:=\frac{\conj{x}}{N(x)}$$ so that
$xx^{-1}=x^{-1}x=1$.  Since $x$ is arbitrary, $\Oc$ has no zero
divisors.  Upon checking that $x^{-1}(xy)=y=(yx)x^{-1}$, the non-associative algebra $\Oc$ is turned into a division algebra.
\\\\
Since $N(x)\ge0$ for any $x\in\Oc$, we can define a non-negative
real-valued function $\norm{\cdot}$ on $\Oc$ by $\norm{x}=
\sqrt{N(x)}$. This is clearly well-defined and $\norm{x}=0$ iff
$x=0$.  In addition, it is not hard to see that, for any $r\in\R$
and $x\in\Oc$, $\norm{rx}= \abs{r}\norm{x}$, and that $\norm{\cdot}$
satisfies the triangular inequality. This makes $\Oc$ into a normed
division algebra.
\\\\
Since the multiplication in $\Q$ is noncommutative, $\Oc$ is
noncommutative.  In fact, if we write $\Q=\C\oplus\C\mathbf{j}$,
where $\C$ are the complex numbers and $\mathbf{j}^2=-1$, then
$B=\lbrace 1,\mathbf{i},\mathbf{j},\mathbf{ij}\rbrace$ is a basis
for the vector space $\Q$ over $\mathbb{R}$.  With the introduction
of $\mathbf{k}\in\Oc$, we quickly check that $\mathbf{k}$
anti-commute with the non-real basis elements in $B$:
$$\mathbf{ik=-ki},\qquad\mathbf{jk=-kj},\qquad\mathbf{(ij)k=-k(ij)}.$$
Furthermore, one checks that $\mathbf{i(jk)=(ji)k=-(ij)k}$, so that
$\Oc$ is not associative.
\\\\
Since $\Oc=\Q\oplus\Q\mathbf{k}$, the set $\lbrace
1,\mathbf{i,j,ij,k,ik,jk,(ij)k}\rbrace$($=B\cup B\mathbf{k}$) is a
basis for $\Oc$ over $\R$. A less messy way to represent these basis
elements is done the following assignment:
\begin{center}
\begin{tabular}{|c|c|c|c|c|c|c|c|c|}
\hline basis element & $1$ & $\mathbf{i}$ & $\mathbf{j}$ &
$\mathbf{ij}$ & $\mathbf{k}$ & $\mathbf{ik}$ & $\mathbf{jk}$ &
$\mathbf{(ij)k}$\\
\hline basis element rewritten & $\mathbf{i_0}$ & $\mathbf{i_1}$ &
$\mathbf{i_2}$ & $\mathbf{i_3}$ & $\mathbf{i_4}$ & $\mathbf{i_5}$
& $\mathbf{i_6}$ & $\mathbf{i_7}$ \\
\hline
\end{tabular}
\end{center}
Any element $x$ of $\Oc$ can thus be expressed uniquely as
$\sum_{n=0}^{7}r_n\mathbf{i_n}$, where $r_n\in\R$.  Using Equations
(1)-(3) above, one can form a multiplication table for these basis
elements $i_n$'s:
\begin{center}
\begin{tabular}{|c|c|c|c|c|c|c|c|}
\hline row$\times$column & $\mathbf{i_1}$ & $\mathbf{i_2}$ & $\mathbf{i_3}$ & $\mathbf{i_4}$ & $\mathbf{i_5}$ & $\mathbf{i_6}$ & $\mathbf{i_7}$ \\
\hline $\mathbf{i_1}$ & -1 & $\mathbf{i_3}$ & -$\mathbf{i_2}$ & $\mathbf{i_5}$ & -$\mathbf{i_4}$ & -$\mathbf{i_7}$ & $\mathbf{i_6}$ \\
\hline $\mathbf{i_2}$ & -$\mathbf{i_3}$ & -1 & $\mathbf{i_1}$ & $\mathbf{i_6}$ & $\mathbf{i_7}$ & -$\mathbf{i_4}$ & -$\mathbf{i_5}$ \\
\hline $\mathbf{i_3}$ & $\mathbf{i_2}$ & -$\mathbf{i_1}$ & -1 & $\mathbf{i_7}$ & -$\mathbf{i_6}$ & $\mathbf{i_5}$ & -$\mathbf{i_4}$ \\
\hline $\mathbf{i_4}$ & -$\mathbf{i_5}$ & -$\mathbf{i_6}$ & -$\mathbf{i_7}$ & -1 & $\mathbf{i_1}$ & $\mathbf{i_2}$ & $\mathbf{i_3}$ \\
\hline $\mathbf{i_5}$ & $\mathbf{i_4}$ & -$\mathbf{i_7}$ & $\mathbf{i_6}$ & -$\mathbf{i_1}$ & -1 & -$\mathbf{i_3}$ & $\mathbf{i_2}$ \\
\hline $\mathbf{i_6}$ & $\mathbf{i_7}$ & $\mathbf{i_4}$ & -$\mathbf{i_5}$ & -$\mathbf{i_2}$ & $\mathbf{i_3}$ & -1 & -$\mathbf{i_1}$ \\
\hline $\mathbf{i_7}$ & -$\mathbf{i_6}$ & $\mathbf{i_5}$ & $\mathbf{i_4}$ & -$\mathbf{i_3}$ & -$\mathbf{i_2}$ & $\mathbf{i_1}$ & -1 \\
\hline
\end{tabular}
\end{center}
Other well known properties of the octonions are
\begin{enumerate}
\item $\conj{xy}=\conj{y}\hspace{2pt}\conj{x}$ for any $x,y\in\Oc$.
\item $N(xy)=N(x)N(y)$ so that $\Oc$ is a composition algebra.  It also proves that the product of sums of eight squares is a sum of eight squares.
\item $\Oc$ is an alternative algebra.  As a result, any two elements of $\Oc$ generate an associative algebra.  If fact, the algebra is isomorphic of one of $\R$, $\C$, and $\Q$.  This is the consequence of \PMlinkname{Artin's Theorem}{ArtinsTheoremOnAlternativeAlgebras}.
\end{enumerate}
%%%%%
%%%%%
\end{document}
