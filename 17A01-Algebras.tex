\documentclass[12pt]{article}
\usepackage{pmmeta}
\pmcanonicalname{Algebras}
\pmcreated{2013-03-22 16:27:20}
\pmmodified{2013-03-22 16:27:20}
\pmowner{Algeboy}{12884}
\pmmodifier{Algeboy}{12884}
\pmtitle{algebras}
\pmrecord{7}{38613}
\pmprivacy{1}
\pmauthor{Algeboy}{12884}
\pmtype{Definition}
\pmcomment{trigger rebuild}
\pmclassification{msc}{17A01}
%\pmkeywords{algebra}
\pmrelated{ring}
\pmrelated{NonAssociativeAlgebra}
\pmrelated{FreeAssociativeAlgebra}
\pmrelated{TopicEntryOnTheAlgebraicFoundationsOfMathematics}
\pmdefines{algebra}

\usepackage{latexsym}
\usepackage{amssymb}
\usepackage{amsmath}
\usepackage{amsfonts}
\usepackage{amsthm}

%%\usepackage{xypic}

%-----------------------------------------------------

%       Standard theoremlike environments.

%       Stolen directly from AMSLaTeX sample

%-----------------------------------------------------

%% \theoremstyle{plain} %% This is the default

\newtheorem{thm}{Theorem}

\newtheorem{coro}[thm]{Corollary}

\newtheorem{lem}[thm]{Lemma}

\newtheorem{lemma}[thm]{Lemma}

\newtheorem{prop}[thm]{Proposition}

\newtheorem{conjecture}[thm]{Conjecture}

\newtheorem{conj}[thm]{Conjecture}

\newtheorem{defn}[thm]{Definition}

\newtheorem{remark}[thm]{Remark}

\newtheorem{ex}[thm]{Example}



%\countstyle[equation]{thm}



%--------------------------------------------------

%       Item references.

%--------------------------------------------------


\newcommand{\exref}[1]{Example-\ref{#1}}

\newcommand{\thmref}[1]{Theorem-\ref{#1}}

\newcommand{\defref}[1]{Definition-\ref{#1}}

\newcommand{\eqnref}[1]{(\ref{#1})}

\newcommand{\secref}[1]{Section-\ref{#1}}

\newcommand{\lemref}[1]{Lemma-\ref{#1}}

\newcommand{\propref}[1]{Prop\-o\-si\-tion-\ref{#1}}

\newcommand{\corref}[1]{Cor\-ol\-lary-\ref{#1}}

\newcommand{\figref}[1]{Fig\-ure-\ref{#1}}

\newcommand{\conjref}[1]{Conjecture-\ref{#1}}


% Normal subgroup or equal.

\providecommand{\normaleq}{\unlhd}

% Normal subgroup.

\providecommand{\normal}{\lhd}

\providecommand{\rnormal}{\rhd}
% Divides, does not divide.

\providecommand{\divides}{\mid}

\providecommand{\ndivides}{\nmid}


\providecommand{\union}{\cup}

\providecommand{\bigunion}{\bigcup}

\providecommand{\intersect}{\cap}

\providecommand{\bigintersect}{\bigcap}










\begin{document}
Let $K$ be a commutative unital ring (often a field) and $A$ a $K$-module.  
Given a bilinear mapping $b:A\times A\rightarrow A$, we say $(K,A,b)$ is a $K$-algebra.  We usually write only $A$ for the tuple $(K,A,b)$.

\begin{remark}
Many authors and applications insist on $K$ as a field, or at least a local
ring, or a semisimple ring.  This enables $A$ to have some notion of dimension
or rank.
\end{remark}

This definition is a compact method to encode the property that our multiplication is distributive: the multiplication is additive in both variables translates to
  \[(a+b)c=ac+bc,\qquad a(b+c)=ab+ac\qquad a,b,c\in A.\]
Furthermore, the assumption that scalars can be passed in and out of the bilinear
product translates to
  \[(la)b=l(ab)=a(lb),\qquad a,b\in A, l\in K.\]

Perhaps the most important outcome of these two axioms of an algebra is the 
opportunity to express polynomial like equations over the algebra.  Without the distributive axiom we cannot establish connections between addition and multiplication.  Without scalar multiplication we cannot describe coefficients.
With these equations we can define certain subalgebras, for example we 
see both axioms at work in

\begin{prop}
Given an algebra $A$, the set 
\[Z_0(A)=\{z\in A: za=az, a\in A\}.\]
$Z_0(A)$ is a submodule of $A$.
\end{prop}
\begin{proof}
For now let elements of $A$ be denoted with $\hat{a}$ to distinguish them
from scalars.  As a module $0\hat{a}=\hat{0}$ for all $a\in A$.  Then
\[\hat{0}\hat{a}=(0\hat{a})\hat{a}=(\hat{a})(0\hat{a})=\hat{a}\hat{0}.\]
So $\hat{0}\in Z_0(A)$.

Also given $\hat{z},\hat{w}\in Z_0(A)$ then for all $a\in A$,
\[(\hat{z}+\hat{w})\hat{a}=\hat{z}\hat{a}+\hat{w}\hat{a}
    =\hat{a}\hat{z}+\hat{a}\hat{w}=\hat{a}(\hat{z}+\hat{w}).\]
So $\hat{z}+\hat{w}\in A$.  

Finally, given $l\in K$ we have 
\[(l\hat{z})\hat{a}=l(\hat{z}\hat{a})=l(\hat{a}\hat{z})=\hat{a}(l\hat{z}).\]
\end{proof}

Although this set $Z(A)$ appears like a reasonable object to define as the 
center of an algebra, it is usually preferable to produce a subalgebra, not
simply a submodule, and for this we need elements that can be regrouped in products associatively, that is, that lie in the nucleus.  So the center is commonly defined as
\[Z(A)=\{z\in A: za=az, z(ab)=(za)b, a(zb)=(az)b, (ab)z=a(bz), a,b\in A\}.\]

When the algebra $A$ has an identity (unity) $1$ then we can go further to identify $K$ as a subalgebra of $A$ by $l1$.  Then we see this subalgebra is
necessarily in the center of $A$.  As a converse, given a unital ring $R$ (associativity is necessary), the center of the ring forms a commutative unital subring over which $R$ is an algebra.  In this way unital rings and associative unital algebras are often interchanged.


%%%%%
%%%%%
\end{document}
