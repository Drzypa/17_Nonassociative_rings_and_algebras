\documentclass[12pt]{article}
\usepackage{pmmeta}
\pmcanonicalname{DerivationsOnARingOfContinuousFunctions}
\pmcreated{2013-03-22 18:37:25}
\pmmodified{2013-03-22 18:37:25}
\pmowner{joking}{16130}
\pmmodifier{joking}{16130}
\pmtitle{derivations on a ring of continuous functions}
\pmrecord{14}{41359}
\pmprivacy{1}
\pmauthor{joking}{16130}
\pmtype{Example}
\pmcomment{trigger rebuild}
\pmclassification{msc}{17A36}
\pmclassification{msc}{16W25}
\pmclassification{msc}{13N15}

% this is the default PlanetMath preamble.  as your knowledge
% of TeX increases, you will probably want to edit this, but
% it should be fine as is for beginners.

% almost certainly you want these
\usepackage{amssymb}
\usepackage{amsmath}
\usepackage{amsfonts}

% used for TeXing text within eps files
%\usepackage{psfrag}
% need this for including graphics (\includegraphics)
%\usepackage{graphicx}
% for neatly defining theorems and propositions
%\usepackage{amsthm}
% making logically defined graphics
%%%\usepackage{xypic}

% there are many more packages, add them here as you need them

% define commands here

\begin{document}
Let $X$ be a topological space and denote by $\mathbb{R}$ the set of reals. Of course the set of all continuous functions $C(X,\mathbb{R})$ is a $\mathbb{R}$-algebra. Let $c\in\mathbb{R}$. By the symbol $\bar{c}$ we will denote constant function at $c$, i.e. $\bar{c}:X\to\mathbb{R}$ is defined by $\bar{c}(x)=c$.

\textbf{Proposition.} If $D:C(X,\mathbb{R})\to C(X,\mathbb{R})$ is a $\mathbb{R}$-derivation, then $D(x)=\bar{0}$ for any $x\in C(X,\mathbb{R})$.

\textit{Proof.} Step one. We will prove that $D(\bar{c})=\bar{0}$ for any $c\in\mathbb{R}$. Indeed $$D(\bar{1})=D(\bar{1}\cdot \bar{1})=\bar{1}\cdot D(\bar{1})+\bar{1}\cdot D(\bar{1})=D(\bar{1})+D(\bar{1})=2\cdot D(\bar{1})$$
and thus $D(\bar{1})=\bar{0}$. Now from linearity of $D$ we obtain that $$D(\bar{c})=D(c\cdot\bar{1})=c\cdot D(\bar{1})=c\cdot\bar{0}=\bar{0}.$$ 

Step two. If $f:X\to\mathbb{R}$ is continuous and $c\in\mathbb{R}$, then $f+\bar{c}$ is continuous and obviously $D(f)=D(f+\bar{c}).$ Moreover, if $x\in X$ then $D(f-\overline{f(x)})=D(f)$, but $(f-\overline{f(x)})(x)=0$. Thus we may assume that $f(x)=0$ for fixed $x\in X$.

Let $x_0\in X$. Now we will restrict only to such maps $f:X\to\mathbb{R}$ that $f(x_0)=0$.

Step three. We now decompose $f$ into sum of two nonnegative functions. Indeed, if $f:X\to\mathbb{R}$ is continuous, then define $f^{+},f^{-}:X\to\mathbb{R}$ by the formula:
$$f^{+}(x)=\mathrm{max}(f(x),0);\ \ \ f^{-}(x)=\mathrm{max}(-f(x),0).$$
Of course both $f^{-}$ and $f^{+}$ are continous, nonnegative and $f=f^{+}-f^{-}$. Thus $$D(f)=D(f^{+})-D(f^{-}),$$ so it is enough to show that $D(f)=\bar{0}$ only for nonnegative and continuous functions.

Step four. Assume that $f:X\to\mathbb{R}$ is nonnegative, continuous and $f(x_0)=0$. Then there exists $g:X\to\mathbb{R}$ continuous such that $g^2=f$ (indeed $g=\sqrt{f}$ and it is well defined, continuous map, because $f$ was nonnegative). Then we have $$D(f)=D(g^2)=g\cdot D(g)+g\cdot D(g)=2\cdot g\cdot D(g).$$ Now we have $g(x_0)=\sqrt{f(x_0)}=0$ and thus $$D(f)(x_0)=2\cdot g(x_0)\cdot D(g)(x_0)=0.$$

Now we can take any $x\in X$ and repeat steps two, three and four to get that for any $x\in X$ we have $$D(f)(x)=0$$ and thus $$D(f)=\bar{0},$$ which completes the proof. $\square$

\textbf{Remark.} Note that this proof cannot be repeated if we (for example) consider the set of all smooth functions $C^{\infty}(M,\mathbb{R})$ on a smooth manifold $M$, because $f^{+}$, $f^{-}$ and $\sqrt{f}$ need not be smooth.
%%%%%
%%%%%
\end{document}
