\documentclass[12pt]{article}
\usepackage{pmmeta}
\pmcanonicalname{CompositionAlgebrasOverFiniteFields}
\pmcreated{2013-03-22 17:18:26}
\pmmodified{2013-03-22 17:18:26}
\pmowner{Algeboy}{12884}
\pmmodifier{Algeboy}{12884}
\pmtitle{composition algebras over finite fields}
\pmrecord{9}{39655}
\pmprivacy{1}
\pmauthor{Algeboy}{12884}
\pmtype{Theorem}
\pmcomment{trigger rebuild}
\pmclassification{msc}{17A75}
\pmrelated{HurwitzsTheorem}
\pmrelated{JacobsonsTheoremOnCompositionAlgebras}

\usepackage{latexsym}
\usepackage{amssymb}
\usepackage{amsmath}
\usepackage{amsfonts}
\usepackage{amsthm}

%%\usepackage{xypic}

%-----------------------------------------------------

%       Standard theoremlike environments.

%       Stolen directly from AMSLaTeX sample

%-----------------------------------------------------

%% \theoremstyle{plain} %% This is the default

\newtheorem{thm}{Theorem}

\newtheorem{coro}[thm]{Corollary}

\newtheorem{lem}[thm]{Lemma}

\newtheorem{lemma}[thm]{Lemma}

\newtheorem{prop}[thm]{Proposition}

\newtheorem{conjecture}[thm]{Conjecture}

\newtheorem{conj}[thm]{Conjecture}

\newtheorem{defn}[thm]{Definition}

\newtheorem{remark}[thm]{Remark}

\newtheorem{ex}[thm]{Example}



%\countstyle[equation]{thm}



%--------------------------------------------------

%       Item references.

%--------------------------------------------------


\newcommand{\exref}[1]{Example-\ref{#1}}

\newcommand{\thmref}[1]{Theorem-\ref{#1}}

\newcommand{\defref}[1]{Definition-\ref{#1}}

\newcommand{\eqnref}[1]{(\ref{#1})}

\newcommand{\secref}[1]{Section-\ref{#1}}

\newcommand{\lemref}[1]{Lemma-\ref{#1}}

\newcommand{\propref}[1]{Prop\-o\-si\-tion-\ref{#1}}

\newcommand{\corref}[1]{Cor\-ol\-lary-\ref{#1}}

\newcommand{\figref}[1]{Fig\-ure-\ref{#1}}

\newcommand{\conjref}[1]{Conjecture-\ref{#1}}


% Normal subgroup or equal.

\providecommand{\normaleq}{\unlhd}

% Normal subgroup.

\providecommand{\normal}{\lhd}

\providecommand{\rnormal}{\rhd}
% Divides, does not divide.

\providecommand{\divides}{\mid}

\providecommand{\ndivides}{\nmid}


\providecommand{\union}{\cup}

\providecommand{\bigunion}{\bigcup}

\providecommand{\intersect}{\cap}

\providecommand{\bigintersect}{\bigcap}










\begin{document}
\begin{thm}
There are 5 non-isomorphic composition algebras over a finite field $k$ of characteristic not 2, 
2 division algebras and 3 split algebras.
\begin{enumerate}
\item The field $k$.
\item The unique quadratic extension field $K/k$.
\item The \emph{exchange} algebra: $k\oplus k$.
\item $2\times 2$ matrices over $k$: $M_2(k)$.
\item The split Cayley algebra.
\end{enumerate}
\end{thm}
\begin{proof}
Following Hurwitz's theorem every composition algebra is given by the Cayley-Dickson construction
and has dimension 1,2, 4 or 8.  Now we consider the possible non-degenerate 
quadratic forms of these dimensions.

Since every anisotropic 2 space corresponds to a quadratic field extension, and our field is finite,
it follows that there is at most one anisotropic 2 subspace of our quadratic form.  Therefore
if $\dim C>2$ then the quadratic form is isotropic and so the algebra is a split.
Therefore in the Cayley-Dickson construction over a finite field there every quaternion 
algebra is split, thus $M_2(k)$.  To build the non-associative division Cayley algebra of dimension 8 requires 
we start the Cayley-Dickson construction with a division ring which is not a field, and thus there are no 
Cayley division algebras over finite fields.
\end{proof}

This result also can be seen as a consequence of Wedderburn's theorem that every finite division ring 
is a field.  
Likewise, a theorem of Artin and Zorn asserts that every finite alternative division ring is in fact associative,
thus excluding the Cayley algebras in a fashion similar to how Wedderburn's theorem excludes division
quaternion algebras.

%%%%%
%%%%%
\end{document}
