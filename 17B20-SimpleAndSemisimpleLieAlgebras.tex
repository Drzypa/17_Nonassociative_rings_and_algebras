\documentclass[12pt]{article}
\usepackage{pmmeta}
\pmcanonicalname{SimpleAndSemisimpleLieAlgebras}
\pmcreated{2013-03-22 13:11:28}
\pmmodified{2013-03-22 13:11:28}
\pmowner{mathcam}{2727}
\pmmodifier{mathcam}{2727}
\pmtitle{simple and semi-simple Lie algebras}
\pmrecord{9}{33644}
\pmprivacy{1}
\pmauthor{mathcam}{2727}
\pmtype{Definition}
\pmcomment{trigger rebuild}
\pmclassification{msc}{17B20}
\pmrelated{LieAlgebra}
\pmrelated{LieGroup}
\pmrelated{RootSystem}
\pmrelated{RootSystemUnderlyingASemiSimpleLieAlgebra}
\pmdefines{simple Lie algebra}
\pmdefines{semi-simple Lie algebra}
\pmdefines{semisimple Lie algebra}
\pmdefines{simple}
\pmdefines{semi-simple}
\pmdefines{semisimple}

\endmetadata

\usepackage{amssymb}
\usepackage{amsmath}
\usepackage{amsfonts}

\newcommand{\fr}[1]{\mathfrak{#1}}
\def\C{\mathbb{C}}
\def\R{\mathbb{R}}
\begin{document}
\PMlinkescapeword{abelian}

A Lie algebra is called {\em simple} if it has no proper ideals and is not abelian. A Lie algebra
is called {\em semi-simple} if it has no proper solvable ideals and is not abelian.

Let $k=\R$ or $\C$. Examples of simple algebras are $\fr{sl}_nk$, the Lie algebra
of the special linear group (traceless matrices), $\fr{so}_nk$, the Lie algebra of the special
orthogonal group (skew-symmetric matrices), and $\fr{sp}_{2n} k$ the Lie algebra of the symplectic group. Over $\R$, there are other simple Lie algebas, such as $\fr{su}_n$, the Lie algebra of the special unitary group
(skew-Hermitian matrices). Any
semi-simple Lie algebra is a direct product of simple Lie algebras.

Simple and semi-simple Lie algebras are one of the most widely studied classes of algebras
for a number of reasons. First of all, many of the most interesting Lie groups have semi-simple
Lie algebras. Secondly, their representation theory is very well understood. Finally, there is
a beautiful classification of simple Lie algebras.

Over $\C$, there are 3 infinite series of simple Lie algebras: $\fr{sl}_n$, $\fr {so}_n$ and
$\fr{sp}_{2n}$, and 5 exceptional simple Lie algebras $\fr g_2,\fr f_4,\fr e_6,\fr e_7$, and $\fr e_8$.
Over $\R$ the picture is more complicated, as several different Lie algebras can have the same complexification (for example, $\fr{su}_n$ and $\fr{sl}_n\R$ both have complexification $\fr{sl}_n\C$).
%%%%%
%%%%%
\end{document}
