\documentclass[12pt]{article}
\usepackage{pmmeta}
\pmcanonicalname{IsotropyRepresentation}
\pmcreated{2013-03-22 12:42:28}
\pmmodified{2013-03-22 12:42:28}
\pmowner{rmilson}{146}
\pmmodifier{rmilson}{146}
\pmtitle{isotropy representation}
\pmrecord{6}{32992}
\pmprivacy{1}
\pmauthor{rmilson}{146}
\pmtype{Definition}
\pmcomment{trigger rebuild}
\pmclassification{msc}{17B10}
\pmrelated{AdjointRepresentation}

\endmetadata

\newcommand{\lag}{\mathfrak{g}}
\newcommand{\lah}{\mathfrak{h}}
\newcommand{\ad}{\mathop{\mathrm{ad}}\nolimits}
\usepackage{amsmath}
\usepackage{amsfonts}
\usepackage{amssymb}
\newcommand{\reals}{\mathbb{R}}
\newcommand{\natnums}{\mathbb{N}}
\newcommand{\cnums}{\mathbb{C}}
\newcommand{\znums}{\mathbb{Z}}
\newcommand{\lp}{\left(}
\newcommand{\rp}{\right)}
\newcommand{\lb}{\left[}
\newcommand{\rb}{\right]}
\newcommand{\supth}{^{\text{th}}}
\newtheorem{proposition}{Proposition}
\newtheorem{definition}[proposition]{Definition}
\newcommand{\nl}[1]{\PMlinkescapetext{{#1}}}
\newcommand{\pln}[2]{\PMlinkname{#1}{#2}}
\begin{document}
Let $\lag$ be a Lie algebra, and $\lah\subset\lag$ a subalgebra.  The
isotropy representation of $\lah$ relative to $\lag$ is the naturally
defined action of $\lah$ on the quotient vector space $\lag/\lah$.

Here is a synopsis of the technical details. As is customary, we will
use  
$$b+\lah,\, b\in\lag$$
to denote the coset elements of $\lag/\lah$.
Let $a\in\lah$ be given.  Since $\lah$ is invariant with respect to
$\ad_\lag(a)$, the adjoint action factors through the quotient to
give a well defined endomorphism of $\lag/\lah$. The action is given
by
$$b+\lah \mapsto [a,b]+\lah,\quad b\in\lag.$$
This is the action
alluded to in the first paragraph.
%%%%%
%%%%%
\end{document}
