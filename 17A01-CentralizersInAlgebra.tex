\documentclass[12pt]{article}
\usepackage{pmmeta}
\pmcanonicalname{CentralizersInAlgebra}
\pmcreated{2013-03-22 17:22:30}
\pmmodified{2013-03-22 17:22:30}
\pmowner{Algeboy}{12884}
\pmmodifier{Algeboy}{12884}
\pmtitle{centralizers in algebra}
\pmrecord{14}{39740}
\pmprivacy{1}
\pmauthor{Algeboy}{12884}
\pmtype{Definition}
\pmcomment{trigger rebuild}
\pmclassification{msc}{17A01}
\pmsynonym{centraliser}{CentralizersInAlgebra}
\pmrelated{CentralizerOfASubsetOfAGroup}
\pmrelated{LieAlgebra}
\pmdefines{centralizer}
\pmdefines{center}
\pmdefines{central}
\pmdefines{centraliser}
\pmdefines{additive commutator}

\endmetadata

\usepackage{latexsym}
\usepackage{amssymb}
\usepackage{amsmath}
\usepackage{amsfonts}
\usepackage{amsthm}

%%\usepackage{xypic}

%-----------------------------------------------------

%       Standard theoremlike environments.

%       Stolen directly from AMSLaTeX sample

%-----------------------------------------------------

%% \theoremstyle{plain} %% This is the default

\newtheorem{thm}{Theorem}

\newtheorem{coro}[thm]{Corollary}

\newtheorem{lem}[thm]{Lemma}

\newtheorem{lemma}[thm]{Lemma}

\newtheorem{prop}[thm]{Proposition}

\newtheorem{conjecture}[thm]{Conjecture}

\newtheorem{conj}[thm]{Conjecture}

\newtheorem{defn}[thm]{Definition}

\newtheorem{remark}[thm]{Remark}

\newtheorem{ex}[thm]{Example}



%\countstyle[equation]{thm}



%--------------------------------------------------

%       Item references.

%--------------------------------------------------


\newcommand{\exref}[1]{Example-\ref{#1}}

\newcommand{\thmref}[1]{Theorem-\ref{#1}}

\newcommand{\defref}[1]{Definition-\ref{#1}}

\newcommand{\eqnref}[1]{(\ref{#1})}

\newcommand{\secref}[1]{Section-\ref{#1}}

\newcommand{\lemref}[1]{Lemma-\ref{#1}}

\newcommand{\propref}[1]{Prop\-o\-si\-tion-\ref{#1}}

\newcommand{\corref}[1]{Cor\-ol\-lary-\ref{#1}}

\newcommand{\figref}[1]{Fig\-ure-\ref{#1}}

\newcommand{\conjref}[1]{Conjecture-\ref{#1}}


% Normal subgroup or equal.

\providecommand{\normaleq}{\unlhd}

% Normal subgroup.

\providecommand{\normal}{\lhd}

\providecommand{\rnormal}{\rhd}
% Divides, does not divide.

\providecommand{\divides}{\mid}

\providecommand{\ndivides}{\nmid}


\providecommand{\union}{\cup}

\providecommand{\bigunion}{\bigcup}

\providecommand{\intersect}{\cap}

\providecommand{\bigintersect}{\bigcap}










\begin{document}
\section{Abstract definitions and properties}

\begin{defn}
Let $S$ be a set with a binary operation $*$.  Let $T$ be a subset of $S$.  Then define 
the \emph{centralizer} in $S$ of $T$ as the subset
\begin{equation*}
   C_S(T)=\{ s\in S : s*t=t*s, \textnormal{ for all } t\in T\}.
\end{equation*}
The \emph{center} of $S$ is defined as $C_S(S)$.  This is commonly denoted $Z(S)$ where $Z$ is derived from the German word \emph{zentral}.  Subsets and elements of the center 
are called \emph{central}.
\end{defn}

If we regard $*:S\times S\to S$ in the \PMlinkescapetext{language} of actions we can perscribe a left
action $s*t$ and a right action $t*s$.  The centralizer is thus the set of elements
for which the left regular action and the right regular action agree when \PMlinkescapetext{restricted}
to $T$.  

It is generally possible to have $s*t$ not lie in $T$ for $s\in C_S(T)$ and
$t\in T$, and likewise, it is also possible that if $s,s'\in C_S(T)$ that $s*s'\neq s'*s$.
Therefore it should not be presumed that the centralizer is central.

With further axioms on the \PMlinkescapetext{type} of operation we can deduce certain natural \PMlinkescapetext{properties}
for the set $C_S(T)$.

\begin{prop}
\begin{enumerate}
\item If $A\subseteq B$, then $C_S(B)\subseteq C_S(A)$.  In particular, $C_S(\varnothing)=S$.  
\item If $S$ has an identity then $C_S(T)$ is non-empty.  In particular, in this case $Z(S)$ is non-empty. \footnote{An identity of $S$ is an element $e\in S$ such that $e*s=s*e$ for all $s\in S$.}
\item If $S$ is associative and $s,s'\in C_S(T)$ then $s*s'\in C_S(T)$,
we say then that $C_S(T)$ is closed to the binary operation of $S$.
\item If $s\in C_S(T)$ and $s$ has an (strong) inverse $s^{-1}$, then
$s^{-1}\in C_S(T)$.\footnote{We say an inverse is strong if $s^{-1}*(s*t)=t=(t*s)*s^{-1}$ for
all $t\in S$.  If the operation is associative then this is given for free.  There
are natural nonassociative operations with this property, such as alternative algebras.}
\item If $S$ is commutative then $C_S(T)=S$.
\item If $T$ is a subset of the center of $S$ then $C_S(T)=S$.
\end{enumerate}
\end{prop}

Note that it is possible for $C_S(T)$ be a subset closed to the opertaion without the 
assumption of associativity, as for example, when $S$ is commutative.


\section{Centralizers in groups}

In the category of groups the centralizer in a group $G$ of a subset $H$ can be redefined as:
\begin{equation*}
   C_G(H)=\{ g\in G : g^{-1} h g=h, \textnormal{ for all } h\in H\}.
\end{equation*}
If one regards conjugation as a group action $h^g:=g^{-1} hg$ then it follows that the
centralizer is the same as the pointwise stabilizer in $G$ of $H$, where the action is
of $G$ on itself by conjugation.  Because of this overlap, in some contexts the \PMlinkescapetext{term}
centralizers is applied to the pointwise stabilizer of a set on which a group acts, \PMlinkescapetext{even}
though this context no longer refers to the action of conjugation.  This
is espeically common when there is a need to distinguish between the pointwise stabilizer
and the setwise stabilizer.

In this category, the centralizer is always a subgroup of $G$.  Furthermore, if $H$ is a normal
subgroup of $G$, then so too is $C_G(H)$.

\section{Centralizers in rings and algebras}

For \PMlinkescapetext{uniformity} we treat rings as algebras over $\mathbb{Z}$ and now speak only of
algebras, which will include nonassociative examples.

In an algebra $A$ there is in fact two binary operations on the set $A$ in
question.  Thus the abstract definition of the centralizer is ambiguous.  However, the additive
operation of rings and algebras is always commutative and so any centralizer with respect to this
operation is the \PMlinkescapetext{entire} set $A$.  Thus it is generally accepted practice to assume that centralizers
in this context always refer to the multiplicative operation.  In this way we have the following
properties.

\begin{prop}
Given an algebra $A$ over a commutative unital ring $R$ and a subset $B$ of $A$, then
\begin{enumerate}
\item $C_A(B)$ is a submodule of $A$.
\item If $A$ is associative then $C_A(B)$ is a subalgebra.
\item $Z(A)\leq C_A(B)$, in particular, if $A$ has a 1 then $1\in C_A(B)$ and so
$R$ embeds in $C_A(B)$.
\end{enumerate}
\end{prop}


\textbf{Remarks}.  
\begin{itemize}
\item
A centralizer in an algebra is also called a commutant.  This terminology is mostly used in algebras of operators in functional analysis.
\item
Let $R$ be a ring (or an algebra).  For every ordered pair $(a,b)$ of elements of $R$, we can define the \emph{additive commutator} of $(a,b)$ to be the element $ab-ba$, written $[a,b]$.  With this, one may alternatively define the centralizer of a set $S\subseteq R$ in a ring $R$ as $$C_R(S):=\{ r\in R\mid [r,s]=0 \textnormal{ for all }s\in S \}.$$  Of course, in this definition, two operations (multiplication and subtraction) are needed instead of one.  But the nice aspect about this definition is that one can ``measure'' commutativity of a ring by the additive commutation operation.  For example, one can show that, in a division ring, if every element additively commutes with every additive commutator, then the ring must be a field.
\end{itemize}

\section{Centralizers in Lie algebras}

Suppose $\mathfrak{g}$ is a Lie aglebra over a commutative ring of characteristic not $2$.
Given a subset $T$ of $\mathfrak{g}$, then $[s,t]=-[t,s]$ for $s\in \mathfrak{g}$ and $t\in T$ from the axioms
of a Lie algebra multiplication.  Therefore whenever $[s,t]=[t,s]$ it follows that
$-[t,s]=[t,s]$ so that $[t,s]=0$.  This motivates the more common redefinition of 
the centralizer in a Lie algebra:
\begin{equation*}
   C_{\mathfrak{g}}(T)=\{ s\in \mathfrak{g} : [s,t]=0, \textnormal{ for all } t\in T\}.
\end{equation*}
Despite the incongruety in characteristic 2, this new definition replaces the original
definition of centralizers for Lie algebras.  The centralizer of a Lie algebra is a subalgebra.

When the Lie multiplication is regarded as a commutator, so $[a,b]=ab-ba$, for example
it it universal enveloping algebra, then $0=[a,b]=ab-ba$ is the same as $ab=ba$ and so
the centralizer of the Lie algebra coincides with the centralizer of the associative 
envelope.


\section{Centralizers in other nonassociative algebras}

The centralizer need not be a subalgebra on account of the lack of associativity.
There are instances of non-associative algebras where the centralizer is however
a subalgebra nontheless, for example, Lie algebras as seen above.  In travial fashion, if
an algebra is commutative then $C_A(B)=A$ and so the centralizer is a subalgebra
but without any useful properties.  There is a suitable additional constraint
to add to centralizers to force them to be subalgebras and carry with them
more useful \PMlinkescapetext{information} in the commutative but nonassociative setting.

We write $[a,b]$ for $ab-ba$, called the commutator in $A$ of $a,b\in A$ and also write
$/a,b,c/$ for $(ab)c-a(bc)$ and call it the associator in $A$ of $a,b,c\in A$.\footnote{This notation
for associators is non-standard but the standard $[a,b,c]=(ab)c-a(bc)$ is likely confusing given the 
usual commutator notation used already.}  Then we
can redefine the centralizer in $A$ of a subset $T$ of $A$ as
\begin{equation*}
   C_{A}(T)=\{ s\in A: [s,t]=0, /s',s,t/=/s',t,s/=/t,s',s/=0\textnormal{ for all } t\in T, s'\in A\}.
\end{equation*}
It follows that $C_A(T)$ is a subalgebra of $A$ on account of the added associator condition
which forces the subset to be closed to the product.

In alternative algebras, if any one of three associators is 0 then the other three are as well
and so the definition reduces to $/s',s,t/=0$.  \PMlinkescapetext{Similar reductions} occur of other nonassociative
algebras. 

%%%%%
%%%%%
\end{document}
