\documentclass[12pt]{article}
\usepackage{pmmeta}
\pmcanonicalname{NilpotentCone}
\pmcreated{2013-03-22 13:58:36}
\pmmodified{2013-03-22 13:58:36}
\pmowner{rmilson}{146}
\pmmodifier{rmilson}{146}
\pmtitle{nilpotent cone}
\pmrecord{9}{34748}
\pmprivacy{1}
\pmauthor{rmilson}{146}
\pmtype{Definition}
\pmcomment{trigger rebuild}
\pmclassification{msc}{17B20}
\pmsynonym{nilcone}{NilpotentCone}

\endmetadata

\usepackage{amsmath}
\usepackage{amsfonts}
\usepackage{amssymb}


\newcommand{\fg}{\mathfrak{g}}
\newcommand{\cN}{\mathcal{N}}
\begin{document}
Let $\fg$ be a finite dimensional semisimple Lie algebra. The
\emph{nilpotent cone} $\cN$ of $\fg$ is the set of elements that
act nilpotently in all representations of $\fg$. In other words,
$$\cN=\{ a\in \fg : \rho(a) \text{ is nilpotent for all representations } \rho:\fg\to \operatorname{End}(V)\}$$
The nilpotent cone is an \PMlinkname{irreducible}{IrreducibleClosedSet}
\PMlinkname{subvariety}{AffineVariety} of $\fg$ (considered as a
$k$-vector space),  and is invariant under the adjoint action of $\fg$
on itself.  

\textbf{Example:} if $\fg=\operatorname{sl}_2$, then the nilpotent cone
is the variety of all matrices in $\fg$ with rank $1$.
%%%%%
%%%%%
\end{document}
