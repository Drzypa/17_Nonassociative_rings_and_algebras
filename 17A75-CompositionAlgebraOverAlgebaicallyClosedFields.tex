\documentclass[12pt]{article}
\usepackage{pmmeta}
\pmcanonicalname{CompositionAlgebraOverAlgebaicallyClosedFields}
\pmcreated{2013-03-22 17:18:22}
\pmmodified{2013-03-22 17:18:22}
\pmowner{Algeboy}{12884}
\pmmodifier{Algeboy}{12884}
\pmtitle{composition algebra over algebaically closed fields}
\pmrecord{6}{39654}
\pmprivacy{1}
\pmauthor{Algeboy}{12884}
\pmtype{Theorem}
\pmcomment{trigger rebuild}
\pmclassification{msc}{17A75}
\pmrelated{HurwitzsTheorem}
\pmrelated{JacobsonsTheoremOnCompositionAlgebras}

\endmetadata

\usepackage{latexsym}
\usepackage{amssymb}
\usepackage{amsmath}
\usepackage{amsfonts}
\usepackage{amsthm}

%%\usepackage{xypic}

%-----------------------------------------------------

%       Standard theoremlike environments.

%       Stolen directly from AMSLaTeX sample

%-----------------------------------------------------

%% \theoremstyle{plain} %% This is the default

\newtheorem{thm}{Theorem}

\newtheorem{coro}[thm]{Corollary}

\newtheorem{lem}[thm]{Lemma}

\newtheorem{lemma}[thm]{Lemma}

\newtheorem{prop}[thm]{Proposition}

\newtheorem{conjecture}[thm]{Conjecture}

\newtheorem{conj}[thm]{Conjecture}

\newtheorem{defn}[thm]{Definition}

\newtheorem{remark}[thm]{Remark}

\newtheorem{ex}[thm]{Example}



%\countstyle[equation]{thm}



%--------------------------------------------------

%       Item references.

%--------------------------------------------------


\newcommand{\exref}[1]{Example-\ref{#1}}

\newcommand{\thmref}[1]{Theorem-\ref{#1}}

\newcommand{\defref}[1]{Definition-\ref{#1}}

\newcommand{\eqnref}[1]{(\ref{#1})}

\newcommand{\secref}[1]{Section-\ref{#1}}

\newcommand{\lemref}[1]{Lemma-\ref{#1}}

\newcommand{\propref}[1]{Prop\-o\-si\-tion-\ref{#1}}

\newcommand{\corref}[1]{Cor\-ol\-lary-\ref{#1}}

\newcommand{\figref}[1]{Fig\-ure-\ref{#1}}

\newcommand{\conjref}[1]{Conjecture-\ref{#1}}


% Normal subgroup or equal.

\providecommand{\normaleq}{\unlhd}

% Normal subgroup.

\providecommand{\normal}{\lhd}

\providecommand{\rnormal}{\rhd}
% Divides, does not divide.

\providecommand{\divides}{\mid}

\providecommand{\ndivides}{\nmid}


\providecommand{\union}{\cup}

\providecommand{\bigunion}{\bigcup}

\providecommand{\intersect}{\cap}

\providecommand{\bigintersect}{\bigcap}










\begin{document}
\begin{thm}
There are 4 non-isomorphic composition algebras over an algebraically closed field $k$:
one division algebra, the field itself, and the three split algebras.
\begin{enumerate}
\item $k$.
\item The \emph{exchange} algebra: $k\oplus k$.
\item $2\times 2$ matrices over $k$: $M_2(k)$.
\item The cross-product of $2\times 2$-matrices over $k$: $M_2(k)\circ M_2(k)$.
\end{enumerate}
\end{thm}
\begin{proof}
To see this recall that every composition algebra comes equipped with a quadratic form.
Any 2-dimensional anisotropic subspace arises from a quadratic field extension.  As
our field is algebraically closed the quadratic form has no anisotropic subspaces and
is therefore the unique quadratic form of maximal Witt index.  Following Hurwitz's theorem
we know the composition algebras come in dimensions 1,2,4, and 8 and arise by the 
Cayley-Dickson method.  Thus we have the field itself and the three split
composition algebras.
\end{proof}
%%%%%
%%%%%
\end{document}
