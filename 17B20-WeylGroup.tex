\documentclass[12pt]{article}
\usepackage{pmmeta}
\pmcanonicalname{WeylGroup}
\pmcreated{2013-03-22 13:11:52}
\pmmodified{2013-03-22 13:11:52}
\pmowner{mathcam}{2727}
\pmmodifier{mathcam}{2727}
\pmtitle{Weyl group}
\pmrecord{6}{33658}
\pmprivacy{1}
\pmauthor{mathcam}{2727}
\pmtype{Definition}
\pmcomment{trigger rebuild}
\pmclassification{msc}{17B20}

\endmetadata

% this is the default PlanetMath preamble.  as your knowledge
% of TeX increases, you will probably want to edit this, but
% it should be fine as is for beginners.

% almost certainly you want these
\usepackage{amssymb}
\usepackage{amsmath}
\usepackage{amsfonts}

% used for TeXing text within eps files
%\usepackage{psfrag}
% need this for including graphics (\includegraphics)
%\usepackage{graphicx}
% for neatly defining theorems and propositions
%\usepackage{amsthm}
% making logically defined graphics
%%%\usepackage{xypic}

% there are many more packages, add them here as you need them

% define commands here
\begin{document}
The Weyl group $W_R$ of a root system $R\subset E$, where $E$ is a Euclidean vector space,
is the subgroup of $\mathrm{GL}(E)$ generated by reflection in the hyperplanes perpendicular
to the roots.  The map of reflection in a root $\alpha$ is given by  
$$r_{\alpha}(v)=v-2\frac{(\alpha,v)}{(\alpha,\alpha)}\alpha.$$

The Weyl group is generated by reflections in the simple roots for any choice of a set of positive roots.
There is a well-defined length function $\ell:W_R\to\mathbb{Z}$, where $\ell(w)$ is the minimal 
number of reflections in simple roots that $w$ can be written as.  This is also the number of positive
roots that $w$ takes to negative roots.
%%%%%
%%%%%
\end{document}
