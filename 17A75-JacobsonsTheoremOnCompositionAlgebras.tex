\documentclass[12pt]{article}
\usepackage{pmmeta}
\pmcanonicalname{JacobsonsTheoremOnCompositionAlgebras}
\pmcreated{2013-03-22 17:18:14}
\pmmodified{2013-03-22 17:18:14}
\pmowner{Algeboy}{12884}
\pmmodifier{Algeboy}{12884}
\pmtitle{Jacobson's theorem on composition algebras}
\pmrecord{4}{39651}
\pmprivacy{1}
\pmauthor{Algeboy}{12884}
\pmtype{Theorem}
\pmcomment{trigger rebuild}
\pmclassification{msc}{17A75}
\pmrelated{CompositionAlgebrasOverMathbbR}
\pmrelated{HurwitzsTheoremOnCompositionAlgebras}
\pmrelated{CompositionAlgebraOverAlgebaicallyClosedFields}
\pmrelated{CompositionAlgebrasOverFiniteFields}
\pmrelated{CompositionAlgebrasOverMathbbQ}

\usepackage{latexsym}
\usepackage{amssymb}
\usepackage{amsmath}
\usepackage{amsfonts}
\usepackage{amsthm}

%%\usepackage{xypic}

%-----------------------------------------------------

%       Standard theoremlike environments.

%       Stolen directly from AMSLaTeX sample

%-----------------------------------------------------

%% \theoremstyle{plain} %% This is the default

\newtheorem{thm}{Theorem}

\newtheorem{coro}[thm]{Corollary}

\newtheorem{lem}[thm]{Lemma}

\newtheorem{lemma}[thm]{Lemma}

\newtheorem{prop}[thm]{Proposition}

\newtheorem{conjecture}[thm]{Conjecture}

\newtheorem{conj}[thm]{Conjecture}

\newtheorem{defn}[thm]{Definition}

\newtheorem{remark}[thm]{Remark}

\newtheorem{ex}[thm]{Example}



%\countstyle[equation]{thm}



%--------------------------------------------------

%       Item references.

%--------------------------------------------------


\newcommand{\exref}[1]{Example-\ref{#1}}

\newcommand{\thmref}[1]{Theorem-\ref{#1}}

\newcommand{\defref}[1]{Definition-\ref{#1}}

\newcommand{\eqnref}[1]{(\ref{#1})}

\newcommand{\secref}[1]{Section-\ref{#1}}

\newcommand{\lemref}[1]{Lemma-\ref{#1}}

\newcommand{\propref}[1]{Prop\-o\-si\-tion-\ref{#1}}

\newcommand{\corref}[1]{Cor\-ol\-lary-\ref{#1}}

\newcommand{\figref}[1]{Fig\-ure-\ref{#1}}

\newcommand{\conjref}[1]{Conjecture-\ref{#1}}


% Normal subgroup or equal.

\providecommand{\normaleq}{\unlhd}

% Normal subgroup.

\providecommand{\normal}{\lhd}

\providecommand{\rnormal}{\rhd}
% Divides, does not divide.

\providecommand{\divides}{\mid}

\providecommand{\ndivides}{\nmid}


\providecommand{\union}{\cup}

\providecommand{\bigunion}{\bigcup}

\providecommand{\intersect}{\cap}

\providecommand{\bigintersect}{\bigcap}










\begin{document}
Recall that composition algebra $C$ over a field $k$ is specified with a quadratic form $q:C\to k$.
Furthermore, two quadratic forms $q:C\to k$ and $r:D\to k$ are isometric if there exists an
invertible linear map $f:C\to D$ such that $r(f(x))=q(x)$ for all $x\in C$.

\begin{thm}[Jacobson]\cite[Theorem 3.23]{Schafer:nonass}
Two unital Cayley-Dickson algebras $C$ and $D$ over a field $k$ of characteristic not $2$
are isomorphic if, and only if, their quadratic forms are isometric.
\end{thm}

A Cayley-Dickson algebra is split if the algebra has non-trivial zero-divisors.

\begin{coro}\cite[Corollary 3.24]{Schafer:nonass}
Upto isomorphism there is only one split Cayley-Dickson algebra and the quadratic form
has Witt index 4.
\end{coro}

Over the real numbers instead of Witt index, we say the signature of the quadratic form is $(4,4)$.

This result is often used together with a theorem of Hurwitz which limits the dimensions
of composition algebras to dimensions 1,2, 4 or 8.  Thus to classify the composition algebras
over a given field $k$ of characteristic not 2, it suffices to classify the non-degenerate
quadratic forms $q:k^n\to k$ with $n=1,2,4$ or $8$.

\begin{thebibliography}{7}
\bibitem{Schafer:nonass}
Richard~D. Schafer, \emph{An introduction to nonassociative algebras}, Pure and
  Applied Mathematics, Vol. 22, Academic Press, New York, 1966. 
\end{thebibliography}

%%%%%
%%%%%
\end{document}
