\documentclass[12pt]{article}
\usepackage{pmmeta}
\pmcanonicalname{WeightLattice}
\pmcreated{2013-03-22 13:11:57}
\pmmodified{2013-03-22 13:11:57}
\pmowner{bwebste}{988}
\pmmodifier{bwebste}{988}
\pmtitle{weight lattice}
\pmrecord{9}{33660}
\pmprivacy{1}
\pmauthor{bwebste}{988}
\pmtype{Definition}
\pmcomment{trigger rebuild}
\pmclassification{msc}{17B20}
\pmdefines{integral weight}

\endmetadata

% this is the default PlanetMath preamble.  as your knowledge
% of TeX increases, you will probably want to edit this, but
% it should be fine as is for beginners.

% almost certainly you want these
\usepackage{amssymb}
\usepackage{amsmath}
\usepackage{amsfonts}

% used for TeXing text within eps files
%\usepackage{psfrag}
% need this for including graphics (\includegraphics)
%\usepackage{graphicx}
% for neatly defining theorems and propositions
%\usepackage{amsthm}
% making logically defined graphics
%%%\usepackage{xypic}

% there are many more packages, add them here as you need them

% define commands here
\begin{document}
The weight lattice $\Lambda_W$ of a root system $R\subset E$ is the lattice $$\Lambda_W=\left\{ e\in E \left| \frac{(e,\alpha)}{(\alpha,\alpha)}\in\mathbb{Z} \text{ for all } r\in R \right. \right\} .$$  Weights which lie in the weight lattice are called {\em \PMlinkescapetext{integral}}. If $R\subset\mathfrak{h}$ is the root system of a semi-simple Lie algebra $\mathfrak{g}$ with Cartan subalgebra $\mathfrak{h}$, then $\Lambda_W$ is exactly the set of weights appearing in finite dimensional representations of $\mathfrak{g}$.
%%%%%
%%%%%
\end{document}
