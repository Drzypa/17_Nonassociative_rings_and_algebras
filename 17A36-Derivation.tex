\documentclass[12pt]{article}
\usepackage{pmmeta}
\pmcanonicalname{Derivation}
\pmcreated{2013-03-22 12:02:41}
\pmmodified{2013-03-22 12:02:41}
\pmowner{djao}{24}
\pmmodifier{djao}{24}
\pmtitle{derivation}
\pmrecord{10}{31089}
\pmprivacy{1}
\pmauthor{djao}{24}
\pmtype{Definition}
\pmcomment{trigger rebuild}
\pmclassification{msc}{17A36}
\pmclassification{msc}{16W25}
\pmclassification{msc}{13N15}

\endmetadata

\usepackage{amssymb}
\usepackage{amsmath}
\usepackage{amsfonts}
\usepackage{graphicx}
%%%\usepackage{xypic}
\newcommand{\x}{\mathbf{x}}
\newcommand{\y}{\mathbf{y}}
\renewcommand{\d}{\mathrm{d}}
\begin{document}
Let $R$ be a commutative ring. A \emph{derivation} $d$ on an $R$-algebra $A$ into an $A$-module $M$ is an $R$-linear transformation $\d\colon A \to M$ satisfying the properties
\begin{itemize}
\item $\d(a\x+b\y) = a\,\d\x + b\,\d\y$
\item $\d(\x\cdot \y) = \x \cdot \d\y + \d\x \cdot \y$
\end{itemize}
for all $a,b \in R$ and $\x,\y \in A$.
%%%%%
%%%%%
%%%%%
\end{document}
