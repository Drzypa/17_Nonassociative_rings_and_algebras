\documentclass[12pt]{article}
\usepackage{pmmeta}
\pmcanonicalname{AdjointRepresentation}
\pmcreated{2015-10-05 17:38:19}
\pmmodified{2015-10-05 17:38:19}
\pmowner{rmilson}{146}
\pmmodifier{rmilson}{146}
\pmtitle{adjoint representation}
\pmrecord{9}{32965}
\pmprivacy{1}
\pmauthor{rmilson}{146}
\pmtype{Definition}
\pmcomment{trigger rebuild}
\pmclassification{msc}{17B10}
\pmrelated{IsotropyRepresentation}
\pmdefines{adjoint action}
\pmdefines{gl}
\pmdefines{general linear Lie algebra}

\endmetadata


\begin{document}
\newcommand{\lag}{\mathfrak{g}}
\DeclareMathOperator{\ad}{ad}
\DeclareMathOperator{\End}{End}


Let $\lag$ be a Lie algebra.  For every $a\in\lag$ we define the
\PMlinkescapetext{{\em  adjoint endomorphism}}, a.k.a. the {\em adjoint action},
$$\ad(a):\lag\rightarrow\lag$$
to be the linear transformation with
action
$$\ad(a): b\mapsto [a,b],\quad b\in\lag.$$

For any vector space $V$, we use $\mathfrak{gl}(V)$ to denote the Lie algebra 
of $\End V$ determined by the commutator bracket.  So 
$\mathfrak{gl}(V)=\End V$ as vector spaces, only the multiplications are different.  

In this notation, treating $\mathfrak{g}$ as a vector space, the linear mapping $\ad:\lag\rightarrow \mathfrak{gl}(\lag)$ with action $$a\mapsto \ad(a),\quad a\in\lag$$
is called the {\em adjoint representation} of $\lag$.  The fact that
$\ad$ defines a representation is a straight-forward consequence of
the Jacobi identity axiom.  Indeed, let $a,b\in \lag$ be given.  We
wish to show that
$$\ad([a,b]) = [\ad(a),\ad(b)],$$
where the bracket on the left is the
$\lag$ multiplication structure, and the bracket on the right is the
commutator bracket.  For all $c\in\lag$ the left hand side maps $c$ to
$$[[a,b],c],$$
while the right hand side maps $c$ to
$$[a,[b,c]]+[b,[a,c]].$$
Taking skew-symmetry of the bracket as a
given, the equality of these two expressions is logically equivalent
to the Jacobi identity:
$$[a,[b,c]] +[b,[c,a]] + [c,[a,b]] = 0.$$
%%%%%
%%%%%
\end{document}
