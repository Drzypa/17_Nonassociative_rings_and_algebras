\documentclass[12pt]{article}
\usepackage{pmmeta}
\pmcanonicalname{CompositionAlgebra}
\pmcreated{2013-03-22 15:11:33}
\pmmodified{2013-03-22 15:11:33}
\pmowner{Algeboy}{12884}
\pmmodifier{Algeboy}{12884}
\pmtitle{composition algebra}
\pmrecord{14}{36949}
\pmprivacy{1}
\pmauthor{Algeboy}{12884}
\pmtype{Definition}
\pmcomment{trigger rebuild}
\pmclassification{msc}{17A75}
%\pmkeywords{composition law}
\pmrelated{TheoremsOnSumsOfSquares}

\endmetadata

\usepackage{latexsym}
\usepackage{amssymb}
\usepackage{amsmath}
\usepackage{amsfonts}
\usepackage{amsthm}
\usepackage{enumerate}
%%\usepackage{xypic}

\DeclareMathOperator{\tr}{T}

%-----------------------------------------------------

%       Standard theoremlike environments.

%       Stolen directly from AMSLaTeX sample

%-----------------------------------------------------

%% \theoremstyle{plain} %% This is the default

\newtheorem{thm}{Theorem}

\newtheorem{coro}[thm]{Corollary}

\newtheorem{lem}[thm]{Lemma}

\newtheorem{lemma}[thm]{Lemma}

\newtheorem{prop}[thm]{Proposition}

\newtheorem{conjecture}[thm]{Conjecture}

\newtheorem{conj}[thm]{Conjecture}

\newtheorem{defn}[thm]{Definition}

\newtheorem{remark}[thm]{Remark}

\newtheorem{ex}[thm]{Example}



%\countstyle[equation]{thm}



%--------------------------------------------------

%       Item references.

%--------------------------------------------------


\newcommand{\exref}[1]{Example-\ref{#1}}

\newcommand{\thmref}[1]{Theorem-\ref{#1}}

\newcommand{\defref}[1]{Definition-\ref{#1}}

\newcommand{\eqnref}[1]{(\ref{#1})}

\newcommand{\secref}[1]{Section-\ref{#1}}

\newcommand{\lemref}[1]{Lemma-\ref{#1}}

\newcommand{\propref}[1]{Prop\-o\-si\-tion-\ref{#1}}

\newcommand{\corref}[1]{Cor\-ol\-lary-\ref{#1}}

\newcommand{\figref}[1]{Fig\-ure-\ref{#1}}

\newcommand{\conjref}[1]{Conjecture-\ref{#1}}


% Normal subgroup or equal.

\providecommand{\normaleq}{\unlhd}

% Normal subgroup.

\providecommand{\normal}{\lhd}

\providecommand{\rnormal}{\rhd}
% Divides, does not divide.

\providecommand{\divides}{\mid}

\providecommand{\ndivides}{\nmid}


\providecommand{\union}{\cup}

\providecommand{\bigunion}{\bigcup}

\providecommand{\intersect}{\cap}

\providecommand{\bigintersect}{\bigcap}






\newcommand{\CayDick}[2]{\left(\frac{#1}{#2}\right)}



\begin{document}
\section{Definition}

The classical definition of a composition algebra is a
non-associative algebra $C$ over a field $k$ where
\begin{enumerate}
\item $C$ admits a non-degenerate quadratic form $q:C\to k$, such
that
\item $q$ is multiplicative: $q(xy)=q(x)q(y)$.
\end{enumerate}
We also say $q$ permits composition or that it obeys the composition law.

This definition is geometric in that quadratic forms give rise to geometric atributes
for a vector space such as length, distance and orthogonality.  Indeed, originally created
over the real numbers such properties seem appropriate for an algebra; however, concepts
of length and distance are less appropriate over arbitrary fields and encourage a second
equivalent definition based solely on the algebraic aspect of such algebras.

Alternatively, a composition algebra can be defined as a unital alternative algebra $C$ 
over a field $k$ with an involution $x\mapsto \bar{x}$, that is an anti-isomorphism
of order at most 2, such that:
\begin{enumerate}
\item $C$ has no non-zero absolute zero divisors (that is, $(xa)x=0$ for all $a\in C$ implies $x=0$);
\item the norm $N(x):=x\bar{x}$ is a scalar multiple of $1$, that is, $N:C\to k1$.
\end{enumerate}

The first definition makes the composition property part of the definition but obscures 
the alternative multiplication as well as the existence of an involutary anti-isomorphism
for the algebra.  The second definition makes both of these properties evident but obscures 
the composition property of the norm, and also hides the property that $N$ is a quadratic form.
However both definitions have merit, the first captures the classical view of an algebra 
respecting a certain geometric condition while the second, introduced by Jacobson, promotes
a purely algebraic treatment.  In our examples and constructions to follows we attempt to exhibit 
both aspects by supplying the norm, the involution, and the product.

Both definitions can be generalized to algebras over commutative unital rings $k$.

Recall that a quadratic form gives rise to a symmetric bilinear from $b:C\times C\to k$ by
$b(x,y)=q(x+y)-q(x)-q(y)$, for all $x,y\in C$.  Some of the immediate properties include:
\begin{enumerate}
\item $b(x,x)=2q(x)$, 
\item $b(xz,yz)=b(x,y)q(z)$,
\item $b(xy,zw)+b(xw,zy)=b(x,y)b(z,w)$.
\end{enumerate}
These strongly limit the structure of composition algebras
and leads to the celebrated theorem of Hurwitz (see Theorem \ref{thm:Hurwitz}) 
which suitably classifies the composition algebras over $\mathbb{R}$.  The work of
many others including Albert, Dickson, Jacobson, and Kaplansky extended the essential
conclussion of Hurwitz to all fields and the resulting generalization is still
refered to as Hurwitz's theorem.

There are other algebras $A$ with norms $q:A\to k$ which permit composition
in the sense that $q(xy)=q(x)q(y)$.  For example, alternative algebras with involutions.
However, the distinguishing property of composition algebras is that $q$ is a quadratic form.
Classifications for such norms have been caried out by Schafer and McCrimmon.

\section{Norms}

Originally, composition algebras were created over the real numbers $k=\mathbb{R}$. 
Here the usual positive definite norm on the real vector space was used instead of 
the quadratic form (the square of the norm is the quadratic form).

The first non-trivial example is the set of complex numbers $\mathbb{C}$ with where
$z=a+bi\in\mathbb{C}$ is assigned:
\begin{eqnarray*} 
\bar{z} & =& a-bi;\\
|z| & = & |a+bi|=\sqrt{a^2+b^2}=\sqrt{z\bar{z}}.
\end{eqnarray*}

More interesting is the non-commutative algebra of Hamiltonians 
$\mathbb{H}$, created by Hamilton, where each $x\in\mathcal{H}$ has the form
$x=a+bi+cj+dk$ and
\begin{eqnarray*}
\bar{x} & = & a-bi-cj-dk;\\
|x| & = & |a+bi+cj+dk|=\sqrt{a^2+b^2+c^2+d^2}=\sqrt{x\bar{x}}.
\end{eqnarray*} 

The last addition to the list was the non-associative algebra of \emph{octonions}
initially created by Cayley and the norm is simply 
\begin{eqnarray*}
  \bar{x} & = & a-bi-cj-dk-fil-gjl-hkl;\\
  |x|&=&|a+bi+cj+dk+el+fil+gjl+hkl| \\ 
     &=& \sqrt{a^2+b^2+c^2+d^2+e^2+f^2+g^2+h^2}=\sqrt{x\bar{x}}.
\end{eqnarray*}

Because general fields do not sufficient squareroots, the use of norms
in the classical Euclidean sense is replaced by the use of quadratic forms.
Furthermore, the lack of ordering a field, such as a finite field, introduces the
need to use non-degenerate rather than positive definite conditions.  Under these 
generalizations composition algebras can be redefined form the classical context
of composition algebras over $\mathbb{R}$ to general composition algebras over
arbitrary fields, as done by our original definitions above.  In this context, 
there are three further composition algebras over $\mathbb{R}$.

\begin{ex}
Let $C=\mathbb{R}\oplus\mathbb{R}$ with $q(x,y)=xy$ for all
$(x,y)\in C$.  Then $C$ is a composition algebra.
\end{ex}
\begin{proof}
Evidently $q(ax,ay)=a^2 q(x,y)$ and the polarization of $q$ is the symmetric bilinear
form $b((x,y),(z,w))=xz-yw$ for all $(x,y),(z,w)\in C$ (so the signature is $(1,-1)$).
Thus $q$ is a quadratic form.

To check that $q$ has the compositional property let $(x,y),(z,w)\in C$.  Then
\begin{equation*}
q((x,y)(z,w))
    =q(xz,yw)=(xz)(yw)
    =(xy)(zw)
    =q(x,y)q(z,w).
\end{equation*}

Note also that by defining $\overline{(x,y)}=(y,x)$ then $(x,y)\overline{(x,y)}
=(xy,yx)=q(x,y)(1,1)$ and $b((x,y),(z,w))(1,1)=(x,y)(z,w)+\overline{(x,y)(z,w)}$.
\end{proof}

\begin{ex}
Let $C=M_2(\mathbb{R})$ with $q(X)=\det X$ for all $X\in C$.
Then $C$ is a composition algebra.
\end{ex}
\begin{proof}
Let $X\in C$ and $a\in \mathbb{R}$.  Then $q(aX)=\det (aX)=\det (aI_2)\det X=a^2\det X=a^2 q(X)$.
It is also evident that if $X=\begin{bmatrix} a & b\\ c & d\end{bmatrix}$ then
setting $\bar{X}=\begin{bmatrix} d & -b\\ -c & a\end{bmatrix}$ makes $(\det X)I_2=X\bar{X}$
and also $\tr(X)I_2=X+\bar{X}$, where $\tr(X)$ is the trace of $X$.  Hence
\begin{multline*}
b(X,Y)=(q(X+Y)-q(X)-q(Y))I_2=(\det (X+Y))I_2-(\det X) I_2-(\det Y) I_2\\
  =(X+Y)\overline{(X+Y)} - X \bar{X} - Y\bar{Y}\\
  =Y\bar{X}+X\bar{Y}
  =\tr (X\bar{Y})I_2.
\end{multline*}
Therefore, $b(X,Y)=\tr (X\bar{Y})$.  Since $\tr (X\bar{Y})=\tr (\bar{Y}X)=\tr(X\bar{Y})$, it follows
that $b$ is a symmetric bilinear form and so $q$ is quadratic form.

Finally, for composition note
\begin{equation*}
q(XY)
    = \det(XY)
    = det X\det Y
    = q(X)q(Y).
\end{equation*}
Therefore $C$ is a composition algebra.
\end{proof}

This gives two new composition algebras over $\mathbb{R}$ and indeed there is a third, constructed
below as the algebra $\CayDick{1,1,1}{\mathbb{R}}$, which is 8-dimensional and non-associative but
unlike the octonions, it has non-trivial zero-divisors.

\begin{defn}
A composition algebra is \emph{split} if the quadratic form is isotropic.
\end{defn}

The example of $\mathbb{R}\oplus\mathbb{R}$ and $M_2(\mathbb{R})$ just given are both examples of
split composition algebras.

\section{Involution}

Define
	\[\bar{x} := b(x,1)1 - x,\qquad \forall x\in C.\]
Immediately it follows that: for all $x,y\in C$,
\begin{enumerate}
\item $\overline{\bar{x}}=x$,
\item $\overline{x+y}=\bar{x}+\bar{y}$,
\item $\bar{1}=1$.
\end{enumerate}

Define the \emph{trace} of $x$ as $T(x)=x+\bar{x}$ and
the \emph{norm} of $x$ as $N(x)=\bar{x}x$.  Then it follows that:
	\[x^2-T(x)x+N(x)=0,\qquad \forall x\in C.\]
So $C$ is a \emph{quadratic} algebra since every element in $x$ has at most
a quadratic minimal polynomial.  In fact $N(x)$ is a quadratic form 
allowing composition.

\section{Constructing composition algebras}


All of the following are composition algebras.
\cite[III.4]{Schafer:nonass}
\begin{description}
\item[$\dim 1$:] $k$, with trivial involution $x=\bar{x}$ for all $x$ in $k$.
\item[$\dim 2$:] For any $\alpha\in k$, a \emph{quadratic} extension of $k$, that is
\[\CayDick{\alpha}{k}=\langle 1,i | i^2=\alpha\rangle.\]
Here $\{1,i\}$ is a basis and has an involution defined by $\bar{1}=1$ and $\bar{i}=-i$.
\item[$\dim 4$:] For any $\alpha,\beta\in k$, a \emph{quaternion} algebra over $k$ 
defined as
\[\CayDick{\alpha,\beta}{k}=\langle 1,i,j | i^2=\alpha, j^2=\beta, ij=-ji\rangle\]
Then $\{1,i,j,ij\}$ forms a basis.\footnote{It is common to use $k$ for $ij$, but $k$ here is used
exclusively for the underlying field.}  An involution is defined by $\bar{1}=1$, 
$\bar{i}=-i$, $\bar{j}=-j$ and extended linearly.
\item[$\dim 8$:] For any $\alpha,\beta,\gamma\in k$, an \emph{octonion} algebra over $k$:
\begin{multline*}
\CayDick{\alpha,\beta,\gamma}{k}=
\langle 1,i,j,l | i^2=\alpha, j^2=\beta, l^2=\gamma,\quad
	ij=-ji, il=-li, jl=-lj, \\
	i(lj)=-l(ij), (li)j=l(ji), (li)(lj)=-\gamma ji
	\rangle.
\end{multline*}
The set $\{1,i,j,ij,l,il,jl,ijl\}$ is a basis.
An involution is defined by $\bar{1}=1$, $\bar{i}=-i$, $\bar{j}=-j$, $\bar{l}=-l$
and extended linearly.
\end{description}

Each of these algebras can be realized by the Cayley-Dickson method which takes $C$ 
an associative $k$-algebra with involution and produces for each $\alpha\in C-\{0\}$ a new 
algebra $\CayDick{\alpha}{C}$ on the vector space $C\oplus C$ with product
	\[(a,b)(c,d)=(ac+\alpha d \bar{b},\bar{a}d+cb).\]
Set the involution on $\CayDick{\alpha}{C}$ to be $\overline{(a,b)}=(\bar{a},-b)$.  

The algebras are equipped with a \emph{trace} $Tr(x)=x+\bar{x}$,
and \emph{norm} $N(x)=x\bar{x}$.  This norm serves as the quadratic map to establish
these algebras a composition algebras.  The images of the trace and norm lie in $k$.

The new algebra is associative only if $C$ is commutative, otherwise it is alternative.  
This means that $\displaystyle k, \CayDick{\alpha}{k},\CayDick{\alpha,\beta}{k}$ are 
the associative composition algebras.

An algebra is a \emph{division} algebra if the only zero-divisor is $0$ 
\cite[II.2]{Schafer:nonass}.  A central simple composition algebra with a non-trivial 
zero-divisor is called a \emph{split} composition algebra.  Finite dimensional split 
central simple composition algebras are unique up to isomorphism to one of
\[ k,\quad \CayDick{1}{k}\cong k\oplus k,\quad \CayDick{1,1}{k}\cong M_2(k),
\quad \CayDick{1,1,1}{k}.\]

\section{Classification theorem}

\begin{thm}\label{thm:Hurwitz}\cite[Theorem 6.2.3]{jac}
A composition algebra $C$ over a field $k$ with quadratic form $q(x)=x\bar{x}$ is isomorphic
to one of the following:
\begin{enumerate}[(i)]
\item A purely inseparable extension field $E/F$ of characteristic $2$ and exponent $1$ (trivial
involution) so $q(x)=x^2$.
\item $k$ with trivial involution, so $q(x)=x^2$,
\item Quadratic composition algebra: $\CayDick{\alpha}{k}$ for $\alpha\in k$,
\item Quaternion algebra: $\CayDick{\alpha,\beta}{k}$ for $\alpha,\beta\in k$,
\item Octonion algebra: $\CayDick{\alpha,\beta,\gamma}{k}$ for $\alpha,\beta,\gamma\in k$.
\end{enumerate}
In particular, all composition algebras over $k$, save perhaps those of type $(i)$, are finite 
dimensional and of dimension $1$, $2$, $4$ or $8$.
\end{thm}

\begin{thebibliography}{7}
\bibitem{tyl} T.Y. Lam: \emph{ Introduction to Quadratic Forms over Fields}, AMS, Providence (2004).
\bibitem{jac} N. Jacobson \emph{Structure theory of Jordan algebras}, The University of Arkansas lecture
notes in mathematics, vol. 5, Fayetteville, 1981.
\bibitem{km} K. McCrimmon: \emph{ A Taste of Jordan Algebras}, Springer, New York (2004).
\bibitem{jcds} J.H. Conway, D.A. Smith: \emph{ On Quaternions and Octonions, Their Geometry, Arithmetic, and Symmetry}, AK Peters, Natick, Mass (2003)
\bibitem{Schafer:nonass}
Richard~D. Schafer, \emph{An introduction to nonassociative algebras}, Pure and
  Applied Mathematics, Vol. 22, Academic Press, New York, 1966. 
\end{thebibliography}

%%%%%
%%%%%
\end{document}
