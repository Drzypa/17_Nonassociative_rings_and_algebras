\documentclass[12pt]{article}
\usepackage{pmmeta}
\pmcanonicalname{LieAlgebra}
\pmcreated{2013-03-22 12:03:36}
\pmmodified{2013-03-22 12:03:36}
\pmowner{djao}{24}
\pmmodifier{djao}{24}
\pmtitle{Lie algebra}
\pmrecord{18}{31113}
\pmprivacy{1}
\pmauthor{djao}{24}
\pmtype{Definition}
\pmcomment{trigger rebuild}
\pmclassification{msc}{17B99}
\pmrelated{CommutatorBracket}
\pmrelated{LieGroup}
\pmrelated{UniversalEnvelopingAlgebra}
\pmrelated{RootSystem}
\pmrelated{SimpleAndSemiSimpleLieAlgebras2}
\pmdefines{Jacobi identity}
\pmdefines{subalgebra}
\pmdefines{ideal}
\pmdefines{normalizer}
\pmdefines{centralizer}
\pmdefines{kernel}
\pmdefines{homomorphism}
\pmdefines{center}
\pmdefines{centre}
\pmdefines{abelian Lie algebra}
\pmdefines{abelian}

\usepackage{amssymb}
\usepackage{amsmath}
\usepackage{amsfonts}
\usepackage{graphicx}
%%%\usepackage{xypic}
\newcommand{\X}{\mathcal{X}}
\newcommand{\g}{\mathfrak{g}}
\newcommand{\h}{\mathfrak{h}}
\begin{document}
A {\em Lie algebra} over a field $k$ is a vector space $\mathfrak{g}$ with a bilinear map $[\ ,\ ] : \mathfrak{g}\times\mathfrak{g}\to\mathfrak{g}$, called the {\em Lie bracket} and denoted $(x,y)\mapsto [x,y]$.  It is required to satisfy:
\begin{enumerate}
\item $[x,x] = 0$ for all $x\in\mathfrak{g}$.
\item The {\em Jacobi identity}: $[x,[y,z]] + [y,[z,x]] + [z,[x,y]] = 0$ for all $x,y,z\in\mathfrak{g}$.
\end{enumerate}

\section{Subalgebras \& Ideals}

A vector subspace $\h$ of the Lie algebra $\g$ is a {\em subalgebra} if $\h$ is closed under the Lie bracket operation, or, equivalently, if $\h$ itself is a Lie algebra under the same bracket operation as $\g$. An {\em ideal} of $\g$ is a subspace $\h$ for which $[x,y] \in \h$ whenever either $x \in \h$ or $y \in \h$. Note that every ideal is also a subalgebra.

Some general examples of subalgebras:
\begin{itemize}
\item The {\em center} of $\g$, defined by $Z(\g) := \{x \in \g \mid [x,y] = 0 \text{for all } y \in \g\}$. It is an ideal of $\g$.
\item The {\em normalizer} of a subalgebra $\h$ is the set $N(\h) := \{x \in \g \mid [x,\h] \subset \h\}$. The Jacobi identity guarantees that $N(\h)$ is always a subalgebra of $\g$.
\item The {\em centralizer} of a subset $X \subset \g$ is the set $C(X) := \{x \in \g \mid [x,X] = 0\}$. Again, the Jacobi identity implies that $C(X)$ is a subalgebra of $\g$.
\end{itemize}

\section{Homomorphisms}
Given two Lie algebras $\g$ and $\g'$ over the field $k$, a {\em homomorphism} from $\g$ to $\g'$ is a linear transformation $\phi: \g \to \g'$ such that $\phi([x,y]) = [\phi(x),\phi(y)]$ for all $x,y \in \g$. An injective homomorphism is called a {\em monomorphism}, and a surjective homomorphism is called an {\em epimorphism}.

The {\em kernel} of a homomorphism $\phi: \g \to \g'$ (considered as a linear transformation) is denoted $\ker(\phi)$. It is always an ideal in $\g$.

\section{Examples}

\begin{itemize}
\item Any vector space can be made into a Lie algebra simply by setting $[x,y] = 0$ for all vectors $x,y$. The resulting Lie algebra is called an {\em abelian} Lie algebra.
\item If $G$ is a Lie group, then the tangent space at the identity forms a Lie algebra over the real numbers.
\item $\mathbb{R}^3$ with the cross product operation is a nonabelian three dimensional Lie algebra over $\mathbb{R}$.
\end{itemize}

\section{Historical Note}
Lie algebras are so-named in honour of Sophus Lie, a Norwegian
mathematician who pioneered the study of these mathematical objects.
Lie's discovery was tied to his investigation of continuous
transformation groups and symmetries.  One joint project with Felix
Klein called for the classification of all finite-dimensional groups
acting on the plane.  The task seemed hopeless owing to the generally
non-linear nature of such group actions.  However, Lie was able to
solve the problem by remarking that a transformation group can be
locally reconstructed from its corresponding ``infinitesimal
generators'', that is to say vector fields corresponding to various
1-parameter subgroups.  In terms of this geometric correspondence, the
group composition operation manifests itself as the bracket of vector
fields, and this is very much a linear operation.  Thus the task of
classifying group actions in the plane became the task of classifying
all finite-dimensional Lie algebras of planar vector field; a project
that Lie brought to a successful conclusion.

This ``linearization trick'' proved to be incredibly fruitful and led
to great advances in geometry and differential equations.  Such
advances are based, however, on various results from the theory of  Lie
algebras.   Lie was the first to make significant contributions to this
purely algebraic theory, but he was surely not the last.
%%%%%
%%%%%
%%%%%
\end{document}
