\documentclass[12pt]{article}
\usepackage{pmmeta}
\pmcanonicalname{Nucleus}
\pmcreated{2013-03-22 14:52:19}
\pmmodified{2013-03-22 14:52:19}
\pmowner{CWoo}{3771}
\pmmodifier{CWoo}{3771}
\pmtitle{nucleus}
\pmrecord{10}{36548}
\pmprivacy{1}
\pmauthor{CWoo}{3771}
\pmtype{Definition}
\pmcomment{trigger rebuild}
\pmclassification{msc}{17A01}
\pmdefines{center of a nonassociative algebra}
\pmdefines{nuclear}

\endmetadata

% this is the default PlanetMath preamble.  as your knowledge
% of TeX increases, you will probably want to edit this, but
% it should be fine as is for beginners.

% almost certainly you want these
\usepackage{amssymb,amscd}
\usepackage{amsmath}
\usepackage{amsfonts}

% used for TeXing text within eps files
%\usepackage{psfrag}
% need this for including graphics (\includegraphics)
%\usepackage{graphicx}
% for neatly defining theorems and propositions
%\usepackage{amsthm}
% making logically defined graphics
%%%\usepackage{xypic}

% there are many more packages, add them here as you need them

% define commands here
\begin{document}
\PMlinkescapeword{associate}

Let $A$ be an algebra, not necessarily associative multiplicatively.  The \emph{nucleus} of $A$ is:
$$\mathcal{N}(A):=\lbrace a\in A\mid [a,A,A]=[A,a,A]=[A,A,a]=0 \rbrace,$$
where $[\ , , ]$ is the associator bracket.  In other words, the nucleus is the set of elements that multiplicatively associate with all elements of $A$.  An element $a\in A$ is \emph{nuclear} if $a\in\mathcal{N}(A)$.

$\mathcal{N}(A)$ is a Jordan subalgebra of $A$.  To see this, let $a,b\in \mathcal{N}(A)$.  Then for any $c,d\in A$,  
\begin{eqnarray}
[ab,c,d] &=& ((ab)c)d-(ab)(cd) = (a(bc))d-(ab)(cd) \\
&=& a((bc)d)-(ab)(cd) = a(b(cd))-(ab)(cd) \\
&=& a(b(cd))-a(b(cd)) = 0
\end{eqnarray}
Similarly, $[c,ab,d]=[c,d,ab]=0$ and so $ab\in\mathcal{N}(A)$.

Accompanying the concept of a nucleus is that of the \emph{center of a nonassociative algebra} $A$ (which is slightly different from the definition of the center of an associative algebra):
$$\mathcal{Z}(A):=\lbrace a\in \mathcal{N}(A)\mid [a,A]=0 \rbrace,$$
where $[\ , ]$ is the commutator bracket.

Hence elements in $\mathcal{Z}(A)$ commute \emph{as well as} associate with all elements of $A$.  Like the nucleus, the center of $A$ is also a Jordan subalgebra of $A$.
%%%%%
%%%%%
\end{document}
