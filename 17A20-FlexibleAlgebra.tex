\documentclass[12pt]{article}
\usepackage{pmmeta}
\pmcanonicalname{FlexibleAlgebra}
\pmcreated{2013-03-22 14:43:30}
\pmmodified{2013-03-22 14:43:30}
\pmowner{CWoo}{3771}
\pmmodifier{CWoo}{3771}
\pmtitle{flexible algebra}
\pmrecord{11}{36351}
\pmprivacy{1}
\pmauthor{CWoo}{3771}
\pmtype{Definition}
\pmcomment{trigger rebuild}
\pmclassification{msc}{17A20}
\pmrelated{Associator}
\pmrelated{AlternativeAlgebra}
\pmdefines{left power}
\pmdefines{right power}
\pmdefines{flexible}

% this is the default PlanetMath preamble.  as your knowledge
% of TeX increases, you will probably want to edit this, but
% it should be fine as is for beginners.

% almost certainly you want these
\usepackage{amssymb,amscd}
\usepackage{amsmath}
\usepackage{amsfonts}

% used for TeXing text within eps files
%\usepackage{psfrag}
% need this for including graphics (\includegraphics)
%\usepackage{graphicx}
% for neatly defining theorems and propositions
%\usepackage{amsthm}
% making logically defined graphics
%%%\usepackage{xypic}

% there are many more packages, add them here as you need them

% define commands here
\begin{document}
A non-associative algebra $A$ is \emph{flexible} if $[\ a,b,a\ ]=0$ for all $a,b\in A$, where $[\ , , ]$ is the associator on $A$.  In other words, we have $(ab)a=a(ba)$ for all $a,b\in A$.  Any associative algebra is clearly flexible.  Furthermore, any alternative algebra with characteristic $\neq 2$ is flexible.

Given an element $a$ in a flexible algebra $A$, define the \emph{left power} of $a$ iteratively as follows:
\begin{enumerate}
\item $L^1(a)=a$,
\item $L^n(a)=a\cdot L^{n-1}(a)$.
\end{enumerate}
Similarly, we can define the \emph{right power} of $a$ as:
\begin{enumerate}
\item $R^1(a)=a$,
\item $R^n(a)=R^{n-1}(a)\cdot a$.
\end{enumerate}
Then, we can show that $L^{n}(a)=R^{n}(a)$ for all positive integers $n$.  As a result, in a flexible algebra, one can define the (multiplicative) power of an element $a$ as $a^n$ unambiguously.
%%%%%
%%%%%
\end{document}
