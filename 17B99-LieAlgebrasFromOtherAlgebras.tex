\documentclass[12pt]{article}
\usepackage{pmmeta}
\pmcanonicalname{LieAlgebrasFromOtherAlgebras}
\pmcreated{2013-03-22 16:37:33}
\pmmodified{2013-03-22 16:37:33}
\pmowner{Algeboy}{12884}
\pmmodifier{Algeboy}{12884}
\pmtitle{Lie algebras from other algebras}
\pmrecord{8}{38825}
\pmprivacy{1}
\pmauthor{Algeboy}{12884}
\pmtype{Example}
\pmcomment{trigger rebuild}
\pmclassification{msc}{17B99}
%\pmkeywords{Lie algebra}
%\pmkeywords{associative envelope}
%\pmkeywords{universal enveloping algebra}
\pmrelated{Algebras2}
\pmdefines{associative envelope}

\usepackage{latexsym}
\usepackage{amssymb}
\usepackage{amsmath}
\usepackage{amsfonts}
\usepackage{amsthm}

\DeclareMathOperator{\Sym}{Sym}
\DeclareMathOperator{\Alt}{Alt}
\DeclareMathOperator{\sgn}{sign}

%%\usepackage{xypic}

%-----------------------------------------------------

%       Standard theoremlike environments.

%       Stolen directly from AMSLaTeX sample

%-----------------------------------------------------

%% \theoremstyle{plain} %% This is the default



%--------------------------------------------------

%       Item references.

%--------------------------------------------------

% Normal subgroup or equal.

\providecommand{\normaleq}{\unlhd}

% Normal subgroup.

\providecommand{\normal}{\lhd}

\providecommand{\rnormal}{\rhd}
% Divides, does not divide.

\providecommand{\divides}{\mid}

\providecommand{\ndivides}{\nmid}


\providecommand{\union}{\cup}

\providecommand{\bigunion}{\bigcup}

\providecommand{\intersect}{\cap}

\providecommand{\bigintersect}{\bigcap}











%%\usepackage{xypic}

%-----------------------------------------------------

%       Standard theoremlike environments.

%       Stolen directly from AMSLaTeX sample

%-----------------------------------------------------

%% \theoremstyle{plain} %% This is the default

\newtheorem{thm}{Theorem}

\newtheorem{coro}[thm]{Corollary}

\newtheorem{lem}[thm]{Lemma}

\newtheorem{lemma}[thm]{Lemma}

\newtheorem{prop}[thm]{Proposition}

\newtheorem{conjecture}[thm]{Conjecture}

\newtheorem{conj}[thm]{Conjecture}

\newtheorem{defn}[thm]{Definition}

\newtheorem{remark}[thm]{Remark}

\newtheorem{ex}[thm]{Example}



%\countstyle[equation]{thm}



%--------------------------------------------------

%       Item references.

%--------------------------------------------------


\newcommand{\exref}[1]{Example-\ref{#1}}

\newcommand{\thmref}[1]{Theorem-\ref{#1}}

\newcommand{\defref}[1]{Definition-\ref{#1}}

\newcommand{\eqnref}[1]{(\ref{#1})}

\newcommand{\secref}[1]{Section-\ref{#1}}

\newcommand{\lemref}[1]{Lemma-\ref{#1}}

\newcommand{\propref}[1]{Prop\-o\-si\-tion-\ref{#1}}

\newcommand{\corref}[1]{Cor\-ol\-lary-\ref{#1}}

\newcommand{\figref}[1]{Fig\-ure-\ref{#1}}

\newcommand{\conjref}[1]{Conjecture-\ref{#1}}


% Normal subgroup or equal.

\providecommand{\normaleq}{\unlhd}

% Normal subgroup.

\providecommand{\normal}{\lhd}

\providecommand{\rnormal}{\rhd}
% Divides, does not divide.

\providecommand{\divides}{\mid}

\providecommand{\ndivides}{\nmid}


\providecommand{\union}{\cup}

\providecommand{\bigunion}{\bigcup}

\providecommand{\intersect}{\cap}

\providecommand{\bigintersect}{\bigcap}










\begin{document}
\section{Lie algebras from associative algebras}

Given an associative (unital) algebra $A$ over a commutative ring $R$, we define $A^-$ as the $R$-module $A$ together with a new multiplication $[,]:A\times A\rightarrow A$ derived from the associative multiplication as follows:
    \[[a,b]=ab-ba.\]
This operation is commonly called the \emph{commutator bracket} on $A$.

\begin{prop}
$A^-$ is a Lie algebra.
\end{prop}
\begin{proof}
We already know $A^-$ is a module so we need simply to confirm that the commutator
bracket is a bilinear mapping and then demonstrate that it is alternating and satisfies the Jacobi identity.

Given $a,b,c\in A$, and $l\in R$ then 
\[[la+b,c]=(la+b)c-c(la+b)=l(ac-ca)+(bc-cb)=l[a,c]+[b,c].\]
The similar argument in the second variable shows that the operation is
bilinear.

Next, $[x,x]=xx-xx=0$ so $[,]$ is alternating.  Finally for the Jacobi 
identity we compute directly.
\begin{multline*}
[[a,b],c]+[[b,c],a]+[[c,a],b] =
  (ab-ba)c-c(ab-ba)+(bc-cb)a-a(bc-cb)+(ca-ac)b-b(ca-ac)\\
 = abc-bac-cab+cba+bca-cba-abc+acb+cab-acb-bca+bac=0.
\end{multline*}
\end{proof}

We notice this produces a functor from the category of associative algebras
to the category of Lie algebras.  However, to every commutative algebra
$A$, $A^-$ is a trivial Lie algebra, and so this functor is not faithful.
More generally, the center of an arbitrary associative algebra $A$ is lost
to the Lie algebra structure $A^-$.

We do observe some relationships between the algebraic structure of $A$ and
that of $A^-$.

\begin{thm}\label{thm:ideals}
If $I\normaleq A$ then $I\normal A^-$.
\end{thm}
\begin{proof}
We observe that a submodule of $A$ is a submodule of $A^-$ as the two are 
identitcal as modules.  It remains to show $[I,A^-]\leq I$.  So given
$a\in I$ and $b\in A$, then $[a,b]=ab-ba$ and as $ab,ba\in I$ we conclude
$[a,b]\in I$.
\end{proof}

\section{Associative envelopes}

Given a Lie algebra $\mathfrak{g}$ it is often desirable to reverse the process
described above, that is, to provide an associative algebra $A$ for which
$\mathfrak{g}=A^-$.  In general this is impossible as we will now explain.

Let $V$ be a vector space and $A$ the endomorphism algebra on $V$.  Then we
give the name $\mathfrak{gl}(V)$ to the Lie algebra $A^-$ (noting that $A$ is
associative under the composition of functions operation.)  Then we can also
define a subalgebra $\mathfrak{sl}(V)$ as the set of linear transformations
with trace 0.

Now we claim that $\mathfrak{sl}_2(\mathbb{C})$ is not equal to $B^-$ for any associative (unital) algebra $B$.  For it is easy to see $\mathfrak{sl}_2(\mathbb{C})$ has a basis of
three elements:
\[e=\begin{bmatrix} 0 & 1\\ 0 & 0\end{bmatrix},\qquad
h=\begin{bmatrix} 1 & 0 \\ 0 & -1\end{bmatrix},\qquad
f=\begin{bmatrix} 0 & 0 \\ 1 & 0\end{bmatrix}.\]
Therefore $B$ would also be 3-dimensional.  We also know that $\mathfrak{sl}_2(\mathbb{C})$ is a simple Lie algebra, that is, it has no 
proper ideals.  Therefore by Theorem \ref{thm:ideals}, $B$ can have no 
ideals either, so $B$ must be simple.  However the finite dimensional 
simple rings over $\mathbb{C}$ are isomorphic to matrix rings (by the Wedderburn-Artin theorem) 
$M_n(\mathbb{C})$ and thus cannot have dimension 3.

This forces the weaker question as to whether a Lie algebra can be embedded
in $A^-$ for some associative algebra $A$.  We call such embeddings \emph{
associative envelopes} of the Lie aglebra.  The existence of associative
envelopes of arbitrary Lie algebras is answered by a corollary
to the Poincare-Birkhoff-Witt theorem.

\begin{thm}
Every Lie algebra $\mathfrak{g}$ embeds in the universal enveloping algebra $\mathfrak{U(g)}^-$, where $\mathfrak{U(g)}$ is an associative algebra.
\end{thm}

Finite dimensional analogues also exist, some of which are simpler to 
observe.  For instance, a Lie aglebra $\mathfrak{g}$ can be represented in $\mathfrak{gl(g)}$ by the adjoint representation.  The representation is
not faithful unless the center of $\mathfrak{g}$ is trivial.  However, 
for semi-simple Lie algebras, the adjoint representation thus suffices
as an associative envelope.

\begin{remark}
This result is in contrast to Jordan algebras where there are isomorphism types (for example $3\times 3$ matrices over the octonions) which cannot be embedded
in $A^+$ for any associative algebra $A$.  [$A^+$ is the derived algebra 
of $A$ under the product $a.b=ab+ba$.]
\end{remark}

\subsection{Lie algebra from non-associative algebras}

If $A$ is not an associative algebra to begin with then we may still determine
the commutator bracket is bilinear and alternating.  However, the Jacobi
identity is in question.  If we define the \emph{associator bracket} as $[a,b,c]=(ab)c-a(bc)$ then we can write the computation for the Jacobi
identity as:
\begin{multline*}
[[a,b],c]+[[b,c],a]+[[c,a],b] =
  (ab-ba)c-c(ab-ba)+(bc-cb)a-a(bc-cb)+(ca-ac)b-b(ca-ac)\\
  = (ab)c-(ba)c-c(ab)+c(ba)+(bc)a-(cb)a-a(bc)+a(cb)+(ca)b-(ac)b-b(ca)+b(ac)\\
  = ((ab)c-a(bc)) + ((bc)a-b(ca)) + ((ca)b-c(ab)) - ((ba)c-b(ac)) - ((cb)a-c(ba))
     -((ac)b-a(cb))\\
  = [a,b,c]+[b,c,a]+[c,a,b]-[b,a,c]-[c,b,a]-[a,c,b].
\end{multline*}
We can write this right hand side using permutations on the set $\{a,b,c\}$
as:
\[
\sum_{\sigma\in\Alt(\{a,b,c\})} [[a\sigma,b\sigma],c\sigma]
=[a,b],c]+[[b,c],a]+[[c,a],b] =
\sum_{\sigma\in \Sym(\{a,b,c\})} \sgn(\sigma)[a\sigma, b\sigma, c\sigma].
\]
That is, in a non-associative algebra the corresponding Jacobi identity
is the possibly non-trivial sum over all permutations of associators.
We consider a few non-associative examples.

\begin{itemize}
\item If $A$ is a commutative non-associative algebra (perhaps a Jordan algebra) then 
   \[[a,b,c]=(ab)c-a(bc)=(ba)c-(bc)a=c(ba)-(cb)a=[c,b,a]\]
so the Jacobi identity holds.  However, if $A$ is commutative
then $[a,b]=0$ to begin with so the associated Lie algebra product is
trivial.

\item If $A$ is an alternative algebra, so $[a,b,c]=-[b,a,c]$, then again
the Jacobi identity holds.  So $A^-$ is a Lie algebra.  The typicall non-associative examples of an alternative algebra are the octonion algebras.
These produce a non-trivial Lie algebra.

\item We can also consider beginning with a Lie algebra $A$ and producing
$A^-$.  To avoid confusing the bracket of $A$ and that of $A^-$ we let the
multiplication of $A$ be denoted by juxtaposition, $ab$, $a,b\in A$.  Recall
that in a Lie algebra of characteristic 0 or odd then $ab=-ba$ so that
$[a,b]=ab-ba=2ab$ in $A^-$.  So we have simply scaled the original product
of $A$ by $2$.  To see the Jacobi identity still holds we note
   \[[a,b,c]=(ab)c-a(bc)=-(ba)c+(bc)a=c(ba)-(cb)a=[c,b,a].\]
So once again the associators cancel.
\end{itemize}

%%%%%
%%%%%
\end{document}
