\documentclass[12pt]{article}
\usepackage{pmmeta}
\pmcanonicalname{Anticommutative}
\pmcreated{2014-02-04 7:50:58}
\pmmodified{2014-02-04 7:50:58}
\pmowner{pahio}{2872}
\pmmodifier{pahio}{2872}
\pmtitle{anticommutative}
\pmrecord{8}{42241}
\pmprivacy{1}
\pmauthor{pahio}{2872}
\pmtype{Definition}
\pmcomment{trigger rebuild}
\pmclassification{msc}{17A01}
\pmsynonym{anticommutative operation}{Anticommutative}
\pmsynonym{anticommutativity}{Anticommutative}
\pmrelated{Supercommutative}
\pmrelated{AlternativeAlgebra}
\pmrelated{Subcommutative}

\endmetadata

% this is the default PlanetMath preamble.  as your knowledge
% of TeX increases, you will probably want to edit this, but
% it should be fine as is for beginners.

% almost certainly you want these
\usepackage{amssymb}
\usepackage{amsmath}
\usepackage{amsfonts}

% used for TeXing text within eps files
%\usepackage{psfrag}
% need this for including graphics (\includegraphics)
%\usepackage{graphicx}
% for neatly defining theorems and propositions
 \usepackage{amsthm}
% making logically defined graphics
%%%\usepackage{xypic}

% there are many more packages, add them here as you need them

% define commands here

\theoremstyle{definition}
\newtheorem*{thmplain}{Theorem}

\begin{document}
A binary operation ``$\star$'' is said to be \emph{anticommutative} if it satisfies the identity
\begin{align}
y\!\star\!x \;=\; -(x\!\star\!y),
\end{align}
where the minus denotes the \PMlinkescapetext{opposite} element in the algebra in question.\, This implies that\, 
$x\!\star\!x \;=\; -(x\!\star\!x)$,\, i.e. $x\!\star\!x$ must be the neutral element of the addition of the algebra:
\begin{align}
x\!\star\!x \;=\; \textbf{0}.
\end{align}
Using the distributivity of ``$\star$'' over ``$+$'' we see that the indentity (2) also implies (1):
$$\textbf{0} \;=\; (x\!+\!y)\!\star\!(x\!+\!y) \;=\; x\!\star\!x+x\!\star\!y+y\!\star\!x+y\!\star\!y
\;=\; x\!\star\!y+y\!\star\!x$$\\

A well known example of anticommutative operations is the vector product in the algebra 
\,$(\mathbb{R}^3,\,+,\,\times)$,\, satisfying
$$\vec{b}\!\times\!\vec{a} \;=\; -(\vec{a}\!\times\!\vec{b}),
\qquad \vec{a}\!\times\!\vec{a} \;=\; \vec{0}.$$
Also we know that the subtraction of numbers obeys \PMlinkescapetext{similar} identities
$$b\!-\!a \;=\; -(a\!-\!b), \qquad a\!-\!a \;=\; 0.$$\\
An important anticommutative operation is the Lie bracket.



%%%%%
%%%%%
\end{document}
