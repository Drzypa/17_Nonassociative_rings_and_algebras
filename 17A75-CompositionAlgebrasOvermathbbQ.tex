\documentclass[12pt]{article}
\usepackage{pmmeta}
\pmcanonicalname{CompositionAlgebrasOvermathbbQ}
\pmcreated{2013-03-22 17:18:29}
\pmmodified{2013-03-22 17:18:29}
\pmowner{Algeboy}{12884}
\pmmodifier{Algeboy}{12884}
\pmtitle{composition algebras over $\mathbb{Q}$}
\pmrecord{6}{39656}
\pmprivacy{1}
\pmauthor{Algeboy}{12884}
\pmtype{Example}
\pmcomment{trigger rebuild}
\pmclassification{msc}{17A75}
\pmrelated{HurwitzsTheorem}
\pmrelated{JacobsonsTheoremOnCompositionAlgebras}

\usepackage{latexsym}
\usepackage{amssymb}
\usepackage{amsmath}
\usepackage{amsfonts}
\usepackage{amsthm}

%%\usepackage{xypic}

\newcommand{\CayDick}[2]{\left(\frac{#1}{#2}\right)}

%-----------------------------------------------------

%       Standard theoremlike environments.

%       Stolen directly from AMSLaTeX sample

%-----------------------------------------------------

%% \theoremstyle{plain} %% This is the default

\newtheorem{thm}{Theorem}

\newtheorem{coro}[thm]{Corollary}

\newtheorem{lem}[thm]{Lemma}

\newtheorem{lemma}[thm]{Lemma}

\newtheorem{prop}[thm]{Proposition}

\newtheorem{conjecture}[thm]{Conjecture}

\newtheorem{conj}[thm]{Conjecture}

\newtheorem{defn}[thm]{Definition}

\newtheorem{remark}[thm]{Remark}

\newtheorem{ex}[thm]{Example}



%\countstyle[equation]{thm}



%--------------------------------------------------

%       Item references.

%--------------------------------------------------


\newcommand{\exref}[1]{Example-\ref{#1}}

\newcommand{\thmref}[1]{Theorem-\ref{#1}}

\newcommand{\defref}[1]{Definition-\ref{#1}}

\newcommand{\eqnref}[1]{(\ref{#1})}

\newcommand{\secref}[1]{Section-\ref{#1}}

\newcommand{\lemref}[1]{Lemma-\ref{#1}}

\newcommand{\propref}[1]{Prop\-o\-si\-tion-\ref{#1}}

\newcommand{\corref}[1]{Cor\-ol\-lary-\ref{#1}}

\newcommand{\figref}[1]{Fig\-ure-\ref{#1}}

\newcommand{\conjref}[1]{Conjecture-\ref{#1}}


% Normal subgroup or equal.

\providecommand{\normaleq}{\unlhd}

% Normal subgroup.

\providecommand{\normal}{\lhd}

\providecommand{\rnormal}{\rhd}
% Divides, does not divide.

\providecommand{\divides}{\mid}

\providecommand{\ndivides}{\nmid}


\providecommand{\union}{\cup}

\providecommand{\bigunion}{\bigcup}

\providecommand{\intersect}{\cap}

\providecommand{\bigintersect}{\bigcap}










\begin{document}
\begin{thm}
There are infinitely many composition algebras over $\mathbb{Q}$.
\end{thm}
\begin{proof}
Every quadratic extension of $\mathbb{Q}$ is a distinct composition algebra.  For example,
$\CayDick{p}{\mathbb{Q}}$ for $p$ a prime number.  This is sufficient to illustrate an infinite
number of quadratic composition algebras. 
\end{proof}
The other families of composition algebras also have an infinite number of non-isomorphic 
division algebras though the proofs are more involved.  It suffices to show provide
an infinite family of non-isometric quadratic forms of the form: 
       \[N_{p,q}(a,b,c,d)=a^2-b^2 p-c^2 q+d^2 pq\]
for rational numbers $p$ and $q$.  Such questions can involve complex number theory as
for instance, if $p$ is a prime congruent to $1$ modulo $4$ then
$N_{-1,-p}$ is isometric to $N_{-1,-1}$ and thus $N_{-1,-p}$ is isometric to $N_{-1,-q}$ for
any other prime $q\equiv 1\pmod{4}$.  But if $p\equiv 3\pmod{4}$ then this cannot be said.
%%%%%
%%%%%
\end{document}
