\documentclass[12pt]{article}
\usepackage{pmmeta}
\pmcanonicalname{BorelSubalgebra}
\pmcreated{2013-03-22 13:12:16}
\pmmodified{2013-03-22 13:12:16}
\pmowner{mathcam}{2727}
\pmmodifier{mathcam}{2727}
\pmtitle{Borel subalgebra}
\pmrecord{6}{33666}
\pmprivacy{1}
\pmauthor{mathcam}{2727}
\pmtype{Definition}
\pmcomment{trigger rebuild}
\pmclassification{msc}{17B20}

% this is the default PlanetMath preamble.  as your knowledge
% of TeX increases, you will probably want to edit this, but
% it should be fine as is for beginners.

% almost certainly you want these
\usepackage{amssymb}
\usepackage{amsmath}
\usepackage{amsfonts}

% used for TeXing text within eps files
%\usepackage{psfrag}
% need this for including graphics (\includegraphics)
%\usepackage{graphicx}
% for neatly defining theorems and propositions
%\usepackage{amsthm}
% making logically defined graphics
%%%\usepackage{xypic}

% there are many more packages, add them here as you need them

% define commands here
\newcommand{\mf}{\mathfrak}
\begin{document}
Let $\mf{g}$ be a semi-simple Lie group, $\mf{h}$ a Cartan subalgebra, $R$ the associated root system, and $R^+\subset R$ a set of positive roots.  We have a root decomposition into the Cartan subalgebra and the root spaces $\mf{g}_\alpha$
$$\mf{g}=\mf{h}\oplus\left(\bigoplus_{\alpha\in R}\mf{g}_\alpha\right).$$
Now let $\mf{b}$ be the direct sum of the Cartan subalgebra and the positive root spaces.
$$\mf{b}=\mf{h}\oplus\left(\bigoplus_{\beta\in R^+}\mf{g}_\beta\right).$$
This is called a {\em Borel subalgebra}.
%%%%%
%%%%%
\end{document}
