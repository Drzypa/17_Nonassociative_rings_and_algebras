\documentclass[12pt]{article}
\usepackage{pmmeta}
\pmcanonicalname{Associator}
\pmcreated{2013-03-22 14:43:21}
\pmmodified{2013-03-22 14:43:21}
\pmowner{CWoo}{3771}
\pmmodifier{CWoo}{3771}
\pmtitle{associator}
\pmrecord{10}{36348}
\pmprivacy{1}
\pmauthor{CWoo}{3771}
\pmtype{Definition}
\pmcomment{trigger rebuild}
\pmclassification{msc}{17A01}
\pmrelated{AlternativeAlgebra}
\pmrelated{PowerAssociativeAlgebra}
\pmrelated{FlexibleAlgebra}
\pmrelated{Commutator}
\pmdefines{anti-associative}

% this is the default PlanetMath preamble.  as your knowledge
% of TeX increases, you will probably want to edit this, but
% it should be fine as is for beginners.

% almost certainly you want these
\usepackage{amssymb,amscd}
\usepackage{amsmath}
\usepackage{amsfonts}

% used for TeXing text within eps files
%\usepackage{psfrag}
% need this for including graphics (\includegraphics)
%\usepackage{graphicx}
% for neatly defining theorems and propositions
%\usepackage{amsthm}
% making logically defined graphics
%%%\usepackage{xypic}

% there are many more packages, add them here as you need them

% define commands here
\begin{document}
\PMlinkescapeword{measures}

Let $A$ be a non-associative algebra over a field.  The \emph{associator} of $A$, denoted by $[\ , , ]$, is a \PMlinkname{trilinear}{multilinear} map from $A\times A\times A$ to $A$ given by:
$$[\ a,b,c\ ]=(ab)c-a(bc).$$
\par
Just as the commutator measures how close an algebra is to being commutative, the associator measures how close it is to being associative.  $[\ , , ]=0$ identically iff $A$ is associative.
\par
\begin{thebibliography}{8}
\bibitem{Shafer} R. D. Schafer, {\em An Introduction on Nonassociative Algebras}, Dover, New York (1995).
\end{thebibliography}
%%%%%
%%%%%
\end{document}
