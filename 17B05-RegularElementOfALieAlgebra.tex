\documentclass[12pt]{article}
\usepackage{pmmeta}
\pmcanonicalname{RegularElementOfALieAlgebra}
\pmcreated{2013-03-22 15:30:53}
\pmmodified{2013-03-22 15:30:53}
\pmowner{benjaminfjones}{879}
\pmmodifier{benjaminfjones}{879}
\pmtitle{regular element of a Lie algebra}
\pmrecord{6}{37382}
\pmprivacy{1}
\pmauthor{benjaminfjones}{879}
\pmtype{Definition}
\pmcomment{trigger rebuild}
\pmclassification{msc}{17B05}
%\pmkeywords{regular}
%\pmkeywords{Lie algebra}
\pmdefines{regular element}

% this is the default PlanetMath preamble.  as your knowledge
% of TeX increases, you will probably want to edit this, but
% it should be fine as is for beginners.

% almost certainly you want these
\usepackage{amssymb}
\usepackage{amsmath}
\usepackage{amsfonts}

% used for TeXing text within eps files
%\usepackage{psfrag}
% need this for including graphics (\includegraphics)
%\usepackage{graphicx}
% for neatly defining theorems and propositions
%\usepackage{amsthm}
% making logically defined graphics
%%%\usepackage{xypic}

% there are many more packages, add them here as you need them

% define commands here
\DeclareMathOperator{\rank}{rank}
\DeclareMathOperator{\diag}{diag}
\begin{document}
An element $X \in \mathfrak{g}$ of a Lie algebra is called \emph{regular} if the dimension of its centralizer $\zeta_{\mathfrak{g}}(X) = \{ Y \in \mathfrak{g} \mid 
[X, Y] = 0 \}$ is minimal among all centralizers of elements in $\mathfrak{g}$.

Regular elements clearly exist and moreover they are Zariski dense in $\mathfrak{g}$. The function $X \mapsto \dim \zeta_{\mathfrak{g}}(X)$ is an upper semi-continuous function $\mathfrak{g} \to \mathbb{Z}_{\ge 0}$. Indeed, it is a constant minus $\rank(ad_X)$ and $X \mapsto \rank(ad_X)$ is lower semi-continuous. Thus the set of elements whose centralizer dimension is (greater than or) equal to that of any given regular element is Zariski open and non-empty.

If $\mathfrak{g}$ is reductive then the minimal centralizer dimension is equal to the rank of $\mathfrak{g}$. 

More generally if $V$ is a representation for a Lie algebra $\mathfrak{g}$,
an element $v \in V$ is called \emph{regular} if the dimension of its stabilizer  is minimal among all stabilizers of elements in $V$.  


\section*{Examples}

\begin{enumerate}
\item In $\mathfrak{sl}_n \mathbb{C}$ a diagonal matrix $X = \diag(s_1, \ldots, s_n)$ is regular iff $(s_i - s_j) \ne 0$ for all pairs $1 \le i < j \le n$. Any conjugate of such a matrix is also obviously regular.

\item In $\mathfrak{sl}_n \mathbb{C}$ the nilpotent matrix
\[ \left( \begin{array}{ccccc} 
0 & 1 & \cdots &  & 0 \\
0 & 0 & 1 & &  \\
\vdots & & \ddots & \ddots & 1 \\
0 & & \cdots & & 0 
\end{array} \right) 
\]
is regular. Moreover, it's adjoint orbit contains the set of all regular nilpotent elements. The centralizer of this matrix is the full subalgebra of
trace zero, diagonal matricies. 
\end{enumerate}
%%%%%
%%%%%
\end{document}
