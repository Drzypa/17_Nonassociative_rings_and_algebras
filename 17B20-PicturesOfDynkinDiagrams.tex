\documentclass[12pt]{article}
\usepackage{pmmeta}
\pmcanonicalname{PicturesOfDynkinDiagrams}
\pmcreated{2013-03-22 13:29:15}
\pmmodified{2013-03-22 13:29:15}
\pmowner{Dr_Absentius}{537}
\pmmodifier{Dr_Absentius}{537}
\pmtitle{pictures of Dynkin diagrams}
\pmrecord{19}{34060}
\pmprivacy{1}
\pmauthor{Dr_Absentius}{537}
\pmtype{Example}
\pmcomment{trigger rebuild}
\pmclassification{msc}{17B20}

\endmetadata

%\documentclass{amsart}
\usepackage{amsmath}
%\usepackage[all,poly,knot,dvips]{xy}
\usepackage{pstricks,pst-poly,pst-node}
%\usepacage[dvips]{pstcol}


\usepackage{amssymb,latexsym}
\usepackage{color}

\usepackage{amsthm}
\usepackage{eucal}

% THEOREM Environments --------------------------------------------------

\newtheorem{thm}{Theorem}
 \newtheorem*{mainthm}{Main~Theorem}
 \newtheorem{cor}[thm]{Corollary}
 \newtheorem{lem}[thm]{Lemma}
 \newtheorem{prop}[thm]{Proposition}
 \newtheorem{claim}[thm]{Claim}
 \theoremstyle{definition}
 \newtheorem{defn}[thm]{Definition}
 \theoremstyle{remark}
 \newtheorem{rem}[thm]{Remark}
\newtheorem*{Rem}{Remark}
 \numberwithin{equation}{subsection}


%---------------------  Greek letters, etc ------------------------- 

\newcommand{\CA}{\mathcal{A}}
\newcommand{\CC}{\mathcal{C}}
\newcommand{\CM}{\mathcal{M}}
\newcommand{\CP}{\mathcal{P}}
\newcommand{\CS}{\mathcal{S}}
\newcommand{\BC}{\mathbb{C}}
\newcommand{\BN}{\mathbb{N}}
\newcommand{\BR}{\mathbb{R}}
\newcommand{\BZ}{\mathbb{Z}}
\newcommand{\FF}{\mathfrak{F}}
\newcommand{\FL}{\mathfrak{L}}
\newcommand{\FM}{\mathfrak{M}}
\newcommand{\Ga}{\alpha}
\newcommand{\Gb}{\beta}
\newcommand{\Gg}{\gamma}
\newcommand{\GG}{\Gamma}
\newcommand{\Gd}{\delta}
\newcommand{\GD}{\Delta}
\newcommand{\Ge}{\varepsilon}
\newcommand{\Gz}{\zeta}
\newcommand{\Gh}{\eta}
\newcommand{\Gq}{\theta}
\newcommand{\GQ}{\Theta}
\newcommand{\Gi}{\iota}
\newcommand{\Gk}{\kappa}
\newcommand{\Gl}{\lambda}
\newcommand{\GL}{\Lamda}
\newcommand{\Gm}{\mu}
\newcommand{\Gn}{\nu}
\newcommand{\Gx}{\xi}
\newcommand{\GX}{\Xi}
\newcommand{\Gp}{\pi}
\newcommand{\GP}{\Pi}
\newcommand{\Gr}{\rho}
\newcommand{\Gs}{\sigma}
\newcommand{\GS}{\Sigma}
\newcommand{\Gt}{\tau}
\newcommand{\Gu}{\upsilon}
\newcommand{\GU}{\Upsilon}
\newcommand{\Gf}{\varphi}
\newcommand{\GF}{\Phi}
\newcommand{\Gc}{\chi}
\newcommand{\Gy}{\psi}
\newcommand{\GY}{\Psi}
\newcommand{\Gw}{\omega}
\newcommand{\GW}{\Omega}
\newcommand{\Gee}{\epsilon}
\newcommand{\Gpp}{\varpi}
\newcommand{\Grr}{\varrho}
\newcommand{\Gff}{\phi}
\newcommand{\Gss}{\varsigma}

\def\co{\colon\thinspace}
\begin{document}
Here is a complete list of connected Dynkin diagrams. In general if the name of a
diagram has $n$ as a subscript then there are $n$ dots in the diagram.
There are four infinite series that correspond to \emph{classical} complex (that is over $\BC$) simple Lie algebras. No pun intended.
\begin{itemize}
\item $\mathrm{A}_n$, for $n\geq 1$ represents the simple complex
  Lie algebra $\mathfrak{sl}_{n+1}$:
  \begin{figure*}[h]
    \centering \colorbox{white}{%
\begin{pspicture}(-1,.5)(5.5,4.5)
        \psdots[dotsize=.2,dotstyle=o](0,4)(0,3)(1,3)(0,2)(1,2)(2,2)(0,.6)(1,.6)(2,.6)(3.8,.6)(4.8,.6)
        \psdots[dotsize=.1](2.7,.6)(2.9,.6)(3.1,.6)(.5,1.1)(.5,1.3)(.5,1.5)
        \psline(.09,3)(.91,3) \psline(.09,2)(.91,2)
        \psline(1.09,2)(1.91,2) \psline(.09,.6)(.91,.6)
        \psline(1.09,.6)(1.91,.6) \psline(2.09,.6)(2.5,.6)
        \psline(3.3,.6)(3.71,.6) \psline(3.89,.6)(4.71,.6)
        \rput(-.8,4){$\mathrm{A}_1$} \rput(-.8,3){$\mathrm{A}_2$}
        \rput(-.8,2){$\mathrm{A}_3$} \rput(-.8,.6){$\mathrm{A}_n$}
      \end{pspicture}}
  \end{figure*}
\item $\mathrm{B}_n$, for $n\geq 1$ represents the simple complex
  Lie algebra $\mathfrak{so}_{2n+1}$:
  \begin{figure*}[h]
    \centering \colorbox{white}{%
      \begin{pspicture}(-1,.5)(5.5,4.5)
        \psdots[dotsize=.2,dotstyle=o](0,4)(0,3)(1,3)(0,2)(1,2)(2,2)(0,.6)(1,.6)(2,.6)(3.8,.6)(4.8,.6)
        \psdots[dotsize=.1](2.7,.6)(2.9,.6)(3.1,.6)(.5,1.1)(.5,1.3)(.5,1.5)
        \psline(.08,3.06)(.92,3.06) \psline(.08,2.94)(.92,2.94)
        \psline(.5,3)(.3,3.2) \psline(.5,3)(.3,2.8)
        \psline(.09,2)(.91,2) \psline(1.08,2.06)(1.92,2.06)
        \psline(1.08,1.94)(1.92,1.94) \psline(1.5,2)(1.3,2.2)
        \psline(1.5,2)(1.3,1.8) \psline(.09,.6)(.91,.6)
        \psline(1.09,.6)(1.91,.6) \psline(2.09,.6)(2.5,.6)
        \psline(3.3,.6)(3.71,.6)
        % \psline(3.89,.6)(4.71,.6)
        \psline(3.89,.66)(4.72,.66) \psline(3.89,.54)(4.72,.54)
        \psline(4.3,.6)(4.1,.8) \psline(4.3,.6)(4.1,.4)
        \rput(-.8,4){$\mathrm{B}_1$} \rput(-.8,3){$\mathrm{B}_2$}
        \rput(-.8,2){$\mathrm{B}_3$} \rput(-.8,.6){$\mathrm{B}_n$}
      \end{pspicture}}
  \end{figure*}
\item $\mathrm{C}_n$, for $n\geq 1$ represents the simple complex
  Lie algebra $\mathfrak{sp}_{2n}$:
  \begin{figure*}[h]
    \centering \colorbox{white}{%
      \begin{pspicture}(-1,.5)(5.5,4.5)
        \psdots[dotsize=.2,dotstyle=o](0,4)(0,3)(1,3)(0,2)(1,2)(2,2)(0,.6)(1,.6)(2,.6)(3.8,.6)(4.8,.6)
        \psdots[dotsize=.1](2.7,.6)(2.9,.6)(3.1,.6)(.5,1.1)(.5,1.3)(.5,1.5)
        \psline(.08,3.06)(.92,3.06) \psline(.08,2.94)(.92,2.94)
        \psline(.5,3)(.7,3.2) \psline(.5,3)(.7,2.8)
        \psline(.09,2)(.91,2) \psline(1.08,2.06)(1.92,2.06)
        \psline(1.08,1.94)(1.92,1.94) \psline(1.5,2)(1.7,2.2)
        \psline(1.5,2)(1.7,1.8) \psline(.09,.6)(.91,.6)
        \psline(1.09,.6)(1.91,.6) \psline(2.09,.6)(2.5,.6)
        \psline(3.3,.6)(3.71,.6)
        % \psline(3.89,.6)(4.71,.6)
        \psline(3.89,.66)(4.72,.66) \psline(3.89,.54)(4.72,.54)
        \psline(4.3,.6)(4.5,.8) \psline(4.3,.6)(4.5,.4)
        \rput(-.8,4){$\mathrm{C}_1$} \rput(-.8,3){$\mathrm{C}_2$}
        \rput(-.8,2){$\mathrm{C}_3$} \rput(-.8,.6){$\mathrm{C}_n$}
      \end{pspicture}}
  \end{figure*}
  \eject \vfill
\item $\mathrm{D}_n$, for $n\geq 3$ represents the simple complex
  Lie algebra $\mathfrak{so}_{2n}$:
  \begin{figure*}[h] \centering \colorbox{white}{%
      \begin{pspicture}(-1,-3)(7.5,5)
        \psdots[dotsize=.2,dotstyle=o](0,4)(.7071068,4.7071068)(.7071068,3.292893)
        (0,2)(1,2)(1.7071068,2.7071068)(1.7071068,1.292893)
        (0,0)(1,0)(2,0)(2.7071068,.7071068)(2.7071068,-.7071068)
        (0,-2.2)(1,-2.2)(2,-2.2)(3,-2.2)(5,-2.2)(6,-2.2)(6.7071068,-1.49289)(6.7071068,-2.9071)
        \psline(.07,4.07)(.6471068,4.6471068)
        \psline(.07,3.93)(.6551068,3.358893) \psline(.09,2)(.91,2)
        \psline(1.07,2.07)(1.6471068,2.6471068)
        \psline(1.07,1.93)(1.6551068,1.358893) \psline(.09,0)(.91,0)
        \psline(1.09,0)(1.91,0)
        \psline(2.07,.07)(2.6471068,.6471068)
        \psline(2.07,-.03)(2.6551068,-.641)
        \psline(.09,-2.2)(.91,-2.2) \psline(1.09,-2.2)(1.91,-2.2)
        \psline(2.09,-2.2)(2.91,-2.2) \psline(3.09,-2.2)(3.5,-2.2)
        \psline(4.5,-2.2)(4.91,-2.2) \psline(5.09,-2.2)(5.91,-2.2)
        \psline(6.07,-2.13)(6.6471068,-1.552893)
        \psline(6.07,-2.27)(6.6551068,-2.841107)
        \psdots(.5,-1.2)(.5,-1.5)(.5,-1.8)
        (3.8,-2.2)(4,-2.2)(4.2,-2.2) \rput(-.8,4){$\mathrm{D}_3$}
        \rput(-.8,2){$\mathrm{D}_4$} \rput(-.8,0){$\mathrm{D}_5$}
        \rput(-.8,-2.2){$\mathrm{D}_n$}
      \end{pspicture}}
  \end{figure*}
\end{itemize}
And then there are the \emph{exceptional} cases that come in finite
families. The corresponding Lie algebras are usually called by the
name of the diagram.
\begin{itemize}
\item There is the $\mathrm{E}$ series that has three members:
  $\mathrm{E}_6$ which represents a $78$--dimensional Lie algebra,
  $\mathrm{E}_7$ which represents a $133$--dimensional Lie algebra,
  and $\mathrm{E}_8$ which represents a $248$--dimensional Lie
  algebra.
  \begin{figure*}[h]
    \centering \colorbox{white}{
      \begin{pspicture}(-1,-.2)(6.5,5.4)
        \psdots[dotsize=.2,dotstyle=o](0,4)(1,4)(2,4)(3,4)(4,4)(2,5)
        (0,2)(1,2)(2,2)(3,2)(4,2)(5,2)(2,3)
        (0,0)(1,0)(2,0)(3,0)(4,0)(5,0)(6,0)(2,1)
        \psline(.09,4)(.91,4) \psline(1.09,4)(1.91,4)
        \psline(2.09,4)(2.91,4) \psline(3.09,4)(3.91,4)
        \psline(2,4.09)(2,4.91) \psline(.09,2)(.91,2)
        \psline(1.09,2)(1.91,2) \psline(2.09,2)(2.91,2)
        \psline(3.09,2)(3.91,2) \psline(4.09,2)(4.91,2)
        \psline(2,2.09)(2,2.91) \psline(.09,0)(.91,0)
        \psline(1.09,0)(1.91,0) \psline(2.09,0)(2.91,0)
        \psline(3.09,0)(3.91,0) \psline(4.09,0)(4.91,0)
        \psline(5.09,0)(5.91,0) \psline(2,.09)(2,.91)
        \rput(-.8,4){$\mathrm{E}_6$} \rput(-.8,2){$\mathrm{E}_7$}
        \rput(-.8,0){$\mathrm{E}_8$}
      \end{pspicture}}
  \end{figure*}
  \eject \vfill
\item There is the $\mathrm{F}_4$ diagram which represents a
  $52$--dimensional complex simple Lie algebra:
  \begin{figure*}[h]
    \centering \colorbox{white}{
      \begin{pspicture}(-2,2)(4,4.5)
        \psdots[dotsize=.2,dotstyle=o](-1,3)(0,3)(1,3)(2,3)
        \psline(1.09,3)(1.91,3) \psline(-.09,3)(-.91,3)
        \psline(.08,3.06)(.92,3.06) \psline(.08,2.94)(.92,2.94)
        \psline(.5,3)(.3,3.2) \psline(.5,3)(.3,2.8)
        \rput(-1.8,3){$\mathrm{F}_4$}
      \end{pspicture}}
  \end{figure*}
\item And finally there is $\mathrm{G}_2$ that represents a
  $14$--dimensional Lie algebra.
  \begin{figure*}[h] \centering \colorbox{white}{
      \begin{pspicture}(-1,3.8)(4,1.5)
        \psdots[dotsize=.2,dotstyle=o](0,3)(1,3)
        \psline(.09,3)(.91,3) \psline(.08,3.06)(.92,3.06)
        \psline(.08,2.94)(.92,2.94) \psline(.5,3)(.3,3.2)
        \psline(.5,3)(.3,2.8) \rput(-.8,3){$\mathrm{G}_2$}
      \end{pspicture}}
  \end{figure*}
\end{itemize}

Notice the low dimensional coincidences:
$$\mathrm{A}_1=\mathrm{B}_1=\mathrm{C}_1$$
which reflects the exceptional isomorphisms
$$\mathfrak{sl}_2\cong\mathfrak{so}_3\cong\mathfrak{sp}_2\,.$$
Also
$$\mathrm{B}_2\cong\mathrm{C}_2$$
reflecting the isomorphism
$$\mathfrak{so}_5\cong\mathfrak{sp}_4\,.$$
And,
$$\mathrm{A}_3\cong\mathrm{D}_3$$
reflecting
$$\mathfrak{sl}_4\cong\mathfrak{so}_6\,.$$
\begin{Rem}
  Often in the literature the listing of Dynkin diagrams is arranged
  so that there are no ``intersections'' between different
  families. However by allowing intersections one gets a graphical
  representation of the low degree isomorphisms. In the same vein
  there is a graphical representation of the isomorphism
$$\mathfrak{so_4} \cong \mathfrak{sl_2}\times \mathfrak{sl_2}\,.$$
Namely, if not for the requirement that the families consist of
connected diagrams, one could start the $\mathrm{D}$ family with
\begin{center}
  \colorbox{white}{
    \begin{pspicture}(-1,3)(1,5)
      \psdots[dotsize=.2,dotstyle=o](.7071068,4.7071068)(.7071068,3.292893)
      \rput(-.8,4){$\mathrm{D}_2$}
    \end{pspicture}}
\end{center}
which consists of two disjoint copies of $\mathrm{A}_2$.
\end{Rem}

%%%%%
%%%%%
\end{document}
