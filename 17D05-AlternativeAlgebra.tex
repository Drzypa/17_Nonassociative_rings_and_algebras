\documentclass[12pt]{article}
\usepackage{pmmeta}
\pmcanonicalname{AlternativeAlgebra}
\pmcreated{2013-03-22 14:43:24}
\pmmodified{2013-03-22 14:43:24}
\pmowner{CWoo}{3771}
\pmmodifier{CWoo}{3771}
\pmtitle{alternative algebra}
\pmrecord{11}{36349}
\pmprivacy{1}
\pmauthor{CWoo}{3771}
\pmtype{Definition}
\pmcomment{trigger rebuild}
\pmclassification{msc}{17D05}
\pmrelated{Associator}
\pmrelated{FlexibleAlgebra}
\pmdefines{Artin's theorem on alternative algebras}
\pmdefines{alternative ring}
\pmdefines{left alternative law}
\pmdefines{right alternative law}

\endmetadata

% this is the default PlanetMath preamble.  as your knowledge
% of TeX increases, you will probably want to edit this, but
% it should be fine as is for beginners.

% almost certainly you want these
\usepackage{amssymb,amscd}
\usepackage{amsmath}
\usepackage{amsfonts}

% used for TeXing text within eps files
%\usepackage{psfrag}
% need this for including graphics (\includegraphics)
%\usepackage{graphicx}
% for neatly defining theorems and propositions
%\usepackage{amsthm}
% making logically defined graphics
%%%\usepackage{xypic}

% there are many more packages, add them here as you need them

% define commands here
\begin{document}
A non-associative algebra $A$ is \emph{alternative} if
\begin{enumerate}
\item (left alternative laws) $[\ a,a,b\ ]=0$, and
\item (right alternative laws) $[\ b,a,a\ ]=0$,
\end{enumerate}
for any $a,b\in A$, where $[\ , , ]$ is the associator on $A$.
\par
\textbf{Remarks}
\begin{itemize}
\item Let $A$ be alternative and suppose $\operatorname{char}(A)\neq2$.  From the fact that $[\ a+b,a+b,c\ ]=0$, we can deduce that the associator $[\ , , ]$ is \emph{anti-commutative}, when one of the three coordinates is held fixed.  That is, for any $a,b,c\in A$,
\begin{enumerate}
\item $[\ a,b,c\ ]=-[\ b,a,c\ ]$
\item $[\ a,b,c\ ]=-[\ a,c,b\ ]$
\item $[\ a,b,c\ ]=-[\ c,b,a\ ]$
\end{enumerate}
Put more succinctly, $$[\ a_1,a_2,a_3\ ]=\operatorname{sgn}(\pi)[\ a_{\pi(1)},a_{\pi(2)},a_{\pi(3)}\ ],$$ where $\pi\in S_3$, the symmetric group on three letters, and $\operatorname{sgn}(\pi)$ is the \PMlinkname{sign}{SignatureOfAPermutation} of $\pi$.
\item An alternative algebra is a flexible algebra, provided that the algebra is not \PMlinkname{Boolean}{BooleanLattice} (\PMlinkname{characteristic}{Characteristic} $\neq2$).  To see this, replace $c$ in the first anti-commutative identities above with $a$ and the result follows.
\item \textbf{Artin's Theorem}:  If a non-associative algebra $A$ is not Boolean, then $A$ is alternative iff every subalgebra of $A$ generated by two elements is associative.  The proof is clear from the above discussion.
\item A commutative alternative algebra $A$ is a Jordan algebra.  This is true since $a^2(ba)=a^2(ab)=(ab)a^2=((ab)a)a=(a(ab))a=(a^2b)a$ shows that the Jordan identity is satisfied.
\item Alternativity can be defined for a general ring $R$: it is a non-associative ring such that for any $a,b\in R$, $(aa)b=a(ab)$ and $(ab)b=a(bb)$.  Equivalently, an alternative ring is an alternative algebra over $\mathbb{Z}$.
\end{itemize}
%%%%%
%%%%%
\end{document}
