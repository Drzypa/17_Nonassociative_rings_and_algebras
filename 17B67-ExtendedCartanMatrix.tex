\documentclass[12pt]{article}
\usepackage{pmmeta}
\pmcanonicalname{ExtendedCartanMatrix}
\pmcreated{2013-03-22 15:30:14}
\pmmodified{2013-03-22 15:30:14}
\pmowner{benjaminfjones}{879}
\pmmodifier{benjaminfjones}{879}
\pmtitle{extended Cartan matrix}
\pmrecord{8}{37363}
\pmprivacy{1}
\pmauthor{benjaminfjones}{879}
\pmtype{Definition}
\pmcomment{trigger rebuild}
\pmclassification{msc}{17B67}
%\pmkeywords{affine Kac-Moody Lie algebra}
\pmrelated{GeneralizedCartanMatrix}
\pmdefines{extended Cartan matrix}

\endmetadata

% this is the default PlanetMath preamble.  as your knowledge
% of TeX increases, you will probably want to edit this, but
% it should be fine as is for beginners.

% almost certainly you want these
\usepackage{amssymb}
\usepackage{amsmath}
\usepackage{amsfonts}

% used for TeXing text within eps files
%\usepackage{psfrag}
% need this for including graphics (\includegraphics)
%\usepackage{graphicx}
% for neatly defining theorems and propositions
%\usepackage{amsthm}
% making logically defined graphics
%%%\usepackage{xypic}

% there are many more packages, add them here as you need them

% define commands here
\begin{document}
Let $A$ be the Cartan matrix of a complex, semi-simple, finite dimensional, Lie algebra $\mathfrak{g}$. Recall that $A = (a_{ij})$ where $a_{ij} = \langle \alpha_i, \alpha_j^\vee \rangle$ where the $\alpha_i$ are simple roots for $\mathfrak{g}$ and the $\alpha_j^\vee$ are simple coroots. 
The \emph{extended Cartan matrix} denoted $\hat A$ is obtained from $A$ by adding a zero-th row and column corresponding to adding a new simple root $\alpha_0 := -\theta$ where $\theta$ is the maximal (relative to $\left\{ \alpha_1,\ldots,\alpha_n \right\}$) root for $\mathfrak{g}$. $\theta$ can be defined as a root of $\mathfrak{g}$ such that when written in terms of simple roots $\theta = \sum_i m_i \alpha_i$ the coefficient sum $\sum_i m_i$ is maximal (i.e. it has maximal height). Such a root can be shown to be unique.

The matrix $\hat A$ is an example of a generalized Cartan matrix. The corresponding Kac-Moody Lie algerba is said to be of affine type.

For example if $\mathfrak{g} = \mathfrak{sl}_n \mathbb{C}$ then $\hat A$ is obtained from
$A$ by adding a zero-th row: $(2,-1,0,\ldots,0,-1)$ and zero-th column $(2,-1,0,\ldots,0,-1)$ simultaneously to the Cartan matrix for $\mathfrak{sl}_n \mathbb{C}$. 


\begin{thebibliography}{1}
\bibitem{Kac}
Victor Kac, \emph{Infinite Dimensional Lie Algebras}, Third edition. Cambridge University Press, Cambridge, 1990.
\end{thebibliography}
%%%%%
%%%%%
\end{document}
