\documentclass[12pt]{article}
\usepackage{pmmeta}
\pmcanonicalname{PoincareBirkhoffWittTheorem}
\pmcreated{2013-03-22 13:03:38}
\pmmodified{2013-03-22 13:03:38}
\pmowner{CWoo}{3771}
\pmmodifier{CWoo}{3771}
\pmtitle{Poincar\'e-Birkhoff-Witt theorem}
\pmrecord{7}{33467}
\pmprivacy{1}
\pmauthor{CWoo}{3771}
\pmtype{Theorem}
\pmcomment{trigger rebuild}
\pmclassification{msc}{17B35}
\pmsynonym{PBW-theorem}{PoincareBirkhoffWittTheorem}
%\pmkeywords{PBW}
%\pmkeywords{ordered monomials}
\pmrelated{LieAlgebra}
\pmrelated{UniversalEnvelopingAlgebra}
\pmrelated{FreeLieAlgebra}

\usepackage{amssymb}
\usepackage{amsmath}
\usepackage{amsfonts}
\newcommand{\NN}{\mathbb{N}}
\begin{document}
Let $\mathfrak{g}$ be a Lie algebra over a field $k$, and let
$B$ be a $k$-basis of $\mathfrak{g}$ equipped with a linear
order $\leq$. The {\em Poincar\'e-Birkhoff-Witt-theorem} (often
abbreviated to {\em PBW-theorem}) states that the monomials
\[ x_1 x_2 \cdots x_n \text{ with } x_1 \leq x_2 \leq \cdots \leq x_n \text{
elements of } B \]
constitute a $k$-basis of the universal enveloping algebra
$U(\mathfrak{g})$ of $\mathfrak{g}$. Such monomials are often called
{\em ordered monomials}  or {\em PBW-monomials}.

It is easy to see that they span $U(\mathfrak{g})$: for all $n \in
\NN$, let $M_n$ denote the set
\[ M_n=\{(x_1,\ldots,x_n)\mid x_1 \leq \cdots \leq x_n\} \subset B^n, \]
and denote by $\pi:\bigcup_{n=0}^\infty B^n \rightarrow U(\mathfrak{g})$ the
multiplication map. Clearly it suffices to prove that
\[ \pi(B^n) \subseteq \sum_{i=0}^n \pi(M_i) \]
for all $n \in \NN$; to this end, we proceed by induction.  For $n=0$
the statement is clear. Assume that it holds for $n-1\geq 0$, and consider a
list $(x_1,\ldots,x_n) \in B^n$. If it is an element of $M_n$, then we are
done. Otherwise, there exists an index $i$ such that $x_i>x_{i+1}$.
Now we have
\begin{align*}
\pi(x_1,\ldots,x_n)&=\pi(x_1,\ldots,x_{i-1},x_{i+1},x_i,x_{i+2},\ldots,x_n)\\
&+x_1\cdots x_{i-1}[x_i,x_{i+1}]x_{i+1}\cdots x_n.
\end{align*}
As $B$ is a basis of $\mathfrak{k}$, $[x_i,x_{i+1}]$ is a linear
combination of $B$. Using this to expand the second term above, we find
that it is in $\sum_{i=0}^{n-1} \pi(M_i)$ by the induction hypothesis.
The argument of $\pi$ in the first term, on the other hand, is
lexicographically smaller than $(x_1,\ldots,x_n)$, but contains the
same entries. Clearly this rewriting proces must end, and this
concludes the induction step.

The proof of linear independence of the PBW-monomials is slightly more
difficult, but can be found in most introductory texts on Lie algebras, such as the classic below.

\begin{thebibliography}{9}
\bibitem{jacobson}
N.\@ Jacobson. \emph{\PMlinkescapetext{Lie Algebras}}. Dover Publications, New York, 1979
\end{thebibliography}
%%%%%
%%%%%
\end{document}
