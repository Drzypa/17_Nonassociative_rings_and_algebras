\documentclass[12pt]{article}
\usepackage{pmmeta}
\pmcanonicalname{JordanAlgebra}
\pmcreated{2013-03-22 14:52:15}
\pmmodified{2013-03-22 14:52:15}
\pmowner{CWoo}{3771}
\pmmodifier{CWoo}{3771}
\pmtitle{Jordan algebra}
\pmrecord{10}{36547}
\pmprivacy{1}
\pmauthor{CWoo}{3771}
\pmtype{Definition}
\pmcomment{trigger rebuild}
\pmclassification{msc}{17C05}
\pmsynonym{Jordan homomorphism}{JordanAlgebra}
\pmsynonym{Jordan isomorphism}{JordanAlgebra}
\pmdefines{Jordan identity}
\pmdefines{special Jordan algebra}
\pmdefines{exceptional Jordan algebra}
\pmdefines{Jordan algebra homomorphism}
\pmdefines{Jordan subalgebra}
\pmdefines{Jordan algebra isomorphism}

% this is the default PlanetMath preamble.  as your knowledge
% of TeX increases, you will probably want to edit this, but
% it should be fine as is for beginners.

% almost certainly you want these
\usepackage{amssymb,amscd}
\usepackage{amsmath}
\usepackage{amsfonts}

% used for TeXing text within eps files
%\usepackage{psfrag}
% need this for including graphics (\includegraphics)
%\usepackage{graphicx}
% for neatly defining theorems and propositions
%\usepackage{amsthm}
% making logically defined graphics
%%%\usepackage{xypic}

% there are many more packages, add them here as you need them

% define commands here
\begin{document}
\PMlinkescapeword{identity}
\PMlinkescapeword{side}

Let $R$ be a commutative ring with $1\neq 0$.  An $R$-algebra $A$ with multiplication not assumed to be associative is called a (commutative) \emph{Jordan algebra} if
\begin{enumerate}
\item $A$ is commutative: $ab=ba$, and
\item $A$ satisfies the \emph{Jordan identity}: $(a^2b)a=a^2(ba)$,
\end{enumerate}
for any $a,b\in A$.

The above can be restated as 
\begin{enumerate}
\item $[A,A]=0$, where $[\ , ]$ is the commutator bracket, and 
\item for any $a\in A$, $[a^2,A,a]=0$, where $[\ , , ]$ is the associator bracket.
\end{enumerate}
If $A$ is a Jordan algebra, a subset $B\subseteq A$ is called a \emph{Jordan subalgebra} if $BB\subseteq B$.  Let $A$ and $B$ be two Jordan algebras.  A \emph{Jordan algebra homomorphism}, or simply \emph{Jordan homomorphism}, from $A$ to $B$ is an algebra homomorphism that respects the above two laws.  A \emph{Jordan algebra isomorphism} is just a bijective Jordan algebra homomorphism.

\textbf{Remarks}.
\begin{itemize}
\item If $A$ is a Jordan algebra such that $\operatorname{char}(A)\neq2$, then $A$ is \PMlinkname{power-associative}{PowerAssociativeAlgebra}.
\item If in addition $2=1+1\neq\operatorname{char}(A)$, then by replacing $a$ with $a+1$ in the Jordan identity and simplifying, $A$ is \PMlinkname{flexible}{FlexibleAlgebra}.
\item Given any associative algebra $A$, we can define a Jordan algebra $A^{+}$.  To see this, let $A$ be an associative algebra with associative multiplication $\cdot$ and suppose $2=1+1$ is invertible in $R$.  Define a new multiplication given by 
\begin{equation}
ab=\frac{1}{2}(a\cdot b+b\cdot a).
\end{equation}
It is readily checked that this new multiplication satisifies both the commutative law and the Jordan identity.  Thus $A$ with the new multiplication is a Jordan algebra and we denote it by $A^{+}$.  However, unlike Lie algebras, not every Jordan algebra is embeddable in an associative algebra.  Any Jordan algebra that is isomorphic to a Jordan subalgebra of $A^{+}$ for some associative algebra $A$ is called a \emph{special Jordan algebra}.  Otherwise, it is called an \emph{exceptional Jordan algebra}.  As a side note, the right hand side of Equation (1) is called the \emph{Jordan product}.
\item An example of an exceptional Jordan algebra is $H_3(\mathbb{O})$, the algebra of $3\times3$ Hermitian matrices over the octonions.
\end{itemize}
%%%%%
%%%%%
\end{document}
