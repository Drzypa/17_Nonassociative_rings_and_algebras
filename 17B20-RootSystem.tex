\documentclass[12pt]{article}
\usepackage{pmmeta}
\pmcanonicalname{RootSystem}
\pmcreated{2013-03-22 13:11:30}
\pmmodified{2013-03-22 13:11:30}
\pmowner{rmilson}{146}
\pmmodifier{rmilson}{146}
\pmtitle{root system}
\pmrecord{13}{33645}
\pmprivacy{1}
\pmauthor{rmilson}{146}
\pmtype{Definition}
\pmcomment{trigger rebuild}
\pmclassification{msc}{17B20}
\pmrelated{SimpleAndSemiSimpleLieAlgebras2}
\pmrelated{LieAlgebra}
\pmdefines{reduced root system}
\pmdefines{root}
\pmdefines{root space}
\pmdefines{root decomposition}
\pmdefines{indecomposable}
\pmdefines{reduced}
\pmdefines{crystallographic}

\endmetadata

\usepackage{amsmath}
\usepackage{amsfonts}
\usepackage{amssymb}
\begin{document}
A root system is a key notion in the classification and the
representation theory of reflection groups and of semi-simple Lie
algebras.  Let $E$ be a Euclidean vector space with inner product
$(\cdot,\cdot)$. A root system is a finite spanning set $R\subset E$
such that for every $u\in R$, the orthogonal reflection $$v\mapsto
v-2\frac{(u,v)}{(u,u)} u,\quad v\in E$$
preserves $R$.

A root system is called \emph{crystallographic} if
$2\frac{(u,v)}{(u,u)}$ is an integer for all $u,v\in R$.

A root system is called {\em reduced} if for all $u\in R$, we have
$ku\in R$ for $k=\pm 1$ only.

We call a root system {\em indecomposable} if there is no proper
decomposition $R=R'\cup R''$ such that every vector in $R'$ is orthogonal to
every vector in $R''$.
%%%%%
%%%%%
\end{document}
