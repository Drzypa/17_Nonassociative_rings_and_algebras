\documentclass[12pt]{article}
\usepackage{pmmeta}
\pmcanonicalname{QuadraticLieAlgebra}
\pmcreated{2013-03-22 15:30:44}
\pmmodified{2013-03-22 15:30:44}
\pmowner{benjaminfjones}{879}
\pmmodifier{benjaminfjones}{879}
\pmtitle{quadratic Lie algebra}
\pmrecord{6}{37379}
\pmprivacy{1}
\pmauthor{benjaminfjones}{879}
\pmtype{Definition}
\pmcomment{trigger rebuild}
\pmclassification{msc}{17B10}
\pmclassification{msc}{17B01}
%\pmkeywords{quadratic}
%\pmkeywords{invariant scalar product}
%\pmkeywords{Lie algebra}
\pmrelated{quadraticAlgebra}
\pmdefines{quadratic Lie algebra}

\endmetadata

% this is the default PlanetMath preamble.  as your knowledge
% of TeX increases, you will probably want to edit this, but
% it should be fine as is for beginners.

% almost certainly you want these
\usepackage{amssymb}
\usepackage{amsmath}
\usepackage{amsfonts}

% used for TeXing text within eps files
%\usepackage{psfrag}
% need this for including graphics (\includegraphics)
%\usepackage{graphicx}
% for neatly defining theorems and propositions
%\usepackage{amsthm}
% making logically defined graphics
%%%\usepackage{xypic}

% there are many more packages, add them here as you need them

% define commands here
\begin{document}
A Lie algebra $\mathfrak{g}$ is said to be \emph{quadratic} if $\mathfrak{g}$ as a representation (under the adjoint action) admits a non-degenerate, invariant scalar product $( \cdot \mid \cdot )$ .

$\mathfrak{g}$ being quadratic implies that the adjoint and co-adjoint representations of $\mathfrak{g}$ are isomorphic.

Indeed, the non-degeneracy of $( \cdot \mid \cdot )$ implies that the induced map $\phi \colon \mathfrak{g} \to \mathfrak{g}^*$ given by $\phi(X)(Z) = (X \mid Z)$ is an isomorphism of vector spaces. Invariance of the scalar product means that 
$( [X,Y] \mid Z) = -(Y \mid [X,Z] ) = (Y \mid [Z,X] )$. This implies that $\phi$ is a map of representations: 

\[ \phi(ad_X(Y))(Z) = \phi([X,Y])(Z) = ( [X,Y] \mid Z ) = ( Y \mid [Z,X] ) = ad^*_X( \phi(Y)(Z) ) \]
%%%%%
%%%%%
\end{document}
