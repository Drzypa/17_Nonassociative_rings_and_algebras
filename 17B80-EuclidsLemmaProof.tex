\documentclass[12pt]{article}
\usepackage{pmmeta}
\pmcanonicalname{EuclidsLemmaProof}
\pmcreated{2013-03-22 11:47:11}
\pmmodified{2013-03-22 11:47:11}
\pmowner{akrowne}{2}
\pmmodifier{akrowne}{2}
\pmtitle{Euclid's lemma proof}
\pmrecord{9}{30258}
\pmprivacy{1}
\pmauthor{akrowne}{2}
\pmtype{Proof}
\pmcomment{trigger rebuild}
\pmclassification{msc}{17B80}
\pmclassification{msc}{81T30}
\pmclassification{msc}{11A05}
\pmclassification{msc}{81-00}

\endmetadata

\usepackage{amssymb}
\usepackage{amsmath}
\usepackage{amsfonts}
\usepackage{graphicx}
%%%%\usepackage{xypic}
\begin{document}
We have $a|bc$, so $bc=na$, with $n$ an integer.  Dividing both sides by $a$, we have $$\frac{bc}{a}=n$$  But $\gcd(a,b)=1$ implies $b/a$ is only an integer if $a=1$.  So $$\frac{bc}{a} = b \frac{c}{a} = n $$ which means $a$ must divide $c$.

Note that this proof relies on the Fundamental Theorem of Arithmetic.  The alternative proof of Euclid's lemma avoids this.
%%%%%
%%%%%
%%%%%
%%%%%
\end{document}
