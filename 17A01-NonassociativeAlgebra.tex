\documentclass[12pt]{article}
\usepackage{pmmeta}
\pmcanonicalname{NonassociativeAlgebra}
\pmcreated{2013-03-22 15:06:44}
\pmmodified{2013-03-22 15:06:44}
\pmowner{CWoo}{3771}
\pmmodifier{CWoo}{3771}
\pmtitle{non-associative algebra}
\pmrecord{10}{36845}
\pmprivacy{1}
\pmauthor{CWoo}{3771}
\pmtype{Definition}
\pmcomment{trigger rebuild}
\pmclassification{msc}{17A01}
\pmrelated{Semifield}
\pmrelated{Algebras}
\pmdefines{non-associative ring}

\endmetadata

% this is the default PlanetMath preamble.  as your knowledge
% of TeX increases, you will probably want to edit this, but
% it should be fine as is for beginners.

% almost certainly you want these
\usepackage{amssymb,amscd}
\usepackage{amsmath}
\usepackage{amsfonts}

% used for TeXing text within eps files
%\usepackage{psfrag}
% need this for including graphics (\includegraphics)
%\usepackage{graphicx}
% for neatly defining theorems and propositions
%\usepackage{amsthm}
% making logically defined graphics
%%%\usepackage{xypic}

% there are many more packages, add them here as you need them

% define commands here
\begin{document}
A \emph{non-associative algebra} is an algebra in which the assumption of multiplicative associativity is dropped.  From this definition, a non-associative algebra does not \PMlinkescapetext{mean} that the associativity fails.  Rather, it enlarges the class of associative algebras, so that any associative algebra is a non-associative algebra.  

In much of the literature concerning non-associative algebras, where the meaning of a ``non-associative algebra'' is clear, the word ``non-associative'' is dropped for simplicity and clarity.

Lie algebras and Jordan algebras are two famous examples of non-associative algebras that are not associative.

If we substitute the word ``algebra'' with ``ring'' in the above paragraphs, then we arrive at the definition of a \emph{non-associative ring}.  Alternatively, a non-associative ring is just a non-associative algebra over the integers.

\begin{thebibliography}{6}
\bibitem{rs} Richard D. Schafer, {\em An Introduction to Nonassociative Algebras}, Dover Publications, (1995).
\end{thebibliography}
%%%%%
%%%%%
\end{document}
