\documentclass[12pt]{article}
\usepackage{pmmeta}
\pmcanonicalname{PositiveRoot}
\pmcreated{2013-03-22 13:11:46}
\pmmodified{2013-03-22 13:11:46}
\pmowner{mathwizard}{128}
\pmmodifier{mathwizard}{128}
\pmtitle{positive root}
\pmrecord{8}{33656}
\pmprivacy{1}
\pmauthor{mathwizard}{128}
\pmtype{Definition}
\pmcomment{trigger rebuild}
\pmclassification{msc}{17B20}
\pmdefines{negative root}

\endmetadata

% this is the default PlanetMath preamble.  as your knowledge
% of TeX increases, you will probably want to edit this, but
% it should be fine as is for beginners.

% almost certainly you want these
\usepackage{amssymb}
\usepackage{amsmath}
\usepackage{amsfonts}

% used for TeXing text within eps files
%\usepackage{psfrag}
% need this for including graphics (\includegraphics)
%\usepackage{graphicx}
% for neatly defining theorems and propositions
%\usepackage{amsthm}
% making logically defined graphics
%%%\usepackage{xypic}

% there are many more packages, add them here as you need them

% define commands here
\begin{document}
If $R\subset E$ is a root system, with $E$ an Euclidean vector space, then a subset 
$R^+\subset R$ is called a set of positive roots if there is a vector $v\in E$ such that
$(\alpha ,v)>0$ if $\alpha\in R^+$, and $(\alpha ,v)<0$ if $\alpha\in R\backslash R^+$.
\PMlinkid{Roots}{3645} which are not positive are called \emph{negative}.  Since $-\alpha$ is negative if and only if
$\alpha$ is positive, exactly half the \PMlinkescapetext{roots} must be positive.
%%%%%
%%%%%
\end{document}
