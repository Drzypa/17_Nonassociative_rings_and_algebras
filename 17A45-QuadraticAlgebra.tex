\documentclass[12pt]{article}
\usepackage{pmmeta}
\pmcanonicalname{QuadraticAlgebra}
\pmcreated{2013-03-22 15:11:38}
\pmmodified{2013-03-22 15:11:38}
\pmowner{CWoo}{3771}
\pmmodifier{CWoo}{3771}
\pmtitle{quadratic algebra}
\pmrecord{4}{36950}
\pmprivacy{1}
\pmauthor{CWoo}{3771}
\pmtype{Definition}
\pmcomment{trigger rebuild}
\pmclassification{msc}{17A45}
\pmrelated{QuadraticLieAlgebra}

\endmetadata

\usepackage{amssymb,amscd}
\usepackage{amsmath}
\usepackage{amsfonts}

% used for TeXing text within eps files
%\usepackage{psfrag}
% need this for including graphics (\includegraphics)
%\usepackage{graphicx}
% for neatly defining theorems and propositions
%\usepackage{amsthm}
% making logically defined graphics
%%%\usepackage{xypic}

% define commands here
\begin{document}
A non-associative algebra $A$ (with unity $1_A$) over a commutative
ring $R$ (with unity $1_R$) is called a \emph{quadratic algebra} if
$A$ admits a quadratic form $Q\colon A\to R$ such that
\begin{enumerate}
\item $Q(1_A)=1_R$,
\item the quadratic equation $x^2-b(1_A,x)x+Q(x)1_A=0$ is satisfied
by all $x\in A$, where $b$ is the associated symmetric bilinear form
given by $b(x,y):=Q(x+y)-Q(x)-Q(y)$.
\end{enumerate}
%%%%%
%%%%%
\end{document}
