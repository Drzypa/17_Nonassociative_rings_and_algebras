\documentclass[12pt]{article}
\usepackage{pmmeta}
\pmcanonicalname{EngelsTheorem}
\pmcreated{2013-03-22 12:42:25}
\pmmodified{2013-03-22 12:42:25}
\pmowner{rmilson}{146}
\pmmodifier{rmilson}{146}
\pmtitle{Engel's theorem}
\pmrecord{5}{32991}
\pmprivacy{1}
\pmauthor{rmilson}{146}
\pmtype{Theorem}
\pmcomment{trigger rebuild}
\pmclassification{msc}{17B30}
\pmclassification{msc}{15A57}

\newcommand{\lag}{\mathfrak{g}}
\newcommand{\lah}{\mathfrak{h}}
\newcommand{\laf}{\mathfrak{f}}
\newcommand{\ad}{\mathop{\mathrm{ad}}\nolimits}
%\newcommand{\ker}{\mathop{\mathrm{ker}}\nolimits}
\newcommand{\End}{\mathop{\mathrm{End}}\nolimits}
\newcommand{\Nil}{\mathop{\mathrm{Nil}}\nolimits}
\newcommand{\cA}{\mathcal{A}}
\newcommand{\cD}{\mathcal{D}}
\newtheorem{theorem}{Theorem}
\newtheorem{lemma}{Lemma}

\usepackage{amsmath}
\usepackage{amsfonts}
\usepackage{amssymb}
\newcommand{\reals}{\mathbb{R}}
\newcommand{\natnums}{\mathbb{N}}
\newcommand{\cnums}{\mathbb{C}}
\newcommand{\znums}{\mathbb{Z}}
\newcommand{\lp}{\left(}
\newcommand{\rp}{\right)}
\newcommand{\lb}{\left[}
\newcommand{\rb}{\right]}
\newcommand{\supth}{^{\text{th}}}
\newtheorem{proposition}{Proposition}
\newtheorem{definition}[proposition]{Definition}
\newcommand{\nl}[1]{\PMlinkescapetext{{#1}}}
\newcommand{\pln}[2]{\PMlinkname{#1}{#2}}
\begin{document}
Before proceeding, it will be useful to recall the definition of a
nilpotent Lie algebra.  Let $\lag$ be a Lie algebra.  The lower central series of $\lag$ is defined to be the filtration of ideals
$$\cD_0\lag \supset \cD_1\lag \supset \cD_2\lag \supset\ldots,$$
where 
$$\cD_0\lag = \lag,\qquad \cD_{k+1}\lag = [\lag,\cD_k\lag],\quad k\in
\natnums.$$
To say that $\lag$ is nilpotent is to say that the lower
central  series has a trivial termination, i.e. that there exists a $k$
such that 
$$\cD_k\lag = 0,$$
or equivalently, that $k$ nested bracket operations
always vanish.

\begin{theorem}[Engel]
  Let $\lag\subset\End V$ be a Lie algebra of endomorphisms of a
finite-dimensional vector space $V$.  Suppose that all elements of
$\lag$ are nilpotent transformations. Then, $\lag$ is a nilpotent Lie
algebra.
\end{theorem}

\begin{lemma}
  Let $X:V\rightarrow V$ be a nilpotent endomorphism of a  vector
  space $V$. Then, the adjoint action
  $$\ad(X):\End V\rightarrow \End V$$
  is also a nilpotent endomorphism.
\end{lemma}
\paragraph{Proof.}
Suppose that $$X^k=0$$
for some $k\in\natnums$.  We will show that
$$\ad(X)^{2k-1}=0.$$
Note that 
$$\ad(X) = l(X)-r(X),$$
where
$$l(X), r(X):\End V\rightarrow \End V,$$
are the endomorphisms
corresponding, respectively, to left and right multiplication by $X$.
These two endomorphisms commute, and hence we can use the binomial
formula to write
$$\ad(X)^{2k-1} = \sum_{i=0}^{2k-1} (-1)^i\, l(X)^{2k-1-i}\, r(X)^i.$$
Each of terms in the above sum vanishes because
$$l(X)^k = r(X)^k = 0.$$
QED

\begin{lemma}
  Let $\lag$ be as in the theorem, and suppose, in addition, that
  $\lag$ is a nilpotent Lie algebra. Then the joint kernel,
  $$\ker \lag = \bigcap_{a\in\lag} \ker a,$$
  is non-trivial.
\end{lemma}

\paragraph{Proof.}  We proceed by induction on the dimension of $\lag$.  
The claim is true for dimension 1, because then $\lag$ is generated by
a single nilpotent transformation, and all nilpotent transformations
are singular.

Suppose then that the claim is true for all Lie algebras of dimension
less than $n=\dim\lag$. We note that $\cD_1\lag$ fits the hypotheses
of the lemma, and has dimension less than $n$, because $\lag$ is
nilpotent. Hence, by the induction hypothesis
$$V_0 = \ker \cD_1\lag$$
is non-trivial.  Now, if we restrict all
actions to $V_0$, we obtain a representation of $\lag$ by abelian
transformations. This is because for all $a,b\in\lag$ and $v\in V_0$
we have
$$a b v-b a v = [a,b]v = 0 .$$
Now a finite number of mutually
commuting linear endomorphisms admits a mutual eigenspace
decomposition.  In particular, if all of the commuting endomorphisms
are singular, their joint kernel will be non-trivial.  We apply
this result to a basis of $\lag/\cD_1\lag$ acting on $V_0$,
and the desired conclusion follows.  QED


\paragraph{Proof of the theorem.}

We proceed by induction on the dimension of $\lag$.  The theorem is
true in dimension 1, because in that circumstance $\cD_1\lag$ is
trivial.  

Next, suppose that the theorem holds for all Lie algebras of dimension
less than $n=\dim\lag$. Let $\lah\subset\lag$ be a properly contained
subalgebra of minimum codimension. We claim that there exists an
$a\in\lag$ but not in $\lah$ such that $[a,\lah]\subset\lah$.

By the induction hypothesis, $\lah$ is nilpotent. To prove the claim
consider the isotropy representation of $\lah$ on $\lag/\lah$.  By
Lemma 1, the action of each $a\in\lah$ on $\lag/\lah$ is a nilpotent
endomorphism.  Hence, we can apply Lemma 2 to deduce that the joint
kernel of all these actions is non-trivial, i.e. there exists a
$a\in\lag$ but not in $\lah$ such that
$$[b,a]\equiv 0 \mod \lah,$$
for all $b\in\lah$. Equivalently, $[\lah,a]\subset \lah$ and the claim
is proved.

Evidently then, the span of $a$ and $\lah$ is a subalgebra of $\lag$.
Since $\lah$ has minimum codimension, we infer that $\lah$ and $a$
span all of $\lag$, and that 
\begin{equation}
  \label{eq:g1inh}
  \cD_1\lag\subset\lah.  
\end{equation}

Next, we claim that all the $\cD_k\lah$ are ideals of $\lag$.  It is
enough to show that
$$[a,\cD_k\lah]\subset \cD_k\lah.$$
We argue by induction on $k$. Suppose the
claim is true for some $k$. Let $b\in\lah, c\in\cD_k\lah$ be given. By
the Jacobi identity
$$[a,[b,c]] = [[a,b],c] + [b,[a,c]].$$
The first term on the right hand-side in $\cD_{k+1}\lah$ because
$[a,b]\in\lah$.  The second term is in $\cD_{k+1}\lah$ by the induction
hypothesis.  In this way the claim is established.

Now $a$ is nilpotent, and hence by Lemma 1, 
\begin{equation}
  \label{eq:adan}
  \ad(a)^n=0  
\end{equation}
for some $n\in \natnums$.  We now claim that 
$$\cD_{n+1}\lag\subset\cD_1\lah.$$
By \eqref{eq:g1inh} it suffices to show that 
$$[\overbrace{\lag,[\ldots
  [\lag}^{n\text{ times}},\lah]\ldots ]]\subset\cD_1\lah.$$
Putting
$$\lag_1=\lag/\cD_1\lah,\quad \lah_1=\lah/\cD_1\lah, $$
this is equivalent to
$$[\overbrace{\lag_1,[\ldots
  [\lag_1}^{n\text{ times}},\lah_1]\ldots ]]=0.$$
However, $\lah_1$ is abelian, and hence, the above follows directly
from \eqref{eq:adan}.

Adapting this argument in the obvious fashion we can show that
$$\cD_{kn+1}\lag\subset\cD_k \lah.$$
Since $\lah$ is nilpotent, $\lag$ must be nilpotent as well. QED

\paragraph{Historical remark.}
In the traditional formulation of Engel's theorem, the hypotheses are
the same, but the conclusion is that there exists a basis $B$ of $V$,
such that all elements of $\lag$ are represented by nilpotent matrices
relative to $B$.   

Let us put this another way.  The vector space of nilpotent matrices
$\Nil$, is a nilpotent Lie algebra, and indeed all subalgebras of $\Nil$ are
nilpotent Lie algebras.  Engel's theorem asserts that the converse
holds, i.e. if all elements of a Lie algebra $\lag$ are  nilpotent
transformations, then $\lag$ is isomorphic to a subalgebra of $\Nil$.

The classical result follows straightforwardly from our version of the
Theorem and from Lemma 2.  Indeed, let $V_1$ be the joint kernel
$\lag$.  We then let $U_2$ be the joint kernel of $\lag$ acting on
$V/V_0$, and let $V_2\subset V$ be the subspace obtained by pulling
$U_2x$ back to $V$.  We do this a finite number of times and obtain a
flag of subspaces
$$0=V_0\subset V_1\subset V_2\subset \ldots \subset V_n=V,$$
such that
$$\lag V_{k+1} = V_k$$
for all $k$.  The choose an adapted basis
relative to this flag, and we're done.
%%%%%
%%%%%
\end{document}
