\documentclass[12pt]{article}
\usepackage{pmmeta}
\pmcanonicalname{TraceFormsOnAlgebras}
\pmcreated{2013-03-22 16:28:01}
\pmmodified{2013-03-22 16:28:01}
\pmowner{Algeboy}{12884}
\pmmodifier{Algeboy}{12884}
\pmtitle{trace forms on algebras}
\pmrecord{4}{38627}
\pmprivacy{1}
\pmauthor{Algeboy}{12884}
\pmtype{Topic}
\pmcomment{trigger rebuild}
\pmclassification{msc}{17A01}
\pmdefines{regular representation}
\pmdefines{trace form}

\endmetadata

\usepackage{latexsym}
\usepackage{amssymb}
\usepackage{amsmath}
\usepackage{amsfonts}
\usepackage{amsthm}

\DeclareMathOperator{\End}{End}
\DeclareMathOperator{\ad}{ad~}
\DeclareMathOperator{\tr}{tr~}
%%\usepackage{xypic}

%-----------------------------------------------------

%       Standard theoremlike environments.

%       Stolen directly from AMSLaTeX sample

%-----------------------------------------------------

%% \theoremstyle{plain} %% This is the default

\newtheorem{thm}{Theorem}

\newtheorem{coro}[thm]{Corollary}

\newtheorem{lem}[thm]{Lemma}

\newtheorem{lemma}[thm]{Lemma}

\newtheorem{prop}[thm]{Proposition}

\newtheorem{conjecture}[thm]{Conjecture}

\newtheorem{conj}[thm]{Conjecture}

\newtheorem{defn}[thm]{Definition}

\newtheorem{remark}[thm]{Remark}

\newtheorem{ex}[thm]{Example}



%\countstyle[equation]{thm}



%--------------------------------------------------

%       Item references.

%--------------------------------------------------


\newcommand{\exref}[1]{Example-\ref{#1}}

\newcommand{\thmref}[1]{Theorem-\ref{#1}}

\newcommand{\defref}[1]{Definition-\ref{#1}}

\newcommand{\eqnref}[1]{(\ref{#1})}

\newcommand{\secref}[1]{Section-\ref{#1}}

\newcommand{\lemref}[1]{Lemma-\ref{#1}}

\newcommand{\propref}[1]{Prop\-o\-si\-tion-\ref{#1}}

\newcommand{\corref}[1]{Cor\-ol\-lary-\ref{#1}}

\newcommand{\figref}[1]{Fig\-ure-\ref{#1}}

\newcommand{\conjref}[1]{Conjecture-\ref{#1}}


% Normal subgroup or equal.

\providecommand{\normaleq}{\unlhd}

% Normal subgroup.

\providecommand{\normal}{\lhd}

\providecommand{\rnormal}{\rhd}
% Divides, does not divide.

\providecommand{\divides}{\mid}

\providecommand{\ndivides}{\nmid}


\providecommand{\union}{\cup}

\providecommand{\bigunion}{\bigcup}

\providecommand{\intersect}{\cap}

\providecommand{\bigintersect}{\bigcap}










\begin{document}
Given an finite dimensional algebra $A$ over a field $k$ we define the 
left(right) regular representation of $A$ as the map $L:A\rightarrow \End_k A$
given by $L_a b:=ab$ ($R_a b:=ba$).  

\begin{ex}
In a Lie algebra the left representation is called the \emph{adjoint}
representation and denoted $\ad~x$ and defined $(\ad~x)(y)=[x,y]$.  Because
$[x,y]=-[y,x]$ in characteristic not 2, there is generally no distinction of
left/right adjoint representations.
\end{ex}

The \emph{trace form} of $A$ is defined as 
$\langle,\rangle:T\times T\rightarrow k$:
  \[\langle a,b\rangle:=\tr(L_a L_b).\]

\begin{prop}
The trace form is a symmetric bilinear form.
\end{prop}
\begin{proof}
Given $a,b,x\in A$ and $l\in k$ then $L_{a+lb} x=(a+lb)x=ax+lbx=L_{a}x+lL_{b}x$.
So $L_{a+lb}=L_a+lL_b$.  So we have
   \[\langle a+lb,x\rangle=\tr(L_{a+lb}L_x)=\tr(L_a L_x+lL_b L_x)
            =\tr(L_a L_x)+l\tr(L_b L_x)=\langle a,x\rangle
                      +l\langle b,x\rangle.\]
Furthermore, $\tr(fg)=\tr(gf)$ is general property of traces, thus
  \[\langle a,b\rangle=\tr(L_a L_b)=\tr(L_bL_a)=\langle b,a\rangle.\]
So the trace form is a symmetric bilinear form.
\end{proof}

The symmetric property can be interpreted as a weak form of commutativity of
the product: $a,b\in A$ commute within their trace from.  A more essential
property arises for certain algebras and can be interpreted as ``the product
is associative within the trace'' and written as
\begin{equation}\label{eq:assoc}
\langle ab,c\rangle=\langle a,bc\rangle.
\end{equation}
We shall call such an algebra \emph{weakly associative} though the term is not standard.

This property is clear for all associative algebras as:
\[L_{ab}L_c(x)=((ab)c)x=(a(bc))x=L_a L_{bc} x.\]
When we use a Lie algebra, the trace form is commonly called the \emph{Killing form} which has property (\ref{eq:assoc}).  A result of Koecher shows that Jordan algebras also have this property.

\begin{prop}
Given a weakly associative algebra, then the radical of the trace form
is an ideal of the algebra. 
\end{prop}
\begin{proof}
We know the radical of form $R$ is a subspace so we must simply show that
$R$ is an ideal.  Given $x\in R$ and $y\in A$ then for all $z\in A$,
$\langle xy,z\rangle=\langle x,yz\rangle=0$.  Thus $xy\in R$.  Likewise
$yz\in R$ so $R$ is a two-sided ideal of $A$.
\end{proof}

From this result many authors define an algebra to be semi-simple if its trace
form is non-degenerate.  In this way, $A/R$, $R$ the radical of $A$, is semi-simple.  [Some variations on this definition are often required over small
fields/characteristics, especially when characteristic is 2.]

More can be said when ideals are considered.

\begin{prop}
Given a weakly associative algebra $A$, then
if $I$ is an ideal of $A$ then so is $I^\perp$.
\end{prop}
\begin{proof}
Given $a\in I^\perp$, then for all $b\in A$ and $c\in I$, then $bc\in I$
as $I$ is an ideal and so $\langle ab,c\rangle=\langle a,bc\rangle=0$
as $a\in I^\perp$.  This makes $ab\in I^\perp$ so $I$ is a right ideal.
Likewise $\langle c,ba\rangle=\langle cb,a\rangle=0$ so $ba\in I^\perp$ and
thus $I^\perp$ is an ideal of $A$.
\end{proof}

To proceed one factors out the radical so that $A$ is semisimple.  Then
given an ideal $I$ of $A$, if $I\intersect I^\perp=0$ then as the trace form
is a non-degenerate bilinear from, $A=I\oplus I^\perp$, and so by iterating
we produce a decomposition of $A$ into minimal ideals:
 \[A=A_1\oplus \cdots\oplus A_s.\]
Hence we arrive at the alternative definition of a semisimple algebra: that the
algebra be a direct product of simple algebras.  To obtain the property $I\intersect I^\perp=0$ it is sufficient to assume $A$ has not ideal $I$ 
such that $I^2=0$.  This is the content of the proof in

\begin{thm}\cite[Thm III.3]{Jac}
Let $A$ be a finite-dimensional weakly associative (trace) semisimple 
algebra over a field $k$ in which no ideal $I\neq 0$ of $A$ has $I^2=0$,
then $A$ is a direct product of minimal ideals, that is, of simple
algebras.
\end{thm}

Alternatively any bilinear form with (\ref{eq:assoc}) can be used.  However,
the trace form is always definable and the desired properties are easily translated into implications about the multiplication of the algebra.

\begin{thebibliography}{8}
\bibitem{Jac}
Jacobson, Nathan \emph{Lie Algebras}, Interscience Publishers, New York, 1962.\\

\bibitem{Koe}
Koecher, Max, \emph{The Minnesota notes on Jordan algebras and their applications}. 
	Edited and annotated by Aloys Krieg and Sebastian Walcher. 
	[B] Lecture Notes in Mathematics 1710. Berlin: Springer. (1999).
\end{thebibliography}

%%%%%
%%%%%
\end{document}
