\documentclass[12pt]{article}
\usepackage{pmmeta}
\pmcanonicalname{CommutatorBracket}
\pmcreated{2013-03-22 12:33:51}
\pmmodified{2013-03-22 12:33:51}
\pmowner{rmilson}{146}
\pmmodifier{rmilson}{146}
\pmtitle{commutator bracket}
\pmrecord{8}{32811}
\pmprivacy{1}
\pmauthor{rmilson}{146}
\pmtype{Definition}
\pmcomment{trigger rebuild}
\pmclassification{msc}{17A01}
\pmclassification{msc}{17B05}
\pmclassification{msc}{18A40}
\pmrelated{LieAlgebra}
\pmdefines{commutator Lie algebra}
\pmdefines{commutator}

\endmetadata

\usepackage{amsmath}
\usepackage{amsfonts}
\usepackage{amssymb}

\newcommand{\End}{\mathrm{End}}
\newcommand{\reals}{\mathbb{R}}
\newcommand{\natnums}{\mathbb{N}}
\newcommand{\cnums}{\mathbb{C}}
\newcommand{\znums}{\mathbb{Z}}

\newcommand{\lp}{\left(}
\newcommand{\rp}{\right)}
\newcommand{\lb}{\left[}
\newcommand{\rb}{\right]}

\newcommand{\supth}{^{\text{th}}}


\newtheorem{proposition}{Proposition}
\begin{document}
Let $A$ be an associative algebra over a field $K$.  For $a,b \in A$,
the element of $A$ defined by
$$[a,b]=ab-ba$$
is called the {\em commutator} of $a$ and $b$.
The corresponding bilinear operation
$$[-,-]: A\times A\rightarrow A$$
is called the commutator bracket.

The commutator bracket is bilinear, skew-symmetric, and also satisfies
the Jacobi identity.  To wit, for $a,b,c\in A$ we have
$$[a,[b,c]] + [b,[c,a]] + [c,[a,b]] = 0.$$
The proof of this assertion is straightforward.  Each of the brackets in
the left-hand side expands to 4 terms, and then everything cancels.

In categorical terms, what we have here is a functor from the category
of associative algebras to the category of Lie algebras over a fixed
field.  The action of this functor is to turn an associative algebra
$A$ into a Lie algebra that has the same underlying vector space as
$A$, but whose multiplication operation is given by the commutator
bracket.  It must be noted that this functor is right-adjoint to the
universal enveloping algebra functor.

\paragraph{Examples}
\begin{itemize}
\item 
Let $V$ be a vector space.  Composition endows the vector space of
endomorphisms $\End V$ with the structure of an associative algebra.
However, we could also regard $\End V$ as a Lie algebra relative to
the commutator bracket:
$$[X,Y] = XY-YX,\quad X,Y\in \End V.$$
\item The algebra of differential operators has some interesting
  properties when viewed as a Lie algebra.  The fact is that even
  though the composition of differential operators is a
  non-commutative operation, it is commutative when restricted to the
  highest order terms of the involved operators.  Thus, if $X, Y$ are
  differential operators of order $p$ and $q$, respectively, the
  compositions $XY$ and $YX$ have order $p+q$.  Their highest order
  term coincides, and hence the commutator $[X,Y]$ has order $p+q-1$.
\item In light of the preceding comments, it is evident that the
  vector space of first-order differential operators is closed with
  respect to the commutator bracket.  Specializing even further we
  remark that, a vector field is just a homogeneous first-order
  differential operator, and that the commutator bracket for vector
  fields, when viewed as first-order operators, coincides with the
  usual, geometrically motivated vector field bracket.
\end{itemize}
%%%%%
%%%%%
\end{document}
