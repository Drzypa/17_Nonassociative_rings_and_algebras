\documentclass[12pt]{article}
\usepackage{pmmeta}
\pmcanonicalname{JacobiIdentityInterpretations}
\pmcreated{2013-03-22 13:03:42}
\pmmodified{2013-03-22 13:03:42}
\pmowner{rspuzio}{6075}
\pmmodifier{rspuzio}{6075}
\pmtitle{Jacobi identity interpretations}
\pmrecord{8}{33468}
\pmprivacy{1}
\pmauthor{rspuzio}{6075}
\pmtype{Definition}
\pmcomment{trigger rebuild}
\pmclassification{msc}{17B99}

\usepackage{amssymb}
\usepackage{amsmath}
\usepackage{amsfonts}
\newcommand{\ad}{\operatorname{ad}}
\newcommand{\End}{\operatorname{End}}
\begin{document}
The Jacobi identity in a Lie algebra $\mathfrak{g}$ has various interpretations that are more transparent, whence easier to remember, than the usual form
\[ [x,[y,z]]+[y,[z,x]]+[z,[x,y]]=0. \]
One is the fact that the adjoint representation 
\footnote{Here, ``$\mathfrak{gl}(\mathfrak{g})$'' means the space o
endomorphisms of $\mathfrak{g}$, viewed as a vector space, with Lie
bracket on $\mathfrak{gl}(\mathfrak{g})$being commutator.}
$\ad:\mathfrak{g} \rightarrow \mathfrak{gl}(\mathfrak{g})$ really is a representation. Yet another way to formulate the identity is 
\[ \ad(x)[y,z]=[\ad(x)y,z]+[y,\ad(x)z], \]
i.e., $\ad(x)$ is a derivation on $\mathfrak{g}$ for all $x \in \mathfrak{g}$.
%%%%%
%%%%%
\end{document}
