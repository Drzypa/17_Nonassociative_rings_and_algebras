\documentclass[12pt]{article}
\usepackage{pmmeta}
\pmcanonicalname{LieAlgebraRepresentation}
\pmcreated{2013-03-22 12:41:13}
\pmmodified{2013-03-22 12:41:13}
\pmowner{mathcam}{2727}
\pmmodifier{mathcam}{2727}
\pmtitle{Lie algebra representation}
\pmrecord{16}{32966}
\pmprivacy{1}
\pmauthor{mathcam}{2727}
\pmtype{Definition}
\pmcomment{trigger rebuild}
\pmclassification{msc}{17B10}
\pmsynonym{representation}{LieAlgebraRepresentation}
\pmrelated{Dimension3}
\pmdefines{irreducible}
\pmdefines{module}
\pmdefines{dimension}
\pmdefines{finite dimensional}
\pmdefines{finite-dimensional}
\pmdefines{infinite dimensional}
\pmdefines{infinite-dimensional}
\pmdefines{faithful}
\pmdefines{direct sum of representations}

\newcommand{\lag}{\mathfrak{g}}
\newcommand{\ad}{\mathop{\mathrm{ad}}\nolimits}
\newcommand{\End}{\mathop{\mathrm{End}}\nolimits}
\usepackage{amsmath}
\usepackage{amsfonts}
\usepackage{amssymb}
\newcommand{\reals}{\mathbb{R}}
\newcommand{\natnums}{\mathbb{N}}
\newcommand{\cnums}{\mathbb{C}}
\newcommand{\znums}{\mathbb{Z}}
\newcommand{\lp}{\left(}
\newcommand{\rp}{\right)}
\newcommand{\lb}{\left[}
\newcommand{\rb}{\right]}
\newcommand{\supth}{^{\text{th}}}
\newtheorem{proposition}{Proposition}
\newtheorem{definition}[proposition]{Definition}
\newcommand{\nl}[1]{\PMlinkescapetext{{#1}}}
\newcommand{\pln}[2]{\PMlinkname{#1}{#2}}
\begin{document}
A representation of a Lie algebra $\lag$ is a Lie algebra homomorphism
$$\rho:\lag \rightarrow \End V,$$
where $\End V$ is the commutator Lie
algebra of some vector space $V$.  In other words, $\rho$ is a linear
mapping that satisfies
$$\rho([a,b]) = \rho(a)\rho(b)-\rho(b)\rho(a),\quad a,b\in\lag$$
Alternatively, one calls $V$ a $\lag$-module, and calls $\rho(a),\,
a\in \lag$ the action of $a$ on $V$.

We call the representation {\em faithful} if $\rho$ is injective.

A invariant subspace or sub-module $W\subset V$ is a subspace of $V$ satisfying $\rho(a)(W)\subset W$ for all $a\in\lag$.  A representation is
called {\em irreducible} or simple if its only invariant subspaces are $\{0\}$
and the whole representation.

The dimension of $V$ is called the dimension of the representation.
If $V$ is infinite-dimensional, then one speaks of an
infinite-dimensional representation.

Given a pair of representations, we can define a new representation, called the direct sum of the two given representations:

If $\rho:\lag\to\End(V)$ and $\sigma:\lag\to\End(W)$ are representations, then $V\oplus W$ has the obvious Lie algebra action, by the embedding $\End(V)\times\End(W)\hookrightarrow\End(V\oplus W)$.
%%%%%
%%%%%
\end{document}
