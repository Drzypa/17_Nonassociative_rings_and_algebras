\documentclass[12pt]{article}
\usepackage{pmmeta}
\pmcanonicalname{PowerassociativeAlgebra}
\pmcreated{2013-03-22 14:43:27}
\pmmodified{2013-03-22 14:43:27}
\pmowner{CWoo}{3771}
\pmmodifier{CWoo}{3771}
\pmtitle{power-associative algebra}
\pmrecord{15}{36350}
\pmprivacy{1}
\pmauthor{CWoo}{3771}
\pmtype{Definition}
\pmcomment{trigger rebuild}
\pmclassification{msc}{17A05}
\pmsynonym{di-associative}{PowerassociativeAlgebra}
\pmsynonym{diassociative}{PowerassociativeAlgebra}
\pmrelated{Associator}

\endmetadata

% this is the default PlanetMath preamble.  as your knowledge
% of TeX increases, you will probably want to edit this, but
% it should be fine as is for beginners.

% almost certainly you want these
\usepackage{amssymb,amscd}
\usepackage{amsmath}
\usepackage{amsfonts}

% used for TeXing text within eps files
%\usepackage{psfrag}
% need this for including graphics (\includegraphics)
%\usepackage{graphicx}
% for neatly defining theorems and propositions
%\usepackage{amsthm}
% making logically defined graphics
%%%\usepackage{xypic}

% there are many more packages, add them here as you need them

% define commands here
\begin{document}
Let $A$ be a non-associative algebra.  A subalgebra $B$ of $A$ is said to be \emph{cyclic} if it is generated by one element.

A non-associative algebra is \emph{power-associative} if, $[B,B,B]=0$ for any cyclic subalgebra $B$ of $A$, where $[-,-,-]$ is the associator.  

If we inductively define the powers of an element $a\in A$ by
\begin{enumerate}
\item (when $A$ is unital with $1\neq0$) $a^0:=1$,
\item $a^1:=a$, and 
\item $a^n:=a(a^{n-1})$ for $n>1$,
\end{enumerate}
then power-associativity of $A$ means that $[a^i,a^j,a^k]=0$ for any non-negative integers $i,j$ and $k$, since the associator is trilinear (linear in each of the three coordinates).  This implies that $a^ma^n=a^{m+n}$.  In addition, $(a^m)^n=a^{mn}$.

A theorem, due to A. Albert, states that any finite power-associative division algebra over the integers of characteristic not equal to 2, 3, or 5 is a field.  This is a generalization of the Wedderburn's Theorem on finite division rings.

\begin{thebibliography}{8}
\bibitem{Shafer} R. D. Schafer, {\em An Introduction on Nonassociative Algebras}, Dover, New York (1995).
\end{thebibliography}
%%%%%
%%%%%
\end{document}
