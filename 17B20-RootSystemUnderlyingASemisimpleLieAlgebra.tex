\documentclass[12pt]{article}
\usepackage{pmmeta}
\pmcanonicalname{RootSystemUnderlyingASemisimpleLieAlgebra}
\pmcreated{2013-03-22 15:28:59}
\pmmodified{2013-03-22 15:28:59}
\pmowner{rmilson}{146}
\pmmodifier{rmilson}{146}
\pmtitle{root system underlying a semi-simple Lie algebra}
\pmrecord{7}{37338}
\pmprivacy{1}
\pmauthor{rmilson}{146}
\pmtype{Result}
\pmcomment{trigger rebuild}
\pmclassification{msc}{17B20}
\pmrelated{SimpleAndSemiSimpleLieAlgebras2}
\pmdefines{Serre relations}
\pmdefines{Chevalley-Serre relations}

\endmetadata

\usepackage{amsmath}
\usepackage{amsfonts}
\usepackage{amssymb}

\newcommand{\Cset}{\mathbb{C}}

\newcommand{\fg}{\mathfrak{g}}
\newcommand{\fh}{\mathfrak{h}}
\DeclareMathOperator{\ad}{ad}
\begin{document}
Crystallographic, reduced root systems are in one-to-one
correspondence with semi-simple, complex Lie algebras.  First, let us
describe how one passes from a Lie algebra to a root system. Let $\fg$
be a semi-simple, complex Lie algebra and let $\fh$ be a Cartan
subalgebra.  Since $\fg$ is semi-simple, $\fh$ is abelian.  Moreover,
$\fh$ acts on $\fg$ (via the adjoint representation) by commuting,
simultaneously diagonalizable linear maps.  The simultaneous
eigenspaces of this $\fh$ action are called {\em root spaces}, and the
decomposition of $\fg$ into $\fh$ and the root spaces is called a {\em
  root decomposition} of $\fg$. To be more precise, for
$\lambda\in\fh^*$, set $$\fg_\lambda=\{a\in\fg \colon [h,a] =
\lambda(h) a \text{ for all } h\in \fh\}.$$
We call a non-zero
$\lambda\in\fh^*$ a root if $\fg_\lambda$ is non-trivial, in which
case $\fg_\lambda$ is called a root space.  It is possible to show
that that $\fg_0$ is just the Cartan subalgebra $\fh$, and that
$\dim\fg_\lambda=1$ for each root $\lambda$.  Letting $R\subset\fh^*$
denote the set of all roots, we have $$\fg=\fh\oplus
\bigoplus_{\lambda\in R} \fg_\lambda.$$

The Cartan subalgebra $\fh$ has a natural inner product, called the
Killing form, which in turn induces an inner product on $\fh^*$. It is
possible to show that, with respect to this inner product, $R$ is a
reduced, crystallographic root system.

Conversely, let $R\subset E$ be a reduced, crystallographic root
system. Let $\Delta$ be a base of positive roots.  We define a Lie
algebra by taking generators
\[H_\lambda,X_\lambda,Y_\lambda,\quad \lambda\in \Delta,\]
subject to the following relations:
\begin{align*}
  [H_\lambda,H_\mu]&=0,\\
  [H_\mu,X_\lambda]&=(\lambda,\mu) X_\lambda,\\
  [H_\mu,Y_\lambda]&=-(\lambda,\mu) Y_{\lambda},\\
  [X_\lambda,Y_\lambda] &= H_\lambda,\\
  [X_\lambda,Y_\mu] &= 0,\quad \lambda\neq \mu;\\
  (\ad X_\lambda)^{-(\lambda,\mu)+1} (X_\mu) &= 0,\quad \lambda\neq \mu,\\
  (\ad Y_\lambda)^{-(\lambda,\mu)+1}(Y_\mu) &= 0,\quad \lambda\neq \mu,\\
\end{align*}
The above are known as the Chevalley-Serre relations The resulting Lie
algebra turns out to be semi-simple, with a root system isomorphic to
the given $R$.


Thanks to the above isomorphism, to the difficult task of classifying
complex semi-simple Lie algebras is transformed into the somewhat
easier task of classifying crystallographic, reduced roots systems.
Furthermore, a complex Lie algebra is simple if and only if the
corresponding root system is indecomposable.  Thus, we only need to
classify indecomposable root systems, since all other root systems and
semi-simple Lie algebras are built out of these.
%%%%%
%%%%%
\end{document}
