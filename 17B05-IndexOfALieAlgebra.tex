\documentclass[12pt]{article}
\usepackage{pmmeta}
\pmcanonicalname{IndexOfALieAlgebra}
\pmcreated{2013-03-22 15:30:47}
\pmmodified{2013-03-22 15:30:47}
\pmowner{benjaminfjones}{879}
\pmmodifier{benjaminfjones}{879}
\pmtitle{index of a Lie algebra}
\pmrecord{6}{37380}
\pmprivacy{1}
\pmauthor{benjaminfjones}{879}
\pmtype{Definition}
\pmcomment{trigger rebuild}
\pmclassification{msc}{17B05}
%\pmkeywords{index}
%\pmkeywords{Lie algebra}
\pmdefines{index of a Lie algebra}
\pmdefines{Frobenius Lie algebra}
\pmdefines{Kirillov form}

% this is the default PlanetMath preamble.  as your knowledge
% of TeX increases, you will probably want to edit this, but
% it should be fine as is for beginners.

% almost certainly you want these
\usepackage{amssymb}
\usepackage{amsmath}
\usepackage{amsfonts}

% used for TeXing text within eps files
%\usepackage{psfrag}
% need this for including graphics (\includegraphics)
%\usepackage{graphicx}
% for neatly defining theorems and propositions
%\usepackage{amsthm}
% making logically defined graphics
%%%\usepackage{xypic}

% there are many more packages, add them here as you need them

% define commands here

\DeclareMathOperator{\ind}{ind \,}
\DeclareMathOperator{\rank}{rank \,}
%\DeclareMathOperator{\dim}{dim}
\begin{document}
Let $\mathfrak{q}$ be a Lie algebra over $\mathbb{K}$ and $\mathfrak{q}^*$ its vector space dual. For $\xi \in \mathfrak{q}^*$ let $\mathfrak{q}_{\xi}$ denote the stabilizer of $\xi$ with respect to the co-adjoint representation. 

The \emph{index} of $\mathfrak{q}$ is defined to be

\[ \ind \mathfrak{q} := \min\limits_{\xi \in \mathfrak{g}^*} \dim 
\mathfrak{q}_{\xi} \]

\section*{Examples}

\begin{enumerate}

\item If $\mathfrak{q}$ is reductive then $\ind \mathfrak{q} = \rank \mathfrak{q}$. Indeed, $\mathfrak{q}$ and $\mathfrak{q}^*$ are isomorphic as
representations for $\mathfrak{q}$ and so the index is the minimal dimension among stabilizers of elements in $\mathfrak{q}$. In particular the minimum is realized in the stabilizer of any \emph{regular} element of $\mathfrak{q}$. These elemtents have stabilizer dimension equal to the rank of $\mathfrak{q}$.

\item If $\ind \mathfrak{q} = 0$ then $\mathfrak{q}$ is called a 
\emph{Frobenius Lie algebra}. This is equivalent to condition that
the \emph{Kirillov form} $K_{\xi} \colon \mathfrak{q} \times \mathfrak{q} \to 
\mathbb{K}$ given by $(X,Y) \mapsto \xi([X,Y])$ is non-singular for some $\xi \in \mathfrak{q}^*$. Another equivalent condition when $\mathfrak{q}$ is the Lie algebra of an algebraic group $Q$ is that $\mathfrak{q}$ is Frobenius if and only if $Q$ has an open orbit on $\mathfrak{q}^*$.

\end{enumerate}
%%%%%
%%%%%
\end{document}
