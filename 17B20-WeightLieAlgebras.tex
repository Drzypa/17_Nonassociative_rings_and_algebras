\documentclass[12pt]{article}
\usepackage{pmmeta}
\pmcanonicalname{WeightLieAlgebras}
\pmcreated{2013-03-22 13:11:42}
\pmmodified{2013-03-22 13:11:42}
\pmowner{GrafZahl}{9234}
\pmmodifier{GrafZahl}{9234}
\pmtitle{weight (Lie algebras)}
\pmrecord{7}{33654}
\pmprivacy{1}
\pmauthor{GrafZahl}{9234}
\pmtype{Definition}
\pmcomment{trigger rebuild}
\pmclassification{msc}{17B20}
\pmsynonym{weight}{WeightLieAlgebras}
%\pmkeywords{representation}
%\pmkeywords{Cartan}
%\pmkeywords{Lie}
%\pmkeywords{abelian}
\pmdefines{diagonalisable}
\pmdefines{diagonalizable}
\pmdefines{multiplicity}
\pmdefines{weight space}

% this is the default PlanetMath preamble.  as your knowledge
% of TeX increases, you will probably want to edit this, but
% it should be fine as is for beginners.

% almost certainly you want these
\usepackage{amssymb}
\usepackage{amsmath}
\usepackage{amsfonts}
\usepackage[latin1]{inputenc}

% used for TeXing text within eps files
%\usepackage{psfrag}
% need this for including graphics (\includegraphics)
%\usepackage{graphicx}
% for neatly defining theorems and propositions
\usepackage{amsthm}
% making logically defined graphics
%%%\usepackage{xypic}

% there are many more packages, add them here as you need them

% define commands here
\newcommand{\Bigcup}{\bigcup\limits}
\newcommand{\DirectSum}{\bigoplus\limits}
\newcommand{\Prod}{\prod\limits}
\newcommand{\Sum}{\sum\limits}
\newcommand{\mbb}{\mathbb}
\newcommand{\mbf}{\mathbf}
\newcommand{\mc}{\mathcal}
\newcommand{\mmm}[9]{\left(\begin{array}{rrr}#1&#2&#3\\#4&#5&#6\\#7&#8&#9\end{array}\right)}
\newcommand{\mf}{\mathfrak}
\newcommand{\ol}{\overline}

% Math Operators/functions
\DeclareMathOperator{\Aut}{Aut}
\DeclareMathOperator{\End}{End}
\DeclareMathOperator{\Frob}{Frob}
\DeclareMathOperator{\cwe}{cwe}
\DeclareMathOperator{\id}{id}
\DeclareMathOperator{\mult}{mult}
\DeclareMathOperator{\we}{we}
\DeclareMathOperator{\wt}{wt}
\begin{document}
Let $\mf{h}$ be an abelian Lie algebra, $V$ a vector space and
$\rho\colon\mf{h}\to\End V$ a representation. Then the representation
is said to be \emph{diagonalisable}, if $V$ can be written as a direct
sum
\begin{equation*}
V=\DirectSum_{\lambda\in\mf{h}^*}V_\lambda
\end{equation*}
where $\mf{h}^*$ is the dual space of $\mf{h}$ and
\begin{equation*}
V_\lambda=\{v\in V\mid\rho(h)v=\lambda(h)v\text{ for all
}h\in\mf{h}\}.
\end{equation*}

Now let $\mf{g}$ be a semi-simple Lie algebra. Fix a Cartan subalgebra
$\mf{h}$, then $\mf{h}$ is abelian. Let $\rho\colon\mf{g}\to\End
V$ be a representation whose restriction to $\mf{h}$ is
diagonalisable. Then for any $\lambda\in\mf{h}^*$, the space
$V_\lambda$ is the \emph{weight space} of $\lambda$ with respect to
$\rho$. The \emph{multiplicity} of
$\lambda$ with respect to $\rho$ is the dimension of $V_\lambda$:
\begin{equation*}
\mult_\rho(\lambda):=\dim V_\lambda.
\end{equation*}
If the multiplicity of $\lambda$ is greater than zero, then $\lambda$
is called a \emph{weight} of the representation $\rho$.

A representation of a semi-simple Lie algebra is determined by the
multiplicities of its weights.
%%%%%
%%%%%
\end{document}
