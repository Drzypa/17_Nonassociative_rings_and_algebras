\documentclass[12pt]{article}
\usepackage{pmmeta}
\pmcanonicalname{ClassificationOfIndecomposableRootSystems}
\pmcreated{2013-03-22 15:28:56}
\pmmodified{2013-03-22 15:28:56}
\pmowner{rmilson}{146}
\pmmodifier{rmilson}{146}
\pmtitle{classification of indecomposable root systems}
\pmrecord{6}{37337}
\pmprivacy{1}
\pmauthor{rmilson}{146}
\pmtype{Result}
\pmcomment{trigger rebuild}
\pmclassification{msc}{17B20}

\endmetadata

\usepackage{amsmath}
\usepackage{amsfonts}
\usepackage{amssymb}

\usepackage{graphicx}

\newcommand{\Cset}{\mathbb{C}}
\newcommand{\Eset}{\mathbf{E}}
\newcommand{\Rset}{\mathbb{R}}
\begin{document}
There are four infinite families of indecomposable root systems :
\begin{align*}
A_n &=\{ \pm e_i\mp e_j \colon 1\leq i<j\leq n+1\};\\
B_n &=\{\pm e_i\pm e_j \colon 1\leq i<j\leq n\}\cup \{ \pm e_i \colon
1\leq i\leq n\};\\
C_n &=\{\pm e_i\pm e_j \colon 1\leq i<j\leq n\}\cup\{\pm 2e_i \colon
1\leq i\leq n\};\\
D_n &=\{\pm e_i\pm e_j \colon 1\leq i<j\leq n\}
\end{align*}

The subscript on the name of the root system is the dimension of $\Eset$,
the ambient Euclidean space containing the root system. In the case of
$A_n$, the ambient $\Eset$ is the $n$-dimensional subspace perpendicular
to $\sum_{i=1}^n e_i$.  In the other 3 cases, $\Eset=\Rset^n$.
Throughout, we endow $\Rset^n$ with the standard Euclidean inner
product, and let $e_i,\; 1\leq i\leq n$ denote the standard basis.

As well, there are 5 exceptional, crystallographic root systems:
\begin{align*}
G_2 &=A_3 \cup \left\{ \pm\frac{1}{3}( 2 e_1 -e_2-e_3),
\pm\frac{1}{3}( - e_1 +2e_2-e_3),\pm\frac{1}{3}( - e_1 -e_2+2e_3)\right\};\\
F_4&=B_4 \cup \left\{ \frac{1}{2}(\pm e_1\pm e_2\pm e_3 \pm e_4)\right\};\\
E_6&= A_6 \cup \left\{ \pm(e_7-e_8) \} \cup \{ \frac{1}{2}(
\sum_{i=1}^6 (\pm e_i) \pm (e_7 - e_8)) \colon
\text{4 minus signs}\right\};\\
E_7&=A_8\cup \left\{\sum_{i=1}^8 (\pm e_i) \colon
\text{4 minus signs}\right\};\\
E_8&=D_8\cup \left\{\sum_{i=1}^8 (\pm e_i) \colon
\text{even number of minus signs}\right\}.
\end{align*}

The following table indicates the cardinality of and the Lie algebras
and Dynkin diagrams corresponding to the above root systems.
  \begin{figure}[h]
    \centering
    \includegraphics{rootsys-diag}
    \caption{Irreducible root systems and simple Lie algebras.}
  \end{figure}
%%%%%
%%%%%
\end{document}
