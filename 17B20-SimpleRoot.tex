\documentclass[12pt]{article}
\usepackage{pmmeta}
\pmcanonicalname{SimpleRoot}
\pmcreated{2013-03-22 13:11:49}
\pmmodified{2013-03-22 13:11:49}
\pmowner{mathcam}{2727}
\pmmodifier{mathcam}{2727}
\pmtitle{simple root}
\pmrecord{6}{33657}
\pmprivacy{1}
\pmauthor{mathcam}{2727}
\pmtype{Definition}
\pmcomment{trigger rebuild}
\pmclassification{msc}{17B20}
\pmdefines{base}

% this is the default PlanetMath preamble.  as your knowledge
% of TeX increases, you will probably want to edit this, but
% it should be fine as is for beginners.

% almost certainly you want these
\usepackage{amssymb}
\usepackage{amsmath}
\usepackage{amsfonts}

% used for TeXing text within eps files
%\usepackage{psfrag}
% need this for including graphics (\includegraphics)
%\usepackage{graphicx}
% for neatly defining theorems and propositions
%\usepackage{amsthm}
% making logically defined graphics
%%%\usepackage{xypic}

% there are many more packages, add them here as you need them

% define commands here
\begin{document}
\PMlinkescapeword{simple}
\PMlinkescapephrase{vector space}

Let $R\subseteq E$ be a root system, with $E$ a Euclidean \PMlinkname{vector space}{VectorSpace}.  If $R^+$ is a set of
positive roots, then a root is called {\em simple} if it is positive, and not the sum of any two
positive roots.  The simple roots form a basis of the vector space $E$, and any positive root
is a positive integer linear combination of simple roots.

A set of roots which is simple with respect to some choice of a set of positive roots is called a 
{\em base}.  The Weyl group of the root system acts simply transitively on the set of bases.
%%%%%
%%%%%
\end{document}
