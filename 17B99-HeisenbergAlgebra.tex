\documentclass[12pt]{article}
\usepackage{pmmeta}
\pmcanonicalname{HeisenbergAlgebra}
\pmcreated{2013-03-22 15:24:57}
\pmmodified{2013-03-22 15:24:57}
\pmowner{GrafZahl}{9234}
\pmmodifier{GrafZahl}{9234}
\pmtitle{Heisenberg algebra}
\pmrecord{6}{37259}
\pmprivacy{1}
\pmauthor{GrafZahl}{9234}
\pmtype{Definition}
\pmcomment{trigger rebuild}
\pmclassification{msc}{17B99}
\pmrelated{WeylAlgebra}
\pmdefines{central element}

% this is the default PlanetMath preamble.  as your knowledge
% of TeX increases, you will probably want to edit this, but
% it should be fine as is for beginners.

% almost certainly you want these
\usepackage{amssymb}
\usepackage{amsmath}
\usepackage{amsfonts}

% used for TeXing text within eps files
%\usepackage{psfrag}
% need this for including graphics (\includegraphics)
%\usepackage{graphicx}
% for neatly defining theorems and propositions
\usepackage{amsthm}
% making logically defined graphics
%%%\usepackage{xypic}

% there are many more packages, add them here as you need them

% define commands here
\newcommand{\<}{\langle}
\renewcommand{\>}{\rangle}
\newcommand{\Bigcup}{\bigcup\limits}
\newcommand{\DirectSum}{\bigoplus\limits}
\newcommand{\Prod}{\prod\limits}
\newcommand{\Sum}{\sum\limits}
\newcommand{\h}{\widehat}
\newcommand{\mbb}{\mathbb}
\newcommand{\mbf}{\mathbf}
\newcommand{\mc}{\mathcal}
\newcommand{\mmm}[9]{\left(\begin{array}{rrr}#1&#2&#3\\#4&#5&#6\\#7&#8&#9\end{array}\right)}
\newcommand{\mf}{\mathfrak}
\newcommand{\ol}{\overline}

% Math Operators/functions
\DeclareMathOperator{\Aut}{Aut}
\DeclareMathOperator{\End}{End}
\DeclareMathOperator{\Frob}{Frob}
\DeclareMathOperator{\cwe}{cwe}
\DeclareMathOperator{\id}{id}
\DeclareMathOperator{\mult}{mult}
\DeclareMathOperator{\we}{we}
\DeclareMathOperator{\wt}{wt}
\begin{document}
\PMlinkescapeword{index}
\PMlinkescapeword{module}
\PMlinkescapeword{theory}
\PMlinkescapeword{vertex}
Let $R$ be a commutative ring. Let $M$ be a \PMlinkid{module}{5420}
over $R$ \PMlinkid{freely generated}{5420} by two sets $\{P_i\}_{i\in
  I}$ and $\{Q_i\}_{i\in I}$, where
$I$ is an index set, and a further element $c$. Define a product
$[\cdot,\cdot]\colon M\times M\to M$ by bilinear extension by setting
\begin{gather*}
[c,c]=[c,P_i]=[P_i,c]=[c,Q_i]=[Q_i,c]=[P_i,P_j]=[Q_i,Q_j]=0\text{ for
all }i,j\in I,\\
[P_i,Q_j]=[Q_i,P_j]=0\text{ for all distinct }i,j\in I,\\
[P_i,Q_i]=-[Q_i,P_i]=c\text{ for all }i\in I.
\end{gather*}
The module $M$ together with this product is called a \emph{Heisenberg
algebra}. The element $c$ is called the \emph{central element}.

It is easy to see that the product $[\cdot,\cdot]$ also fulfills the
Jacobi identity, so a Heisenberg algebra is actually a Lie algebra of
rank $|I|+1$ (opposed to the rank of $M$ as free module, which is
$2|I|+1$) with one-dimensional center generated by $c$.

Heisenberg algebras arise in quantum mechanics with $R=\mbb{C}$ and
typically $I=\{1,2,3\}$, but also in the theory of vertex
\PMlinkescapetext{algebras} with $I=\mbb{Z}$.

In the case where $R$ is a field, the Heisenberg algebra is related to
a Weyl algebra: let $U$ be the universal enveloping algebra of
$M$, then the quotient $U/\<c-1\>$ is isomorphic to the $|I|$-th Weyl
algebra over $R$.
%%%%%
%%%%%
\end{document}
