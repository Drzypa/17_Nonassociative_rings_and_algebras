\documentclass[12pt]{article}
\usepackage{pmmeta}
\pmcanonicalname{SolvableLieAlgebra}
\pmcreated{2013-03-22 12:41:06}
\pmmodified{2013-03-22 12:41:06}
\pmowner{djao}{24}
\pmmodifier{djao}{24}
\pmtitle{solvable Lie algebra}
\pmrecord{4}{32964}
\pmprivacy{1}
\pmauthor{djao}{24}
\pmtype{Definition}
\pmcomment{trigger rebuild}
\pmclassification{msc}{17B30}
\pmdefines{nilpotent Lie algebra}
\pmdefines{solvable}
\pmdefines{nilpotent}
\pmdefines{lower central series}
\pmdefines{upper central series}

% this is the default PlanetMath preamble.  as your knowledge
% of TeX increases, you will probably want to edit this, but
% it should be fine as is for beginners.

% almost certainly you want these
\usepackage{amssymb}
\usepackage{amsmath}
\usepackage{amsfonts}

% used for TeXing text within eps files
%\usepackage{psfrag}
% need this for including graphics (\includegraphics)
%\usepackage{graphicx}
% for neatly defining theorems and propositions
%\usepackage{amsthm}
% making logically defined graphics
%%%\usepackage{xypic} 

% there are many more packages, add them here as you need them

% define commands here
\newcommand{\g}{\mathfrak{g}}
\newcommand{\D}{\mathcal{D}}
\newcommand{\h}{\mathfrak{h}}
\begin{document}
Let $\g$ be a Lie algebra. The {\em lower central series} of $\g$ is the filtration of subalgebras
$$
\D_1 \g \supset \D_2 \g \supset \D_3 \g \supset \cdots \supset \D_k \g \supset \cdots
$$
of $\g$, inductively defined for every natural number $k$ as follows:
\begin{eqnarray*}
\D_1 \g & := & [\g,\g] \\
\D_k \g & := & [\g, \D_{k-1} \g]
\end{eqnarray*}

The {\em upper central series} of $\g$ is the filtration
$$
\D^1 \g \supset \D^2 \g \supset \D^3 \g \supset \cdots \supset \D^k \g \supset \cdots
$$
defined inductively by
\begin{eqnarray*}
\D^1 \g & := & [\g,\g] \\
\D^k \g & := & [\D^{k-1} \g, \D^{k-1} \g]
\end{eqnarray*}

In fact both $\D^k \g$ and $\D_k \g$ are ideals of $\g$, and $\D^k \g \subset \D_k \g$ for all $k$. The Lie algebra $\g$ is defined to be {\em nilpotent} if $\D_k \g = 0$ for some $k \in \mathbb{N}$, and {\em solvable} if $\D^k \g = 0$ for some $k \in \mathbb{N}$.

A subalgebra $\h$ of $\g$ is said to be {\em nilpotent} or {\em solvable} if $\h$ is nilpotent or solvable when considered as a Lie algebra in its own right. The terms may also be applied to ideals of $\g$, since every ideal of $\g$ is also a subalgebra.
%%%%%
%%%%%
\end{document}
