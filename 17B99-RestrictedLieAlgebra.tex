\documentclass[12pt]{article}
\usepackage{pmmeta}
\pmcanonicalname{RestrictedLieAlgebra}
\pmcreated{2013-03-22 16:26:48}
\pmmodified{2013-03-22 16:26:48}
\pmowner{Algeboy}{12884}
\pmmodifier{Algeboy}{12884}
\pmtitle{restricted Lie algebra}
\pmrecord{7}{38602}
\pmprivacy{1}
\pmauthor{Algeboy}{12884}
\pmtype{Definition}
\pmcomment{trigger rebuild}
\pmclassification{msc}{17B99}
%\pmkeywords{restricted Lie algebra}
%\pmkeywords{power map}
\pmrelated{ModularTheory}
\pmdefines{restricted Lie algebra}
\pmdefines{modular Lie algebra}
\pmdefines{restrictable Lie algebra}

\usepackage{latexsym}
\usepackage{amssymb}
\usepackage{amsmath}
\usepackage{amsfonts}
\usepackage{amsthm}

\DeclareMathOperator{\End}{End}
\DeclareMathOperator{\ad}{ad~}

%%\usepackage{xypic}

%-----------------------------------------------------

%       Standard theoremlike environments.

%       Stolen directly from AMSLaTeX sample

%-----------------------------------------------------

%% \theoremstyle{plain} %% This is the default

\newtheorem{thm}{Theorem}

\newtheorem{coro}[thm]{Corollary}

\newtheorem{lem}[thm]{Lemma}

\newtheorem{lemma}[thm]{Lemma}

\newtheorem{prop}[thm]{Proposition}

\newtheorem{conjecture}[thm]{Conjecture}

\newtheorem{conj}[thm]{Conjecture}

\newtheorem{defn}[thm]{Definition}

\newtheorem{remark}[thm]{Remark}

\newtheorem{ex}[thm]{Example}



%\countstyle[equation]{thm}



%--------------------------------------------------

%       Item references.

%--------------------------------------------------


\newcommand{\exref}[1]{Example-\ref{#1}}

\newcommand{\thmref}[1]{Theorem-\ref{#1}}

\newcommand{\defref}[1]{Definition-\ref{#1}}

\newcommand{\eqnref}[1]{(\ref{#1})}

\newcommand{\secref}[1]{Section-\ref{#1}}

\newcommand{\lemref}[1]{Lemma-\ref{#1}}

\newcommand{\propref}[1]{Prop\-o\-si\-tion-\ref{#1}}

\newcommand{\corref}[1]{Cor\-ol\-lary-\ref{#1}}

\newcommand{\figref}[1]{Fig\-ure-\ref{#1}}

\newcommand{\conjref}[1]{Conjecture-\ref{#1}}


% Normal subgroup or equal.

\providecommand{\normaleq}{\unlhd}

% Normal subgroup.

\providecommand{\normal}{\lhd}

\providecommand{\rnormal}{\rhd}
% Divides, does not divide.

\providecommand{\divides}{\mid}

\providecommand{\ndivides}{\nmid}


\providecommand{\union}{\cup}

\providecommand{\bigunion}{\bigcup}

\providecommand{\intersect}{\cap}

\providecommand{\bigintersect}{\bigcap}










\begin{document}
\begin{defn}
Given a modular Lie algebra $L$, that is, a Lie algebra defined over a field of
positive characteristic $p$, then we say $L$ is \emph{restricted} if there
exists a \emph{power map} $[p]:L\rightarrow L$ satisfying
\begin{enumerate}
\item $(\ad a)^p=\ad (a^{[p]})$ for all $a\in L$ (where $(\ad a)^p=\ad a\cdots \ad a$ in the associative product of linear transformations in $\mathfrak{gl}_k V=\End_k V$.)
\item For every $\alpha\in k$ and $a\in L$ then $(\alpha a)^{[p]}=\alpha^p a^{[p]}$.
\item For all $a,b\in L$,
\[(a+b)^{[p]}=a^{[p]}+b^{[p]}+\sum_{i=1}^{p-1} s_i(a,b)\]
where the terms $s_i(a,b)$ are determined by the formula
\[(\ad (a\otimes x+b\otimes 1))^{p-1} (a\otimes 1)
=\sum_{i=1}^{p-1} i s_i(a,b)\otimes x^{i-1}\]
in $L\otimes_k k[x]$.
\end{enumerate} 
\end{defn}

The definition and terminology of a restricted Lie algebra was developed by 
N. Jacobson as a method to mimic the properties of Lie algebras of characteristic 0.  The usual methods of using minimal polynomials to establish
the Jordan normal form of a transform fail in positive characteristic as
certain polynomials become inseparable or reducible.  Thus one cannot
simply establish the typical nilpotent+semisimple decomposition of elements. 

However, given a linear Lie algebra (subalgebra of $\mathfrak{gl}_k V$) over a field $k$ of positive characteristic $p$ then the map $x\mapsto x^p$ on the matrices captures many of the properties of the field.  For example, a diagonal
matrix $D$ with entries in $\mathbb{Z}_p$ satisfies $D^p=D$ simply because 
the power map $p$ is a field automorphism.  Thus in various ways the power
map captures the requirements of semisimple and toral elements of Lie
algebras.  By modifying the definitions of semisimple, toral, and nilpotent
elements to use the power maps of the field, Jacobson and others were able to
reproduce much of the classical theory of Lie algebras for restricted Lie algebras.

The definition given above reflects the abstract requirements for a restricted Lie algebra.  However an important observation is that given a linear Lie algebra $L$ over a field of characteristic $p$, then the usual associative power map serves as a power map for establishing a restricted Lie algebra.  The only added requirement is that $L^p\subseteq L$.

\begin{defn}
A Lie algebra is said to be \emph{restrictable} if it can be given 
a power mapping which makes it a restricted Lie algebra.
\end{defn}

\begin{remark}
It is generally not true that a restrictable Lie algebra has a unique 
power mapping.  Notice that the definition of a power mapping relates
the power mapping to the linear power mapping of the adjoint representation.
This suggests (correctly) that power maps can be defined in various ways
but agree modulo the center of the Lie algebra.
\end{remark}

Jacobson, Nathan \emph{Lie algebras}, Interscience Tracts in Pure and Applied Mathematics, No. 10, Interscience Publishers (a division of John Wiley \& Sons),
              New York-London, 1962.

Strade, Helmut and Farnsteiner, Rolf \emph{Modular {L}ie algebras and their representations}, Monographs and Textbooks in Pure and Applied Mathematics,
vol. 116, Marcel Dekker Inc., New York, 1988.

%%%%%
%%%%%
\end{document}
